\documentclass[12pt]{article}

% Packages for math and formatting
\usepackage[utf8]{inputenc}
\usepackage{amsmath, amssymb, amsthm}
\usepackage{geometry}
\usepackage{enumitem}
\usepackage{hyperref}
\usepackage{mathtools}
\usepackage{epigraph}
\usepackage{tikz}
\usetikzlibrary{arrows.meta, decorations.pathmorphing, positioning}

% Page setup
\geometry{margin=1in}

% Theorem-like environments
\newtheorem{theorem}{Theorem}[section]
\newtheorem{proposition}[theorem]{Proposition}
\newtheorem{lemma}[theorem]{Lemma}
\newtheorem{corollary}[theorem]{Corollary}
\theoremstyle{definition}
\newtheorem{definition}[theorem]{Definition}
\newtheorem{problem}[theorem]{Problem}
\theoremstyle{remark}
\newtheorem{remark}[theorem]{Remark}
\newtheorem{example}[theorem]{Example}

% Custom commands
\newcommand{\R}{\mathbb{R}}
\newcommand{\N}{\mathbb{N}}
\newcommand{\Q}{\mathbb{Q}}
\newcommand{\Z}{\mathbb{Z}}
\newcommand{\C}{\mathbb{C}}
\newcommand{\PP}{\mathbb{P}}
\newcommand{\D}{\mathbb{D}}
\newcommand{\HH}{\mathbb{H}} % upper half-plane

% Operator names
\DeclareMathOperator{\Aut}{Aut}
\DeclareMathOperator{\FractLin}{FractLin}
\DeclareMathOperator{\ord}{ord}
\DeclareMathOperator{\Res}{Res}
\DeclareMathOperator{\SL}{SL}

\title{Elliptic Functions, Part 3}
\author{Joe}
\date{October 22, 2025}

\begin{document}
	
	\maketitle
	
	\epigraph{The teaching of maths is to make everything obvious.}{\textit{Prof. MOK Ngaiming}}
	
	\tableofcontents
	\vspace{1em}
	
\section{The Weierstrass Problem on \(\mathbb{C}/L\)}


Let us consider Weierstrass data
\[
\{(x_k, n_k)\}_{k=1}^s,
\]
where each \(x_k\) is a distinct point on \(X = \mathbb{C}/L\), and \(n_k \in \mathbb{Z}\) is an integer (possibly positive or negative) associated to \(x_k\).

To lift these points to the universal covering space, choose representatives
\[
a_k \in \mathbb{C} \quad \text{such that} \quad \pi(a_k) = x_k,
\]
where \(\pi : \mathbb{C} \to X = \mathbb{C}/L\) is the canonical projection.  

In the simpler case, we assume the following condition holds:
\[
\sum_{k=1}^{s} n_k a_k = 0.
\]




\underline{Observation:}
Recall that the shifted Weierstrass sigma function,
\[
\sigma_c(z) := \sigma(z + c),
\]
has a \emph{simple zero at} \(z = -c\) and no other zeros.  



So a possible candidate for the desired function is
\[
f(z) = \prod_{k=1}^{s} \sigma(z - a_k)^{n_k}.
\]
Each factor \(\sigma(z - a_k)\) contributes a zero of order \(n_k\) at \(z = a_k\) (if \(n_k > 0\)), or a pole of order \(|n_k|\) (if \(n_k < 0\)).

We have
\[
\operatorname{ord}_{a_k}(f) = n_k, \qquad k = 1, 2, \ldots, s,
\]
and for any \(a \not\equiv a_k \pmod{L}\) for all \(k\),
\[
\operatorname{ord}_{a}(f) = 0.
\]

It remains to check whether \(f\) is an elliptic function with respect to the lattice \(L\).
Recall that the sigma function satisfies
\[
\sigma(z+\omega) = \exp(A_\omega z + B_\omega)\, \sigma(z),
\qquad \forall\, \omega \in L.
\]
Therefore,
\[
f(z+\omega)
= \prod_{k=1}^{s} \sigma(z+\omega - a_k)^{n_k},
\qquad
f(z)
= \prod_{k=1}^{s} \sigma(z - a_k)^{n_k}.
\]


So we have 
\[
\frac{f(z+\omega)}{f(z)}
= \prod_{k=1}^{s} 
\left(
\frac{\sigma((z-a_k)+\omega)}{\sigma(z-a_k)}
\right)^{n_k}.
\]
By applying the transformation formula 
\(\sigma(z+\omega)=\exp(A_\omega z+B_\omega)\sigma(z)\), we obtain
\[
\frac{f(z+\omega)}{f(z)}
= \prod_{k=1}^{s}
\bigl(\exp(A_\omega(z-a_k)+B_\omega)\bigr)^{n_k}
= \prod_{k=1}^{s}
\exp\!\bigl(n_k(A_\omega(z-a_k)+B_\omega)\bigr).
\]
Expanding inside the product,
\[
\frac{f(z+\omega)}{f(z)}
= \prod_{k=1}^{s}
\exp\!\bigl(n_kA_\omega z - n_kA_\omega a_k + n_kB_\omega\bigr)
= \exp\!\left(
\sum_{k=1}^{s}(n_kA_\omega z - n_kA_\omega a_k + n_kB_\omega)
\right).
\]
Combine terms in the exponent:
\[
\frac{f(z+\omega)}{f(z)}
= \exp\!\left(
A_\omega z \sum_{k=1}^{s} n_k
- A_\omega \sum_{k=1}^{s} n_k a_k
+ B_\omega \sum_{k=1}^{s} n_k
\right)
= \exp\!\left((A_\omega z+B_\omega)\sum_{k=1}^{s}n_k\right)
\exp\!\left(-A_\omega\sum_{k=1}^{s}n_k a_k\right).
\]





If the coefficients \(n_k\) satisfy
\[
\sum_{k=1}^{s} n_k = 0
\quad \text{and} \quad
\sum_{k=1}^{s} n_k a_k = 0,
\]
then the exponent equals zero, and therefore
\[
\frac{f(z+\omega)}{f(z)} = 1
\quad \text{for all } \omega \in L.
\]
Thus \(f\) is an elliptic function with respect to \(L\).

Next, we show that it suffices to require the weaker condition
\[
\sum_{k=1}^{s} n_k a_k \equiv 0 \pmod{L},
\]
instead of the strict equality \(\sum_{k=1}^{s} n_k a_k = 0.\)

Recall that we have \(\pi: \mathbb{C} \to \mathbb{C}/L\), and the points \(a_k\) are chosen so that 
\(\pi(a_k) = x_k\). We can replace each \(a_k\) by 
\[
a_k' = a_k + w_0, \qquad w_0 \in L.
\]

\underline{Difficulty:} If we replace \((x_s, a_s)\) by \((x_s, a_s + w_0)\) and define
\[
h(z) = \prod_{k=1}^{s-1} \sigma(z - a_k)^{n_k} \cdot \sigma(z - (a_s + w_0))^{n_s},
\]
then
\[
\frac{h(z + \omega)}{h(z)} 
= \exp\!\left(-\Big(\sum_{k=1}^s n_k a_k' \Big) A_\omega \right),
\]
where \(a_k' = a_k\) for \(k \neq s\), and \(a_s' = a_s + w_0\).

Hence,
\[
\frac{h(z + \omega)}{h(z)} 
= \exp\!\left( -\Big(\sum_{k=1}^s n_k a_k + n_s w_0 \Big) A_\omega \right).
\]
Let
\[
\mu = \sum_{k=1}^s n_k a_k.
\]
By assumption, we know that \(\mu \in L\).

The Weierstrass periodicity condition will be solved by setting
\[
\mu + n_s w_0 = 0, 
\quad \text{i.e.} \quad 
w_0 = -\frac{\mu}{n_s} \in \tfrac{1}{n_s} L.
\]

The difficulty arises because \(n_s\) may not be \(\pm 1\).
A solution in the general case is obtained as follows (we may assume \(n_s \ge 1\)):
\[
h(z) 
= \prod_{k=1}^{s-1} \sigma(z - a_k)^{n_k}
\cdot \sigma(z - a_s)^{n_s - 1}
\cdot \sigma(z - a_s'),
\]
where \(a_s' = a_s-w_0\) for some \(w_0\in L\) to be determined.

Recall that
\[
\sigma(z + \omega) = e^{A_\omega z + B_\omega} \sigma(z).
\]
Then
\[
\sigma(z - a_s') 
= \sigma\bigl(z - (a_s - w_0)\bigr) 
= \sigma\bigl((z - a_s) + w_0\bigr)
= e^{A_{w_0}(z - a_s) + B_{w_0}} \sigma(z - a_s),
\]
and
\[
\frac{\sigma(z - a_s' + \omega)}{\sigma(z - a_s')} 
= e^{A_\omega (z - a_s') + B_\omega}.
\]

Note that
\[
h(z) = 
\frac{f(z)}{\sigma(z - a_s)} \, \sigma(z - a_s').
\]
Hence
\[
\frac{h(z + \omega)}{h(z)}
= \frac{f(z + \omega)}{f(z)}
\cdot
\frac{\sigma(z - a_s )}{\sigma(z - a_s+\omega)}
\cdot
\frac{\sigma(z - a_s' + \omega)}{\sigma(z - a_s')}.
\]
Using the quasi-periodicity of \(\sigma\),
\[
\frac{\sigma(z - a_s + \omega)}{\sigma(z - a_s)} 
= e^{A_\omega (z - a_s) +B_\omega}, 
\qquad
\frac{\sigma(z - a_s' + \omega)}{\sigma(z - a_s')} 
= e^{A_\omega (z - a_s') + B_\omega},
\]
we obtain
\[
\frac{h(z + \omega)}{h(z)} 
= \frac{f(z + \omega)}{f(z)} 
e^{-A_\omega (z - a_s) - B_\omega}
e^{A_\omega (z - a_s') + B_\omega}
= \frac{f(z + \omega)}{f(z)}
\exp\!\big(A_\omega (a_s - a_s')\big).
\]


Since
\[
\frac{f(z + \omega)}{f(z)}
= \exp\!\left(-A_\omega \sum_{k=1}^s n_k a_k\right),
\]
we conclude that
\[
\frac{h(z + \omega)}{h(z)}
= \exp\!\left(-A_\omega \mu + A_\omega (a_s - a_s')\right)
= \exp\!\big(-A_\omega \mu + A_\omega w_0\big).
\]
Thus, periodicity holds (\(\frac{h(z+\omega)}{h(z)}=1\)) when \(w_0 = \mu.\)

\section{The Field of Meromorphic Functions on \( \mathbb{C}/L \)}

We have already established the necessary and sufficient conditions for solving both 
the Mittag–Leffler (ML) problem and the Weierstrass problem (WP) .

In this section, we describe the function field \(\mathcal{M}(X)\) explicitly.

Recall that for the Eisenstein series \(E_k\) with \(k \ge 3\), we have
\[
\wp'(z) = -2E_3(z).
\]
As will be seen, the functions \(\wp\) and \(\wp'\) are the most important ones on \(X\);
every meromorphic function on \(X\) will turn out to be expressible in terms of them.

We will prove the following:
\begin{enumerate}
	\item \(\wp\) and \(\wp'\) are algebraically related;
	\item \(\wp\) and \(\wp'\) generate \(\mathcal{M}(X)\) as a field.
\end{enumerate}


\subsection{An Algebraic Relation Between \(\wp\) and \(\wp'\)}

What tool can we use to prove that an elliptic function \(f \equiv 0\)?
The basic principle is the \emph{maximum principle}, which implies that any holomorphic
elliptic function must be constant.

\medskip

\noindent
\textbf{Basic technique:} use the Laurent series expansion at the poles.
If the Laurent expansion of \(f\) at each possible pole actually reduces to a Taylor series
(i.e., all terms of negative degree vanish) and the constant term at one possible pole is zero,
then \(f \equiv 0\).

\medskip

\noindent
\textbf{Objective:}
to find an algebraic relation between \(\wp\) and \(\wp'\), the simpler the better.

\medskip

\noindent
\underline{Observation:}
\(\wp\) has a double pole at each lattice point \(\omega \in L\) and no other poles,
while \(\wp'\) has a triple pole at each \(\omega \in L\) and no other poles. 
Thus,
\[
\operatorname{ord}_\omega(\wp^3) = 6, 
\qquad
\operatorname{ord}_\omega((\wp')^2) = 6.
\]
This suggests a connection between \(\wp^3\) and \((\wp')^2\).

\medskip

Recall that
\[
\wp(z) 
= \frac{1}{z^2}
+ \sum_{\omega \in L^*} 
\left( \frac{1}{(z+\omega)^2} - \frac{1}{\omega^2} \right),
\qquad
E_3(z) = \sum_{\omega \in L} \frac{1}{(z+\omega)^3},
\qquad
\wp'(z) = -2 E_3(z).
\]

Since \(\wp\) is even and \(\wp'\) is odd, we can write their local expansions near \(z=0\) as
\[
\wp(z) = \frac{1}{z^2} + a_2 z^2 + a_4 z^4 + \cdots,
\qquad
\wp'(z) = -\frac{2}{z^3} + b_1 z + b_3 z^3 + \cdots.
\]

Therefore,
\[
\wp^3(z) = \frac{1}{z^6} + \cdots, 
\qquad
(\wp'(z))^2 = \frac{4}{z^6} + \cdots,
\]
which leads us to \emph{guess} that
\[
(\wp'(z))^2 = 4\,\wp^3(z) + \cdots \;?
\]
We will make this relation precise below.




We begin by expanding \((\wp'(z))^2\) and \(\wp^3(z)\) using their Laurent series near \(z = 0\).

\medskip
\noindent
Since
\[
\wp'(z) = -\frac{2}{z^3} + b_1 z + b_3 z^3 + \cdots,
\]
we have
\[
\begin{aligned}
	(\wp'(z))^2 
	&= \left(-\frac{2}{z^3} + b_1 z + b_3 z^3 + \cdots \right)^2 \\[4pt]
	&= \frac{4}{z^6} + 2\!\left(-\frac{2}{z^3}\right)\!(b_1 z + b_3 z^3 + \cdots)
	+ \underbrace{(b_1 z + b_3 z^3 + \cdots)^2}_{\text{Taylor expansion, constant term } 0} \\[4pt]
	&= \frac{4}{z^6} - \frac{4b_1}{z^2} - 4b_3 + \cdots.
\end{aligned}
\]

\medskip
\noindent
Next, for \(\wp(z)\), recall
\[
\wp(z) = \frac{1}{z^2} + a_2 z^2 + a_4 z^4 + \cdots.
\]
Using \((x + y)^3 = x^3 + 3x^2 y + 3x y^2 + y^3\), we get
\[
\begin{aligned}
	\wp^3(z)
	&= \left( \frac{1}{z^2} + (a_2 z^2 + a_4 z^4 + \cdots) \right)^3 \\[2pt]
	&= \frac{1}{z^6} + 3\frac{1}{z^4}(a_2 z^2 + a_4 z^4 + \cdots)
	+ \underbrace{\frac{3}{z^2}(a_2 z^2 + a_4 z^4 + \cdots)^2}_{\text{Taylor expansion, constant term } 0} \\[4pt]
	&= \frac{1}{z^6} + \frac{3a_2}{z^2} + 3a_4 + \cdots.
\end{aligned}
\]
Hence,
\[
4\wp^3(z) = \frac{4}{z^6} + \frac{12a_2}{z^2} + 12a_4 + \cdots.
\]

\medskip
\noindent
Subtracting gives
\[
(\wp'(z))^2 - 4\wp^3(z)
= \frac{-4b_1 - 12a_2}{z^2} + (-4b_3 - 12a_4) + \cdots.
\]
Define
\[
g_2 = -4b_1 - 12a_2, 
\qquad 
g_3 = -4b_3 - 12a_4,
\]
and set
\[
f(z) = (\wp'(z))^2 - 4\wp^3(z) - g_2 \wp(z) - g_3.
\]
Then, the principal part of \(f\) at \(z = 0\) vanishes, and the constant term there is zero. 
Since \(f\) is elliptic, its principal part at every lattice point \(\omega \in L\) also vanishes, 
and its constant term is zero.  
Hence, by the basic principle for elliptic functions, we have \(f \equiv 0\).

\begin{theorem}
	Let \(L = \mathbb{Z}\omega_1 + \mathbb{Z}\omega_2\) be a lattice.
	Then there exist constants \(g_2, g_3 \in \mathbb{C}\) (depending only on \(L\)) 
	such that the Weierstrass functions \(\wp\) and \(\wp'\) satisfy the identity
	\[
	(\wp'(z))^2 = 4\wp^3(z) - g_2 \wp(z) - g_3
	\quad \text{for all } z \in \mathbb{C}.
	\]
\end{theorem}
	
	
	
\subsection{The Generation of \(\mathcal{M}(X)\) by \(\wp\) and \(\wp'\)}

We now turn to the explicit description of the field of meromorphic functions on 
$
X = \mathbb{C}/L.
$

\begin{theorem}
 \(\mathcal{M}(X)\) is generated by the Weierstrass functions
	\(\wp\) and \(\wp'\).
\end{theorem}

\noindent
\underline{Reduction:}
Let \(f\) be an elliptic function. 
We decompose \(f\) into its even and odd parts:
\[
f(z) = \frac{f(z) + f(-z)}{2} + \frac{f(z) - f(-z)}{2} 
= f^{\mathrm{even}}(z) + f^{\mathrm{odd}}(z).
\]
Define
\[
\mathcal{M}^{\mathrm{even}}(X)
= \{ f \in \mathcal{M}(X) \mid f(-z) = f(z) \}, 
\qquad
\mathcal{M}^{\mathrm{odd}}(X)
= \{ f \in \mathcal{M}(X) \mid f(-z) = -f(z) \}.
\]
Then we have \(\wp \in \mathcal{M}^{\mathrm{even}}(X)\) 
and \(\wp' \in \mathcal{M}^{\mathrm{odd}}(X)\).
Moreover, if \(h \in \mathcal{M}^{\mathrm{odd}}(X)\), then \(h\,\wp' \in \mathcal{M}^{\mathrm{even}}(X)\), 
since the product of two odd functions is even.

\begin{proposition}
	The field of even elliptic functions on $X = \C / L$ is generated over~$\C$ 
	by the Weierstrass function~$\wp$; that is,
	\[
	\mathcal{M}^{\mathrm{even}}(X)
	= \C(\wp)
	= \left\{
	\frac{P(\wp(z))}{Q(\wp(z))} 
	\ \Big| \ 
	P,Q \in \C[x], \ Q \not\equiv 0
	\right\}.
	\]
\end{proposition}

\begin{proof}
	Let $f \in \mathcal{M}^{\mathrm{even}}(X)$, so that $f(z)=f(-z)$. 
	We proceed in two stages.
	
	\paragraph{Construction of a meromorphic function holomorphic at the lattice points.}
	The Laurent expansion of $f$ near $z=0$ involves only even powers:
	\[
	f(z) = \sum_{m=k}^{\infty} a_{2m} z^{2m},
	\]
	since $f$ is even. Hence $\ord_0(f) = 2s$ for some $s\in\Z$.
	
	If $f$ has a pole of order $2n$ at $0$ (so $\ord_0(f) = -2n$), define
	\[
	h_0(z) = \frac{f(z)}{\wp(z)^n}.
	\]
	Since $\ord_0(\wp^n) = -2n$, we get $\ord_0(h_0) = 0$, so $h_0$ is holomorphic at $0$. 
	Because $h_0$ is also even and elliptic, its periodicity ensures 
	$h_0(z+\omega) = h_0(z)$ for all $\omega \in L$; 
	hence $h_0$ is holomorphic at every lattice point.
	
	Thus we have produced a meromorphic function on $\C$, 
	elliptic with respect to $L$, that is holomorphic at all lattice points.
	
	\paragraph{Determination of the function via the Weierstrass problem on $X$.}
	According to the Weierstrass problem on the torus $X$, 
	once we specify a divisor
	\[
	D = \sum_{a \in X} n_a [a], 
	\quad \text{with} \quad 
	\sum_{a} n_a = 0,
	\]
	there exists a meromorphic function on $X$ having zeros and poles exactly as prescribed by $D$, 
	unique up to multiplication by a nonzero constant.
	
	Because $f$ is even, its zeros and poles occur in symmetric patterns.
	We distinguish two cases:
	
	\medskip
	\noindent \textbf{Case 1.} When $a \not\equiv -a \pmod{L}$: 
	
	Consider the elliptic function $\wp(z) - \wp(a)$. 
	Since $\wp$ is even, $\wp(-a) = \wp(a)$. 
	Therefore, $\wp(z) - \wp(a)$ has simple zeros at $z = a$ and $z = -a$, 
	and a double pole at each lattice point $\omega \in L$.
	Specifically,
	\[
	\ord_0(\wp - \wp(a)) = -2, 
	\qquad 
	\ord_a(\wp - \wp(a)) = 
	\ord_{-a}(\wp - \wp(a)) = 1,
	\]
	and there are no other zeros or poles modulo $L$.
	
	For points $a_k$ with $2a_k \not\equiv 0 \pmod{L}$, 
	we have symmetric pairs $\{a_k,-a_k\}$ with equal orders:
	\[
	\ord_{a_k}(f) = \ord_{-a_k}(f) = n_k.
	\]
	
	\medskip
	\noindent \textbf{Case 2.} When $a \equiv -a \pmod{L}$:
	
	This occurs exactly when $2a \equiv 0 \pmod{L}$, 
	i.e.\ when $a$ is a \emph{half-period}. 
	The half-periods of the lattice $L = \Z\omega_1 + \Z\omega_2$ are
	\[
	0, \quad \frac{\omega_1}{2}, \quad \frac{\omega_2}{2}, \quad 
	\omega_3 := \frac{\omega_1 + \omega_2}{2}.
	\]
	
	At each half-period $\omega_i/2$ (where $i=1,2,3$), 
	we have $\wp'(\omega_i/2) = 0$, so
	the function $\wp(z) - \wp(\omega_i/2)$ has a double zero at $z = \omega_i/2$.
	The orders at half-periods must be even:
	\[
	\ord_{\frac{\omega_i}{2}}(f) = 2t_i, \quad t_i\in\Z.
	\]
	
	\medskip
	\noindent
	Combining both cases, define
	\[
	h(z)
	= \prod_{k=1}^{s} \bigl(\wp(z) - \wp(a_k)\bigr)^{n_k}
	\prod_{i=1}^{3} \bigl(\wp(z) - \wp(\tfrac{\omega_i}{2})\bigr)^{t_i}.
	\]
	Each factor is even and elliptic, and the overall divisor of $h$ coincides with that of $f$.
	
	By the Weierstrass problem on $X$, 
	the meromorphic function with this divisor is unique up to a multiplicative constant.
	Therefore the quotient $f/h$ has no zeros or poles, hence $f/h \equiv c \in \C^{\times}$.
	
	We obtain
	\[
	f(z) 
	= c\,
	\prod_{k=1}^{s} \bigl(\wp(z)-\wp(a_k)\bigr)^{n_k}
	\prod_{i=1}^{3} \bigl(\wp(z)-\wp(\tfrac{\omega_i}{2})\bigr)^{t_i},
	\]
	so $f$ is a rational function of $\wp(z)$.
	
	Conversely, every rational function of $\wp$ is even and elliptic.
	Hence
	\[
	\mathcal{M}^{\mathrm{even}}(X) = \C(\wp).
	\qedhere
	\]
\end{proof}



\subsection{The Projective Embedding of the Complex Torus \(\mathbb{C}/L\)}

The \emph{complex projective space} \(\mathbb{P}^n(\mathbb{C})\) is defined as
\[
\mathbb{P}^n
= (\mathbb{C}^{\,n+1} \setminus \{0\})/\!\sim,
\]
where the equivalence relation is
\[
u \sim v 
\quad \Longleftrightarrow \quad
\exists\, \lambda \in \mathbb{C}^* \text{ such that } u = \lambda v.
\]
Thus, two nonzero vectors in \(\mathbb{C}^{n+1}\) represent the same point in
\(\mathbb{P}^n\) if they differ by a nonzero scalar multiple.  
The equivalence class of \((z_0, z_1, \ldots, z_n)\) is written as
\[
[z_0 : z_1 : \cdots : z_n],
\]
and these are called the \emph{homogeneous coordinates} of the point.

\medskip

The projective space \(\mathbb{P}^n\) can be covered by the open sets
\[
U_k = \{ [z_0 : \cdots : z_n] \mid z_k \neq 0 \},
\qquad k = 0, 1, \ldots, n.
\]
Then
\[
\mathbb{P}^n = \bigcup_{k=0}^{n} U_k.
\]

On \(U_k\), we define the \emph{inhomogeneous coordinates} by dividing by \(z_k\):
\[
[z_0 : \cdots : z_k : \cdots : z_n]
\ \longmapsto\
\Bigl(
\frac{z_0}{z_k},\,
\dots,\,
\frac{z_{k-1}}{z_k},\,
\widehat{\frac{z_k}{z_k}=1},\,
\frac{z_{k+1}}{z_k},\,
\dots,\,
\frac{z_n}{z_k}
\Bigr),
\]
where the \(\,\widehat{\phantom{a}}\,\) symbol indicates that the corresponding component
is \emph{omitted} (since \(z_k/z_k=1\) is fixed). 

Thus, in these coordinates,
\[
U_k \;\cong\; \mathbb{C}^n,
\]
with coordinates
\[
\bigl(z_0/z_k,\, \dots,\, z_{k-1}/z_k,\, z_{k+1}/z_k,\, \dots,\, z_n/z_k\bigr).
\] 

\medskip

When \(n=1\), we obtain the \emph{complex projective line}
\[
\mathbb{P}^1 = \{ [\xi_0 : \xi_1] \mid (\xi_0, \xi_1) \neq (0,0) \},
\]
covered by
\[
U_0 = \{ [\xi_0 : \xi_1] \mid \xi_0 \neq 0 \},
\qquad
U_1 = \{ [\xi_0 : \xi_1] \mid \xi_1 \neq 0 \}.
\]
On \(U_0\) we set \(z = \xi_1 / \xi_0\), and on \(U_1\) we set
\(w = \xi_0 / \xi_1\).  
On the overlap \(U_0 \cap U_1\), these coordinates satisfy \(z w = 1\).  
Thus each \(U_i \cong \mathbb{C}\), their intersection \(U_0 \cap U_1 \cong \mathbb{C}^*\),
and
\[
\mathbb{P}^1 \cong \mathbb{C} \cup \{\infty\},
\]
the Riemann sphere.

\medskip

For \(n=2\), we obtain the \emph{complex projective plane}
\[
\mathbb{P}^2 = \bigcup_{k=0}^2 U_k,
\]
where, for example, on \(U_0\) (with \(\xi_0 \neq 0\)) the inhomogeneous coordinates are
\[
[\xi_0 : \xi_1 : \xi_2]
\ \longmapsto\
\left(
z_1^{(0)} = \frac{\xi_1}{\xi_0},\;
z_2^{(0)} = \frac{\xi_2}{\xi_0}
\right),
\]
so \(U_0 \cong \mathbb{C}^2\).  
Similarly, \(U_1\) and \(U_2\) are also copies of \(\mathbb{C}^2\),
glued together through the appropriate rational transition functions.

\[
\begin{aligned}
	\mathbb{P}^2
	&= U_0
	\,\cup\,
	\Bigl\{
	[\xi_0 : \xi_1 : \xi_2]
	\ \Big|\
	(\xi_0, \xi_1, \xi_2) \in \mathbb{C}^3 \setminus \{0\},
	\ \xi_0 = 0
	\Bigr\} \\[6pt]
	&= U_0
	\,\cup\,
	\Bigl\{
	[0 : \xi_1 : \xi_2]
	\ \Big|\
	(\xi_1, \xi_2) \in \mathbb{C}^2 \setminus \{0\}
	\Bigr\}.
\end{aligned}
\]

The second set is naturally identified with the projective line:
\[
\Bigl\{
[0 : \xi_1 : \xi_2]
\ \Big|\
(\xi_1, \xi_2) \in \mathbb{C}^2 \setminus \{0\}
\Bigr\}
\;\cong\;
\mathbb{P}^1.
\]

Hence,
\[
\mathbb{P}^2
= \mathbb{C}^2
\,\sqcup\, \mathbb{P}^1
= \mathbb{C}^2
\,\sqcup\, \mathbb{C}
\,\sqcup\, \{\text{pt}\}.
\]

\medskip

\noindent
More generally, one obtains the recursive cell decomposition
\[
\boxed{
	\mathbb{P}^n
	= \mathbb{C}^n
	\,\sqcup\, \mathbb{C}^{n-1}
	\,\sqcup\, \cdots
	\,\sqcup\, \mathbb{C}
	\,\sqcup\, \{\text{pt}\}.
}
\]


\begin{theorem}
	Let \( X = \mathbb{C}/L \) be a complex torus, where \(L\) is a lattice in \(\mathbb{C}\).  
	Then \(X\) is biholomorphic to an algebraic curve 
	\(E \subset \mathbb{P}^2\); that is,
	\[
	X = \mathbb{C}/L
	\ \underset{\Phi}{\overset{[1 : \wp(z) : \wp'(z)]}{\longrightarrow}}\
	E \subset \mathbb{P}^2,
	\]
	where \(\wp(z)\) denotes the Weierstrass \(\wp\)-function associated with \(L\), 
	and \(E\) is given by the Weierstrass equation
	\[
	E : y^2 = 4x^3 + g_2 x + g_3.
	\]
	
	For \(z \not\equiv 0 \pmod{L}\),
	\[
	z \ \longmapsto\ (\wp(z), \wp'(z)) \in \mathbb{C}^2 = U_0.
	\]
\end{theorem}


\begin{figure}[h]
	\centering
	\includegraphics[width=0.5\linewidth]{1}

\end{figure}




If we write the projective plane as 
\[
\mathbb{P}^2
= \bigl\{[\omega_0,\omega_1,\omega_2]
: (\omega_0,\omega_1,\omega_2)\in\mathbb{C}^3
\setminus \{0\}\bigr\},
\]
then on the affine chart 
\[
U_0 = \{\omega_0 \neq 0\}, \qquad 
x = \frac{\omega_1}{\omega_0}, \quad 
y = \frac{\omega_2}{\omega_0},
\]
the Weierstrass equation of the elliptic curve takes the form
\[
y^2 = 4x^3 + g_2 x + g_3.
\]

Multiplying by \(\omega_0^4\) yields the homogeneous cubic equation
\[
\omega_0^2\,\omega_2^{\,2}
= 4\,\omega_1^{\,3}
+ g_2\,\omega_1\,\omega_0^{\,2}
+ g_3\,\omega_0^{\,3}.
\]

Hence, the projective curve \(E\) is given by
\[
E
= \Bigl\{[\omega_0,\omega_1,\omega_2]\in\mathbb{P}^2
: P(\omega_0,\omega_1,\omega_2)=0\Bigr\},
\]
where \(P\) is the homogeneous cubic polynomial
\[
P(\omega_0,\omega_1,\omega_2)
= \omega_0^{2}\omega_2^{2}
- \bigl(4\,\omega_1^{3}
+ g_2\,\omega_1\,\omega_0^{2}
+ g_3\,\omega_0^{3}\bigr).
\]

\begin{definition}[Projective embedding of a compact Riemann surface]
	Let \( X \) be a compact Riemann surface, and 
	\[
	\Phi : X \longrightarrow \mathbb{P}^N
	\]
	be a holomorphic mapping.  
	We call \(\Phi\) a \emph{holomorphic embedding} if and only if:
	\begin{enumerate}
		\item[(a)] \(\Phi\) is a holomorphic immersion at every point \(x \in X\).
		\item[(b)] \(\Phi\) separates points, i.e.\ \(\Phi(x) \neq \Phi(y)\) whenever \(x \neq y\), \(x,y \in X\).
	\end{enumerate}
\end{definition}

\begin{remark}
	A \emph{holomorphic mapping} means a continuous map that is holomorphic with respect to local holomorphic coordinate charts.  
	For example, if \(x_0 \in \mathbb{P}^N = \bigcup_{k=0}^N U_k\), where \(U_k \cong \mathbb{C}^N\),
	then near \(x_0\) the map \(\Phi\) can be written in inhomogeneous coordinates as
	\[
	\Phi(z) = \bigl(f_1(z), \dots, f_N(z)\bigr),
	\]
	where each \(f_k\) is holomorphic.
\end{remark}


\begin{definition}[Holomorphic immersion]
	A holomorphic map 
	\[
	\Phi : X \longrightarrow \mathbb{P}^N
	\]
	is said to be a \emph{holomorphic immersion} at \(x_0 \in X\) if the differential
	\[
	d\Phi_{x_0} : T_{x_0}(X) \longrightarrow T_{\Phi(x_0)}(\mathbb{P}^N) \cong \mathbb{C}^N
	\]
	is injective.  
	
	When \(\dim_{\mathbb{C}} X = 1\) and on a local coordinate chart we write 
	\(\Phi = (f_1, \dots, f_N)\),
	this condition simply means that
	\[
	f_k'(x_0) \neq 0 \quad \text{for some } k, \; 1 \leq k \leq N.
	\]
\end{definition}


\begin{theorem}
	Let \( L \subset \mathbb{C} \) be a lattice, and define
	\[
	\widetilde{\Phi} : \mathbb{C} \longrightarrow \mathbb{P}^2, 
	\qquad 
	\widetilde{\Phi}(z) = [\,1 : \wp(z) : \wp'(z)\,] \in U_0.
	\]
	For every point \( z \in \mathbb{C} \setminus L \), the map \(\widetilde{\Phi}\) is holomorphic.  
	It extends to a holomorphic mapping on all of \(\mathbb{C}\), still denoted by the same symbol
	\[
	\widetilde{\Phi} : \mathbb{C} \longrightarrow \mathbb{P}^2.
	\]
	Furthermore, \(\widetilde{\Phi}\) is invariant under translation by any \(\omega \in L\); that is,
	\[
	\widetilde{\Phi}(z+\omega) = \widetilde{\Phi}(z), \qquad \forall\, \omega \in L.
	\]
	Hence, \(\widetilde{\Phi}\) descends to a well-defined holomorphic map
	\[
	\Phi : X = \mathbb{C}/L \longrightarrow \mathbb{P}^2.
	\]
	Moreover, \(\Phi\) is a holomorphic embedding, mapping \(X\) biholomorphically onto a smooth projective curve
	\[
	Z = \Phi(X),
	\]
	which is defined by a homogeneous cubic polynomial.
\end{theorem}

\begin{proof}
	The map 
	\[
	\widetilde{\Phi} : \mathbb{C} \setminus L \longrightarrow U_0 \subset \mathbb{P}^2,
	\qquad 
	\widetilde{\Phi}(z) = [\,1 : \wp(z) : \wp'(z)\,],
	\]
	is holomorphic and invariant under translation by any lattice point 
	\(\omega \in L\), since both \(\wp\) and \(\wp'\) are elliptic with respect to \(L\).
	By definition, for every \(z \in \mathbb{C} \setminus L\),
	\[
	\widetilde{\Phi}(z) = [\,1 : \wp(z) : \wp'(z)\,].
	\]
	
	
	
	\textbf{Extension to a holomorphic map on all of \(\mathbb{C}\).}
	Near \(z = 0\), the Laurent expansions are
	\[
	\wp(z) = \frac{1}{z^2} + h(z), 
	\qquad
	\wp'(z) = -\frac{2}{z^3} + h'(z),
	\]
	where \(h\) is a holomorphic function (given by a Taylor expansion).
	Hence
	\[
	\widetilde{\Phi}(z)
	= [\,1 : \tfrac{1}{z^2} + h(z) : -\tfrac{2}{z^3} + h'(z)\,].
	\]
	Multiplying by \(z^{3}\) for homogenization, we get
	\[
	\widetilde{\Phi}(z)
	= [\,z^{3} : z^{3}h(z) + z : -2 + z^{3}h'(z)\,] 
	= \Biggl[\,\frac{z^{3}}{-2 + z^{3}h'(z)} 
	:\; \frac{z + z^{3}h(z)}{-2 + z^{3}h'(z)}
	:\; 1\,\Biggr]
	\in U_{2}.
	\]
	Define \(\widetilde{\Phi}(0) = [\,0 : 0 : 1\,] \in U_2\).  
	This gives a holomorphic extension near \(z = 0\).
	The same argument works near any \(\omega \in L\), so we obtain a global
	holomorphic map
	\[
	\widetilde{\Phi} : \mathbb{C} \longrightarrow \mathbb{P}^2,
	\]
	with \(\widetilde{\Phi}(\omega) = [\,0 : 0 : 1\,]\) for all \(\omega \in L\).
	Hence \(\widetilde{\Phi}\) descends to a holomorphic map
	\[
	\Phi : X = \mathbb{C}/L \longrightarrow \mathbb{P}^2.
	\]
	

	
	\underline{Claim 1} 
	\(\Phi : X \to \mathbb{P}^2\) is a holomorphic immersion.
	
	(a) Near \(z = 0\), using the previous expansion,
	\[
	\widetilde{\Phi}(z)
	= (-\tfrac{z^3}{2} + \cdots ,\; -\tfrac{z}{2} + \cdots) \in U_2 \cong \mathbb{C}^2.
	\]
	Hence 
	\(\widetilde{\Phi}'(0) = (0, -\tfrac12) \neq (0,0)\),
	so \(\widetilde{\Phi}\) is an immersion at \(0\), and therefore also at every \(\omega \in L\).
	
	(b) If \(z_0 \in \mathbb{C} \setminus L\) and \(\widetilde{\Phi}\) were not immersive at \(z_0\),
	then \(\wp'(z_0) = \wp''(z_0) = 0.\)
	Consider \(f(z) = \wp(z) - \wp(z_0)\).  
	Then \(f(z_0) = f'(z_0) = f''(z_0) = 0\), so \(\operatorname{ord}_{z_0}(f) \ge 3.\)
	However, \(f\) has only double poles at lattice points and no other poles; 
	by the principle that for elliptic functions the number of zeros equals the number of poles (counting multiplicities), this is impossible.  
	Hence \(\Phi\) is immersive everywhere.
	

	\underline{Claim 2} 
	\(\Phi : X \to \mathbb{P}^2\) separates points.
	
	We argue by contradiction.
	Note first that \(\Phi(0) = [\,0 : 0 : 1\,] \notin U_0\),
	so \(\Phi(0) \neq \Phi(x)\) for all \(x \in X \setminus \{0\}\).
	It remains to consider \(x,y \in X \setminus \{0\}\) with \(\Phi(x) = \Phi(y)\).
	
	Choose lifts \(a,b \in \mathbb{C}\) such that 
	\(\pi(a) = x\), \(\pi(b) = y\),
	where \(\pi : \mathbb{C} \to X = \mathbb{C}/L\) is the projection.
	In the affine chart \(U_0 \cong \mathbb{C}^2\),
	\(\Phi(z) = (\wp(z), \wp'(z))\),
	so \(\Phi(a) = \Phi(b)\) implies
	\[
	\wp(a) = \wp(b), \qquad \wp'(a) = \wp'(b).
	\]
	Consider \(f(z) = \wp(z) - \wp(a)\).
	

	
	\emph{Case 1.} \(a \not\equiv -a \ (\mathrm{mod}\; L)\).  
	Since \(\wp\) is even, \(f(z)\) has precisely two simple zeros at \(a\) and \(-a\).
	If \(\wp(a)=\wp(b)=\wp(-a)\) and \(a\not\equiv b\ (\mathrm{mod}\; L)\),
	then \(b\equiv -a\ (\mathrm{mod}\; L)\).
	But then, from \(\wp'(a)=\wp'(b)\) and the oddness of \(\wp'\),
	we get
	\(\wp'(a)=\wp'(-a)=-\wp'(a)\),
	so \(\wp'(a)=0\).
	This gives \(\operatorname{ord}_a(f)\ge2\) and \(\operatorname{ord}_{-a}(f)\ge2\),
	contradicting the zero–pole count for elliptic functions.  
	

	
	\emph{Case 2.} \(a \equiv -a \ (\mathrm{mod}\; L)\); that is, \(2a \equiv 0\), 
	so \(a\) is a half-period:
	\[
	a \equiv \frac{\omega_i}{2} \quad \mathrm{mod}\; L,\qquad i=1,2,3.
	\]
	Suppose \(a,b \in \{\tfrac{\omega_i}{2}+L\}\) with \(a\not\equiv b\ (\mathrm{mod}\; L)\) and
	\(\wp(a)=\wp(b)\).
	But then \(f(z)=\wp(z)-\wp(a)\) has zeros of order \(2\) at both \(a\) and \(b\),
	which again contradicts the zero–pole count.
	Therefore no such distinct \(a,b\) exist.
	
	Thus, \(\Phi\) separates points.
	

	
	Since \(\Phi\) is both a holomorphic immersion and separates points,
	it is a holomorphic embedding.
	Its image is the smooth cubic curve
	\[
	Z = \Phi(X) = \left\{[\,\omega_0:\omega_1:\omega_2\,] \in \mathbb{P}^2 
	\mid 
	P(\omega_0,\omega_1,\omega_2)=0\right\},
	\]
	where \(P\) is the homogeneous cubic polynomial
	\[
	P(\omega_0,\omega_1,\omega_2)
	= 
	\omega_0^2\omega_2^2
	- \bigl(4\omega_1^3 + g_2 \omega_1 \omega_0^2 + g_3 \omega_0^3\bigr).
	\]
\end{proof}
	
	
	
	
	
	
	
	






	
\end{document}