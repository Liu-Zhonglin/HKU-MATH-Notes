\documentclass[12pt]{article}

% Packages for math and formatting
\usepackage[utf8]{inputenc}
\usepackage{amsmath, amssymb, amsthm}
\usepackage{geometry}
\usepackage{enumitem}
\usepackage{hyperref}
\usepackage{mathtools}
\usepackage{epigraph}
\usepackage{tikz}
\usetikzlibrary{arrows.meta, decorations.pathmorphing, positioning}

% Page setup
\geometry{margin=1in}

% Theorem-like environments
\newtheorem{theorem}{Theorem}[section]
\newtheorem{proposition}[theorem]{Proposition}
\newtheorem{lemma}[theorem]{Lemma}
\newtheorem{corollary}[theorem]{Corollary}
\theoremstyle{definition}
\newtheorem{definition}[theorem]{Definition}
\newtheorem{problem}[theorem]{Problem}
\theoremstyle{remark}
\newtheorem{remark}[theorem]{Remark}
\newtheorem{example}[theorem]{Example}

% Custom commands
\newcommand{\R}{\mathbb{R}}
\newcommand{\N}{\mathbb{N}}
\newcommand{\Q}{\mathbb{Q}}
\newcommand{\Z}{\mathbb{Z}}
\newcommand{\C}{\mathbb{C}}
\newcommand{\PP}{\mathbb{P}}
\newcommand{\D}{\mathbb{D}}
\newcommand{\HH}{\mathbb{H}} % upper half-plane

% Operator names
\DeclareMathOperator{\Aut}{Aut}
\DeclareMathOperator{\FractLin}{FractLin}
\DeclareMathOperator{\ord}{ord}
\DeclareMathOperator{\Res}{Res}
\DeclareMathOperator{\SL}{SL}

\title{Compact Riemann surfaces of genus $g\geq 2$,\\
	Part 2}
\author{Joe}
\date{November 19, 2025}

\begin{document}
	
	\maketitle
	

	
%	\tableofcontents
%	\vspace{1em}
	



Let $X$ be a compact Riemann surface of genus $g \ge 2$.

Recall Poincaré Series.
Let $h \in H^\infty(\mathbb{D})$ and let $k \ge 2$ be an integer.
Define the Poincaré series of weight $k$ by
\[
P^k_\Gamma(h)(z)
= \sum_{\gamma \in \Gamma} h(\gamma z) \, (\gamma'(z))^k.
\]
For sufficiently large $k$, the series converges, and we obtain a holomorphic function
$f = P^k_\Gamma(h)$ on $\mathbb{D}$ satisfying
\[
f(\gamma z) \, (\gamma'(z))^k = f(z),
\qquad \forall \gamma \in \Gamma.
\]
Thus $f(z)(dz)^k$ defines a holomorphic $k$‑differential on $X = \mathbb{D} \backslash \Gamma$.

\paragraph{Purpose.}
We wish to prove that there exists an integer $k \gg 0$ such that
the space of holomorphic $k$‑differentials on $X$ provides a projective embedding
\[
\Phi : X \hookrightarrow \mathbb{P}^N,
\qquad
\Phi = [\,s_0, s_1, \ldots, s_N\,].
\]
That is, there exist holomorphic $k$‑differentials
\[
s_j(z) = f_j(z) (dz)^k, \qquad 0 \le j \le N,
\]
such that $\Phi$ is an embedding.

Locally, we may also write
\[
s_j(z) = f_j(z) \, \lambda(z) \, (dz)^k,
\]
where $\lambda(z)$ is a nowhere‑vanishing holomorphic function.
Then
\[
[s_0(z) : s_1(z) : \cdots : s_N(z)]
= [\,\lambda(z)s_0(z) : \lambda(z)s_1(z) : \cdots : \lambda(z)s_N(z)\,],
\]
so the projective point $\Phi(z)$ is independent of the local trivialization.
Hence $\Phi$ is well defined on $X$.

\begin{example}
	It is possible to obtain an embedding of 
	\[
	X = \C / L \longrightarrow \mathbb{P}^2
	\]
	by means of \emph{theta functions}.
	Define
	\[
	\tilde{\Phi}(z)
	= [\,\theta_0(z) : \theta_1(z) : \theta_2(z)\,],
	\]
	where, for $k = 0,1,2$, the functions $\theta_k$ are classical theta functions satisfying
	\[
	\theta_k(z + \omega)
	= \exp(A_\omega z + B_\omega)\,\theta_k(z),
	\qquad \forall\, \omega \in L.
	\]
	Then for all $\omega \in L$ and $z \in \C$,
	\[
	\tilde{\Phi}(z + \omega) = \tilde{\Phi}(z).
	\]
	Hence $\tilde{\Phi}$ descends to a well‑defined holomorphic map
	\[
	\Phi : X \longrightarrow \mathbb{P}^2,
	\]
	which provides an embedding of $X$ into projective space via theta functions.
\end{example}
	
	\medskip
	
We have proved previously that if 
\[
X = \mathbb{D} / \Gamma,
\]
where $\Gamma$ is a discrete torsion‑free subgroup of $\mathrm{Aut}(\mathbb{D})$,
then for any point $z_0 \in \mathbb{D}$, 
there exists a sufficiently large integer $k$ 
such that there exists a holomorphic $k$‑differential
\[
\phi(z) = f(z)(dz)^k,
\]
invariant under~$\Gamma$ and satisfying
\[
\phi(z_0) \neq 0.
\]

\medskip
We begin by decomposing
\[
\Gamma = \Gamma' \sqcup \Gamma'',
\qquad
\Gamma' = \{\,\gamma \in \Gamma : |\gamma'(z)| \ge \tfrac{1}{2}\,\}.
\]

\medskip
\underline{Claim:}
We can choose a function $h$ of the form $h = P|_{\mathbb{D}}$, 
where $P$ is a polynomial $P \in \C[z]$ satisfying
\[
P(z_0) = 1, 
\qquad
P(\gamma z_0) = 0
\quad \text{for all } \gamma \in \Gamma' \setminus \{\mathrm{id}\}.
\]

\medskip
Then we form the Poincaré series of weight~$k$:
\[
f(z)
= P^k_\Gamma(h)
= \sum_{\gamma \in \Gamma}
h(\gamma z)\,(\gamma'(z))^k.
\]
This sum can be split as
\[
f(z)
= 
\underbrace{
	\sum_{\gamma \in \Gamma'} h(\gamma z)\,(\gamma'(z))^k
}_{f(z_0)=1}
\;+\;
\underbrace{
	\sum_{\gamma \in \Gamma''} h(\gamma z)\,(\gamma'(z))^k
}_{\le M}.
\]


\begin{proof}[Proof of the Claim]
	In general, let $z_1, z_2, \dots, z_m \in \C$ be distinct points, 
	and let $a_1, a_2, \dots, a_m \in \C$ be prescribed values.  
	We seek a polynomial $P \in \C[z]$ such that
	\[
	P(z_k) = a_k, \qquad \text{for } k = 1,2,\dots,m.
	\]
	This classical interpolation problem is solved by the 
	\emph{Lagrange interpolation formula}.
	
	\smallskip
	Define
	\[
	Q(z) = \prod_{k=1}^{m}(z - z_k),
	\qquad
	R_k(z) = \frac{Q(z)}{(z - z_k)}.
	\]
	Then
	\[
	R_k(z_k) \neq 0, 
	\qquad 
	R_\ell(z_k) = 0 \quad \text{for all } \ell \ne k.
	\]
	Hence the required polynomial is given by
	\[
	P(z)
	= \sum_{k=1}^{m}
	a_k \frac{R_k(z)}{R_k(z_k)}
	= \sum_{k=1}^{m}
	a_k 
	\frac{\displaystyle\prod_{\ell=1}^{m}(z - z_\ell)}
	{\displaystyle (z - z_k) \prod_{\ell \ne k}(z_k - z_\ell)}.
	\]
	This polynomial satisfies $P(z_k) = a_k$ for all $k=1,\dots,m$.
\end{proof}

\medskip

By the compactness of $X$, we may choose $k \gg 0$ and finitely many 
holomorphic $k$‑differentials on $X$,
\[
\phi_0, \phi_1, \dots, \phi_N,
\]
such that for every $x \in X$, 
there exists some index $i \in \{0,1,\dots,N\}$ with
\[
\phi_i(x) \ne 0.
\]


For instance, suppose there exist points $x_1, x_2 \in X$ and differentials $\phi_1,\phi_2$ satisfying
\[
\begin{cases}
	\phi_1(x_1) \neq 0, \\
	\phi_1(x_2) = 0,
\end{cases}
\qquad
\begin{cases}
	\phi_2(x_1) = 0, \\
	\phi_2(x_2) \neq 0.
\end{cases}
\]
By scaling and small perturbation, we may assume
\[
\begin{cases}
	\phi_1(x_1) = 1, & |\phi_1(x_2)| < \varepsilon,\\[3pt]
	|\phi_2(x_1)| < \varepsilon, & \phi_2(x_2) = 1,
\end{cases}
\]
for some small $\varepsilon > 0$.  
Then the map
\[
\Phi : X \to \mathbb{P}^N, \qquad
\Phi(x) = [\,\phi_0(x) : \phi_1(x) : \cdots : \phi_N(x)\,],
\]
is non‑constant, since the images of $x_1$ and $x_2$ differ:
\[
\Phi(x_1) \ne \Phi(x_2).
\]


\noindent
\textbf{Alternative argument.}
Fix a point $x_0 \in X$ and choose a lift $z_0 \in \mathbb{D}$ 
so that $\pi(z_0) = x_0$ (i.e. $x_0 = z_0 \bmod \Gamma$).  
Let $\phi_0$ be a holomorphic $k$‑differential on $X$ for $k \gg 0$.  
Because $X$ is compact, $\phi_0$ must have a zero somewhere on $X$;
that is, there exists $x_1 \in X$, $x_1 \neq x_0$, such that
\[
\phi_0(x_1) = 0, \phi_0(x_0)\neq 0
\]
(When $g=1$, i.e. $X$ is an elliptic curve, there exists a 
nowhere‑vanishing holomorphic $k$‑differential for some $k$,
and this distinction characterizes $g=1.$ )

Now choose another holomorphic $k$‑differential $\phi_1$ such that 
$\phi_1(x_1) \ne 0$.  
Then the pairs of projective coordinates
\[
(\phi_0(x_0), \phi_1(x_0))
\quad\text{and}\quad
(\phi_0(x_1), \phi_1(x_1)) = (0, \phi_1(x_1))
\]
are not proportional.  
Hence
\[
\Phi(x_0) \ne \Phi(x_1),
\]
showing that $\Phi : X \to \mathbb{P}^N$ is a non‑constant holomorphic map.


Now we aim to conduct the next three steps:

\medskip
\noindent
\textbf{Step 1.} 
\[
\Phi = [\,\phi_0, \dots, \phi_N\,] : X \hookrightarrow \mathbb{P}^N
\quad \text{holomorphic map (actually non‑constant).}
\]
This part is done.

\medskip
\noindent
\textbf{Step 2.} 
\[
\Phi \text{ is a holomorphic immersion for some } k \gg 0.
\]

\medskip
\noindent
\textbf{Step 3.} 
To prove that 
\[
\Phi \text{ separates points (for some } k \gg 0).
\]


\paragraph{Step 2.}  
To prove that $\Phi$ is a holomorphic immersion for $k \gg 0$,  
it suffices to show the following local statement:

Given $z_0 \in \mathbb{D}$, find two holomorphic $k$‑differentials
$f$ and $g$ such that  
\[
\text{(a)}\; f = P^k_\Gamma(t) \text{ satisfies } f(z_0) \ne 0,
\qquad
\text{(b)}\; g = P^k_\Gamma(s) \text{ satisfies }
\left(\frac{g}{f}\right)'(z_0) \ne 0.
\]
Indeed, in the inhomogeneous coordinates
\[
\Phi = [t,s_1,\dots,s_m] 
= \biggl(\frac{s_1}{t}, \dots, \frac{s_m}{t} \biggr),
\]
the immersion condition requires 
$\bigl(\frac{s_j}{t}\bigr)'(z_0) \ne 0$ for some $j$,
i.e. the derivative does not vanish at $z_0$.



\begin{proof}
	We first construct a pair of polynomials realizing prescribed
	value and derivative data under the $\Gamma$–relations.
	
	We wish to prove that for given complex numbers $a,b \in \C$,
	there exists a polynomial $R(z)$ such that
	\[
	R(z_0) = a, \qquad R'(z_0) = b,
	\]
	and moreover
	\[
	R(\gamma z_0) = 0, \qquad R'(\gamma z_0) = 0
	\quad \text{for all } \gamma \in \Gamma \setminus \{\mathrm{id}\}.
	\]
	
	Let
	\[
	P(z) = \prod_{i=2}^{N}(z - \gamma_i z_0)^2,
	\qquad 
	\Gamma_0 = \{\gamma_1 = \mathrm{id}, \gamma_2, \dots, \gamma_N\}.
	\]
	Define
	\[
	Q(z) = (z - z_0) \prod_{i=2}^{N}(z - \gamma_i z_0)^2
	= (z - z_0) P(z).
	\]
	Then
	\[
	P(z_0) =a \ne 0, 
	\qquad 
	P'(z_0) = b,
	\]
	and hence
	\[
	Q(z_0) = 0, 
	\qquad 
	Q'(z_0) = P(z_0) \ne 0.
	\]
	
	Therefore, at $z_0$ we have
	\[
	(P(z_0), P'(z_0)) = (a,b),
	\qquad 
	(Q(z_0), Q'(z_0)) = (0,a), \quad a \ne 0.
	\]
	Since these pairs are linearly independent,
	for any prescribed $(\tilde a, \tilde b) \in \C^2$
	there exist $\alpha, \beta \in \C$ such that
	\[
	R(z) = \alpha P(z) + \beta Q(z)
	\]
	satisfies
	\[
	R(z_0) = \tilde a, \qquad R'(z_0) = \tilde b.
	\]
	We may take in particular
	\[
	(\tilde a, \tilde b) = (1,0)
	\quad \text{(Polynomial A)},
	\qquad
	(\tilde a, \tilde b) = (0,1)
	\quad \text{(Polynomial B)}.
	\]
	
	Observe that
	\[
	P(\gamma_i z_0) = P'(\gamma_i z_0) = 0,
	\qquad 
	Q(\gamma_i z_0) = Q'(\gamma_i z_0) = 0,
	\quad \forall\, 2 \le i \le N.
	\]
	
	Now let
	\[
	t = A|_{\mathbb{D}},
	\qquad
	s = B|_{\mathbb{D}},
	\]
	and for $k \gg 0$ define
	\[
	f = P^k_\Gamma(t), 
	\qquad 
	g = P^k_\Gamma(s),
	\]
	where $P^k_\Gamma$ denotes the Poincaré series operator.
	
	\medskip
	\noindent\textbf{Claim.}
	For sufficiently large $k$, the pair $(f,g)$ satisfies the required
	conditions in the proposition:
	\[
	f(z_0) \ne 0,
	\qquad 
	\bigl(\tfrac{g}{f}\bigr)'(z_0) \ne 0,
	\]
	where
	\[
	f(z) = \sum_{\gamma \in \Gamma_0} t(\gamma z)\,(\gamma'(z))^k,
	\qquad 
	g(z) = \sum_{\gamma \in \Gamma \setminus \Gamma_0}
	s(\gamma z)\,(\gamma'(z))^k.
	\]
	Thus $\Phi$ is a holomorphic immersion when $k \gg 0$.
\end{proof}


\paragraph{Step 3.}
To complete the proof of the theorem that 
\[
X \hookrightarrow \mathbb{P}^N
\]
is an embedding for sufficiently large \(k\), we must show that
for the pluricanonical map
\[
\Phi = [s_0, s_1, \dots, s_N],
\]
we have
\[
\Phi(x_0) \neq \Phi(x_1)
\quad
\text{for all distinct } x_0, x_1 \in X,
\quad \text{when } k \gg 0.
\]




For two distinct points \(x_0, x_1 \in X\), it suffices to find
two holomorphic \(k\)-differentials \(s_0, s_1\) such that
\[
\begin{cases}
	s_0(x_0) \neq 0, & s_1(x_0) = 0,\\[4pt]
	s_0(x_1) = 0,   & s_1(x_1) \neq 0.
\end{cases}
\]
Then
\[
\Phi(x_0) = [1:0], \qquad \Phi(x_1) = [0:1],
\]
so the image points are distinct. Hence, \(\Phi\) separates \(x_0\) and \(x_1\).


Subtlety (Heine–Borel type issue).
At first, one can solve the separation problem for each \emph{individual pair}
\((x_0, x_1)\) of distinct points by constructing corresponding
sections \(s_0, s_1\). However, the degree \(k\) needed may depend
on the chosen pair:
\[
k = k(x_0, x_1).
\]
To obtain a single embedding map \(\Phi_k\) valid for all pairs,
we must ensure that one sufficiently large \(k\) works \emph{uniformly} for
every pair \((x_0, x_1)\).

Let
\[
S = \{(x_0, x_1) \in X \times X : x_0 \neq x_1\}
= (X \times X) \setminus \mathrm{Diag}(X).
\]
Although \(X \times X\) is compact,
\(S\) is open (the complement of the diagonal), hence \emph{not compact}.
Therefore, we cannot directly argue by compactness to obtain a uniform \(k\).


Suppose now that for some large \(k\), the map
\[
\Phi = [\phi_0, \dots, \phi_N] : X \hookrightarrow \mathbb{P}^N
\]
is already a holomorphic immersion (by Step 2). Then for each
\(x \in X\), there exists a neighborhood \(U(x)\) such that
for all \(x_1, x_2 \in U(x)\), \(x_1 \neq x_2\), we have
\[
\Phi(x_1) \neq \Phi(x_2).
\]
Thus, the image points are separated locally.

Consequently, there exists a symmetric open set
\(T \subset X \times X\) (a \emph{tubular neighborhood} of the diagonal)
such that
\[
(x_1, x_2) \in T \quad \Rightarrow \quad \Phi(x_1) \neq \Phi(x_2),
\qquad \text{and} \qquad (x_1, x_2) \in T \iff (x_2, x_1) \in T.
\]

The complement
\[
(X \times X) \setminus T
\]
is closed in \(X \times X\), hence compact (since \(X \times X\) is compact).

\paragraph{Further subtlety}
Start with an integer \(k_0 \gg 0\) chosen so that the map
\(\Phi_{k_0}\) satisfies the immersion property (Step 1 and Step 2).
To separate points, we work with pairs \((x_1,x_2)\in U=X\times X\setminus T\),
where \(T\subset X\times X\) is the symmetric tubular neighborhood of the diagonal,
introduced previously.

For such pairs, we may need a larger degree
\(k=k(x_1,x_2) > k_0\)
to achieve point separation.
This suggests a potential further subtlety: the degree required might depend on the pair of points,
preventing a single, uniform choice of \(k\).

\textbf{Conceptual explanation: why we can avoid this subtlety.}
Suppose we already have a map
\[
\Phi = [s_0, s_1, s_2] : X \longrightarrow \mathbb{P}^2
\]
that is a holomorphic immersion and separates points
\((x_1, x_2)\in T \subset X\times X\).
To increase its separating power globally, we can apply the
\emph{Veronese embedding} of projective space.

Define a new map
\[
\nu_2 \circ \Phi : X \longrightarrow \mathbb{P}^{N'}
\]
given by all quadratic monomials in the coordinates of \(\Phi\):
\[
[s_0, s_1, s_2]
\longmapsto
[s_0^2,\, s_1^2,\, s_2^2,\, s_0s_1,\, s_0s_2,\, s_1s_2].
\]
This realizes the composition
\[
X \hookrightarrow \mathbb{P}^N \hookrightarrow \mathbb{P}^{N'},
\]
where
\[
N' = \frac{(N+1)(N+2)}{2} - 1.
\]
For instance, when \(N=2\), we have the classical Veronese embedding
\(\mathbb{P}^2 \hookrightarrow \mathbb{P}^5\).

Explicitly, if \((x,y,z)\) are homogeneous coordinates on \(\mathbb{P}^2\),
then
\[
(x,y,z) \longmapsto (x,y,z)(x,y,z)^T
=
\begin{bmatrix}
	x^2 & xy & xz \\
	xy & y^2 & yz \\
	xz & yz & z^2
\end{bmatrix},
\]
whose entries (up to order) yield the six homogeneous coordinates of \(\mathbb{P}^5\).
The target space here,
\(\mathbb{P}(S^2 \mathbb{C}^3)\),
is the projectivization of the space of symmetric \(3\times3\) matrices,
which has dimension \(6\).


In general, for any \(N\), the \emph{Veronese embedding}
\[
\nu_m : \mathbb{P}^N \hookrightarrow \mathbb{P}^{N_m}, 
\qquad N_m = \binom{N+m}{m} - 1,
\]
is holomorphic and injective.  
Composing \(\Phi_k\) with \(\nu_m\) corresponds to replacing the sections \(s_i\)
by all degree‑\(m\) monomials in them, i.e. by sections of \((K_X^{\otimes k})^{\otimes m} = K_X^{\otimes mk}\).

Thus, after fixing a symmetric tubular neighborhood \(T=T_k\subset X\times X\),
the higher pluricanonical map
\[
\Phi_{mk}: X \longrightarrow \mathbb{P}^{N_{mk}}
\]
is still a holomorphic immersion and now separates all points
\((x_1, x_2)\in X\times X \setminus T\).


\paragraph{Final argument.}
Given distinct points \((x_1, x_2) \in X \times X\) with \(x_1 \neq x_2\),
we want to find \(k'_0\) large enough such that
for every \(k \ge k'_0\), there exist holomorphic \(k\)-differentials
\(s_1, s_2 \in H^0(X, K_X^{\otimes k})\) satisfying
\[
\begin{cases}
	s_1(x_1) \neq 0, & s_1(x_2) = 0, \\[4pt]
	s_2(x_1) = 0,   & s_2(x_2) \neq 0.
\end{cases}
\]
Then their corresponding evaluation points in projective space satisfy
\[
[s_1(x_1):s_2(x_1)] \ne [s_1(x_2):s_2(x_2)].
\]
Hence the pluricanonical map
\[
\Phi_k = [s_0, s_1, \dots, s_N]: X \longrightarrow \mathbb{P}^N
\]
separates the distinct points \(x_1\) and \(x_2\).







\begin{proof}
	Let $\Gamma$ be the covering (Fuchsian) group acting on the unit disk $D$, 
	with quotient $X = D/\Gamma$.
	Fix two lifts $z_0, z_1 \in D$ corresponding to distinct points 
	$x_0, x_1 \in X$.
	
	Define
	\[
	\Gamma' = 
	\bigl\{\,
	\gamma \in \Gamma :
	|\gamma'(z_0)| \ge \tfrac{1}{2}
	\ \text{or}\
	|\gamma'(z_1)| \le \tfrac{1}{2}
	\,\bigr\}.
	\]
	The set $\Gamma'$ is finite, because
	\(\sum_{\gamma \in \Gamma} |\gamma'(z_0)|^2 < \infty\)
	implies that only finitely many $\gamma$ have derivatives
	not exponentially small.
	Denote its complement by
	\[
	\Gamma'' = \Gamma \setminus \Gamma'
	= \bigl\{\,
	\gamma \in \Gamma : 
	|\gamma'(z_0)| < \tfrac{1}{2}
	\ \text{and}\
	|\gamma'(z_1)| > \tfrac{1}{2}
	\,\bigr\}.
	\]
	

	We construct holomorphic polynomials \(P_1, P_2\) (by Lagrange interpolation) 
	satisfying for all $\gamma \in \Gamma'$:
	\[
	\begin{cases}
		P_1(z_0) = 1,\quad 
		P_1(\gamma z_0) = 0 \ \text{for } \gamma \ne \mathrm{id},\\[2pt]
		P_1(\gamma z_1) = 0,\\[4pt]
		P_2(z_1) = 1,\quad 
		P_2(\gamma z_1) = 0 \ \text{for } \gamma \ne \mathrm{id},\\[2pt]
		P_2(\gamma z_0) = 0.
	\end{cases}
	\]
	The finite nature of $\Gamma'$ guarantees that these interpolation conditions
	determine $P_1$ and $P_2$.
	
	Let \(f_j = P_j|_D\) for \(j = 1,2\).
	

	For each integer \(k > 0\),
	define
	\[
	s_j(z) = \sum_{\gamma \in \Gamma}
	f_j(\gamma z)\, (\gamma'(z))^k, \qquad j=1,2.
	\]
	These functions are $\Gamma$‑automorphic $k$‑differentials since
	for all $\delta \in \Gamma$,
	\[
	s_j(\delta z)\,(\delta'(z))^k = s_j(z).
	\]
	
	We split the sum as
	\[
	s_j = 
	\underbrace{\sum_{\gamma \in \Gamma'} 
		f_j(\gamma z)(\gamma'(z))^k}_{\text{main part}}
	\;+\;
	\underbrace{\sum_{\gamma \in \Gamma''}
		f_j(\gamma z)(\gamma'(z))^k}_{\text{tail}}.
	\]
	For $\gamma \in \Gamma''$,
	we have $|\gamma'(z_0)| < \tfrac{1}{2}$,
	thus
	\(|(\gamma'(z))^k| \le 2^{-k}\),
	so the tail decays exponentially with \(k\)
	and vanishes uniformly as \(k\to\infty\).

	By interpolation properties:
	\[
	f_1(z_0)=1, \quad
	f_1(\gamma z_0)=0\ (\gamma\ne\mathrm{id}),\quad
	f_1(\gamma z_1)=0,
	\]
	and analogously
	\[
	f_2(z_1)=1, \quad
	f_2(\gamma z_1)=0\ (\gamma\ne\mathrm{id}),\quad
	f_2(\gamma z_0)=0.
	\]
	Hence for the constructed series,
	\[
	\begin{aligned}
		s_1(z_0)&= f_1(z_0) + O(2^{-k}) = 1 + O(2^{-k}), & 
		s_1(z_1)&=0,\\
		s_2(z_0)&=0, &
		s_2(z_1)&= f_2(z_1) + O(2^{-k}) = 1 + O(2^{-k}).
	\end{aligned}
	\]
	Therefore, for sufficiently large \(k\),
	\[
	[s_1(z_0):s_2(z_0)] = [1:0],
	\qquad
	[s_1(z_1):s_2(z_1)] = [0:1].
	\]
	

	So as $k\to\infty$,
	\(
	[s_1(z_0):s_2(z_0)] \ne [s_1(z_1):s_2(z_1)].
	\)
	Thus the differentials \(s_1, s_2\) separate
	the pair of points $x_0, x_1$ on $X$.
	

	Therefore, for every distinct pair \((x_0,x_1) \in X\times X\),
	there exists \(k'_0\) such that for all \(k \ge k'_0\)
	we can construct holomorphic $k$‑differentials \(s_1,s_2\) with
	\[
	[s_1(x_0):s_2(x_0)] \neq [s_1(x_1):s_2(x_1)].
	\]
	Hence the map $\Phi_k$ separates all points for $k \gg 0$,
	and therefore defines a holomorphic embedding.
\end{proof}




	
	
	
\end{document}tex