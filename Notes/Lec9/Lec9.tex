\documentclass[12pt]{article}

% Packages for math and formatting
\usepackage[utf8]{inputenc}
\usepackage{amsmath, amssymb, amsthm}
\usepackage{geometry}
\usepackage{enumitem}
\usepackage{hyperref}
\usepackage{mathtools}
\usepackage{epigraph}
\usepackage{tikz}
\usetikzlibrary{arrows.meta, decorations.pathmorphing, positioning}

% Page setup
\geometry{margin=1in}

% Theorem-like environments
\newtheorem{theorem}{Theorem}[section]
\newtheorem{proposition}[theorem]{Proposition}
\newtheorem{lemma}[theorem]{Lemma}
\newtheorem{corollary}[theorem]{Corollary}
\theoremstyle{definition}
\newtheorem{definition}[theorem]{Definition}
\newtheorem{problem}[theorem]{Problem}
\theoremstyle{remark}
\newtheorem{remark}[theorem]{Remark}
\newtheorem{example}[theorem]{Example}

% Custom commands
\newcommand{\R}{\mathbb{R}}
\newcommand{\N}{\mathbb{N}}
\newcommand{\Q}{\mathbb{Q}}
\newcommand{\Z}{\mathbb{Z}}
\newcommand{\C}{\mathbb{C}}
\newcommand{\PP}{\mathbb{P}}
\newcommand{\D}{\mathbb{D}}
\newcommand{\HH}{\mathbb{H}} % upper half-plane

% Operator names
\DeclareMathOperator{\Aut}{Aut}
\DeclareMathOperator{\FractLin}{FractLin}
\DeclareMathOperator{\ord}{ord}
\DeclareMathOperator{\Res}{Res}
\DeclareMathOperator{\SL}{SL}

\title{Sheaf Cohomology}
\author{Joe}
\date{December 3, 2025}

\begin{document}
	
	\maketitle
	
	
		\epigraph{You have to be able to pose questions otherwise there will be no mathematics.}{\textit{Prof. MOK Ngaiming}}
		\tableofcontents
		\vspace{1em}
	


In topology, let $X$ be a topological space. We have the singular (co)homology groups
\[
H_p(X, \mathbb{Z}), \qquad H^p(X, \mathbb{Z}).
\]
For example, for a compact Riemann surface $X$ of genus $g$,
\[
H_1(X, \mathbb{Z}) \cong \mathbb{Z}^{2g}, \qquad
H^1(X, \mathbb{Z}) \cong \mathbb{Z}^{2g}.
\]
In this case, $H_p(X, \mathbb{Z}) = H_p^{\mathrm{sing}}(X, \mathbb{Z})$.

\medskip

\noindent
\textbf{Cellular (co)homology:}
Cells of dimension $n$ are homeomorphic to $\mathbb{R}^n$.

\medskip

\noindent
\textbf{Example:}
\[
\mathbb{P}^n = \mathbb{C}^n \sqcup \mathbb{P}^{n-1}
= \mathbb{C}^n \sqcup \mathbb{C}^{n-1} \sqcup \cdots \sqcup \{0\}.
\]
Then
\[
H^{k}(\mathbb{P}^m, \mathbb{Z}) \ \text{is the same, and} \
H_{2k}(\mathbb{P}^m, \mathbb{Z}) \cong \mathbb{Z}
\quad \text{for } k = 0, 1, 2, \dots, m.
\]

\section{Čech Cohomology for $\mathbb{Z}$}

Let $X$ be a topological manifold, and let 
$\mathcal{U} = \{ U_\alpha \}$ 
be an open cover of $X$ such that every finite intersection
\[
U_{\alpha_0\alpha_1\cdots \alpha_m}
:= U_{\alpha_0} \cap U_{\alpha_1} \cap \cdots \cap U_{\alpha_m}
\]
is either empty or connected.



Let $p > 0$ be an integer.  
Define the abelian group of $p$--cochains
\[
C^p(\mathcal{U}, \Z)
\]
as follows:

A $p$--cochain is a collection of integers
\[
\{\, c_{\alpha_0 \alpha_1 \cdots \alpha_p} \,\}
\]
assigned to each nonempty $(p+1)$--fold intersection
$U_{\alpha_0 \alpha_1 \cdots \alpha_p}$.

The cochain is \emph{alternating}, meaning that for every permutation 
$\sigma \in \mathrm{Perm}\{0,1,\ldots,p\}$,
\[
c_{\alpha_{\sigma(0)} \alpha_{\sigma(1)} \cdots \alpha_{\sigma(p)}}
= \mathrm{sign}(\sigma)\, c_{\alpha_0 \alpha_1 \cdots \alpha_p},
\]
where if $\sigma$ is a composition of $q$ transpositions, then
\[
\mathrm{sign}(\sigma) = (-1)^q.
\]




Define the coboundary operator
\[
\delta_p : C^p(\mathcal{U}, \Z) \longrightarrow C^{p+1}(\mathcal{U}, \Z)
\]
as follows.

For a cochain $c \in C^p(\mathcal{U}, \Z)$, the cochain 
$e = \delta_p c \in C^{p+1}(\mathcal{U}, \Z)$ is defined by
\[
(\delta_p c)_{\alpha_0 \alpha_1 \cdots \alpha_{p+1}}
=
\sum_{i=0}^{p+1} 
(-1)^i \, c_{\alpha_0 \cdots \widehat{\alpha_i} \cdots \alpha_{p+1}},
\]
where the hat $\widehat{\alpha_i}$ means that the index $\alpha_i$ is omitted.

\begin{lemma}
	For all $p \ge 0$, one has
	\[
	\delta_{p+1} \circ \delta_p = 0.
	\]
\end{lemma}

\begin{example}
	We check the case $p = 1$, that is, $\delta_2 \circ \delta_1 = 0$.
	
	Let $c = \{ c_{\alpha_0 \alpha_1} \}$ be a $1$--cochain.  
	Then $\delta_1 c = e = \{ e_{\alpha_0 \alpha_1 \alpha_2} \}$ is given by
	\[
	e_{\alpha_0 \alpha_1 \alpha_2}
	= c_{\alpha_1 \alpha_2}
	- c_{\alpha_0 \alpha_2}
	+ c_{\alpha_0 \alpha_1}.
	\]
	Now apply $\delta_2$ to $e$:
	\[
	(\delta_2 e)_{\alpha_0 \alpha_1 \alpha_2 \alpha_3}
	= e_{\alpha_1 \alpha_2 \alpha_3}
	- e_{\alpha_0 \alpha_2 \alpha_3}
	+ e_{\alpha_0 \alpha_1 \alpha_3}
	- e_{\alpha_0 \alpha_1 \alpha_2}.
	\]
	Expanding each $e_{\alpha_i \alpha_j \alpha_k}$ in terms of the $c_{\bullet\bullet}$ shows that all
	terms cancel, hence
	\[
	(\delta_2 e)_{\alpha_0 \alpha_1 \alpha_2 \alpha_3} = 0.
	\]

\end{example}
	
\begin{definition}
	\leavevmode
	\begin{enumerate}
		\item[(a)] A cochain $c \in C^p(\mathcal{U}, \Z)$ is called a 
		\emph{$p$--cocycle} if and only if 
		$\delta_p c = 0$.  
		The set of all $p$--cocycles is denoted
		\[
		Z^p(\mathcal{U}, \Z) := \ker(\delta_p).
		\]
		
		\item[(b)] A cochain $c \in C^p(\mathcal{U}, \Z)$ is called a 
		\emph{$p$--coboundary} if there exists 
		$b \in C^{p-1}(\mathcal{U}, \Z)$ such that 
		$c = \delta_{p-1} b$.  
		The set of all $p$--coboundaries is denoted
		\[
		B^p(\mathcal{U}, \Z) := \operatorname{im}(\delta_{p-1}).
		\]
	\end{enumerate}
\end{definition}

\begin{lemma}
	\[
	B^p(\mathcal{U}, \Z) \subset Z^p(\mathcal{U}, \Z)
	\quad \text{as a subgroup of } C^p(\mathcal{U}, \Z).
	\]
\end{lemma}

\begin{proof}
	Let $c = \delta_{p-1} b \in B^p(\mathcal{U}, \Z)$.
	Then
	\[
	\delta_p c
	= \delta_p (\delta_{p-1} b)
	= (\delta_p \circ \delta_{p-1})(b)
	= 0,
	\]
	since $\delta_p \circ \delta_{p-1} = 0$.

\end{proof}

\begin{definition}\label{def:cech-cohomology}
	\leavevmode
	\begin{enumerate}
		\item[(a)] For a given open cover $\mathcal{U}$ of $X$, the 
		$p$--th Čech cohomology group of the cover is defined as
		\[
		H^p(\mathcal{U}, \Z)
		:= Z^p(\mathcal{U}, \Z) \big/ B^p(\mathcal{U}, \Z).
		\]
		
		\item[(b)] The Čech cohomology of $X$ with coefficients in $\Z$ is defined as the direct limit over all open covers:
		\[
		\check{H}^p(X, \Z)
		:= \varinjlim_{\mathcal{U}} \, H^p(\mathcal{U}, \Z).
		\]
	\end{enumerate}
\end{definition}



If $\mathcal{U} = \{ U_\alpha \}$ and 
$\mathcal{V} = \{ V_i \}$ are two open covers such that, for every $i$, there exists an $\alpha(i)$ with
\[
V_i \subset U_{\alpha(i)},
\]
then $\mathcal{V}$ is called a \emph{refinement} of $\mathcal{U}$.
In this case there is a natural \emph{restriction homomorphism}
\[
H^p(\mathcal{U}, \Z) \xrightarrow{\ \mathrm{Res}\ } H^p(\mathcal{V}, \Z),
\]
which is compatible with the direct limit system that defines 
$\check{H}^p(X, \Z)$.

\begin{theorem}
	Let $\mathcal{U}$ be an \emph{acyclic cover} for the constant sheaf $\Z$ over $X$
	—for example, if every finite intersection 
	$U_{\alpha_0 \ldots \alpha_p} 
	= U_{\alpha_0} \cap \cdots \cap U_{\alpha_p}$ 
	is contractible (hence acyclic) for all $p, \alpha_0, \ldots, \alpha_p$.
	Then we have a natural isomorphism
	\[
	\check{H}^p(X, \Z) \cong H^p(\mathcal{U}, \Z).
	\]
\end{theorem}

\begin{theorem}
	For a reasonable topological space (e.g., paracompact, such as a manifold),
	the Čech cohomology with constant coefficients agrees with singular cohomology:
	\[
	\check{H}^p(X, \Z) \cong H^p_{\mathrm{sing}}(X, \Z).
	\]
\end{theorem}

\medskip

\underline{\text{Observation:}}  

Let $c \in C^0(\mathcal{U}, \Z)$, given by local sections
$c = (c_\alpha)$.
Then $\delta c \in C^1(\mathcal{U}, \Z)$ is defined by
\[
(\delta c)_{\alpha_0 \alpha_1} 
= c_{\alpha_1} - c_{\alpha_0} 
\quad \text{on } U_{\alpha_0} \cap U_{\alpha_1}.
\]
Hence, $\delta c = 0$ if and only if 
\[
c_{\alpha_1} = c_{\alpha_0}
\quad \text{on } U_{\alpha_0} \cap U_{\alpha_1},
\]
i.e. the local data $\{ c_\alpha \}$ glue together to define a global element
of $C^0(X, \Z)$.
	
	
	
\section{Čech Cohomology of Sheaves}

\begin{definition}[Sheaf of Abelian groups over topological manifolds $X$]
	A continuous map 
	\[
	\pi : \mathcal{F} \to X
	\]
	is called a \emph{sheaf of abelian groups over $X$} if the following conditions hold:
	
	\begin{enumerate}
		\item $\mathcal{F}$ is a topological space and 
		$\pi$ is a continuous map which is a \emph{local homeomorphism}.
		
		\item For each $x \in X$, the fibre 
		\[
		\mathcal{F}_x := \pi^{-1}(x)
		\]
		is an abelian group.
		
		\item Let $e_x \in \mathcal{F}_x$ denote the identity element of the group $\mathcal{F}_x$. 
		Then the map 
		\[
		e : X \longrightarrow \mathcal{F}, 
		\qquad x \longmapsto e_x,
		\]
		is continuous.
		
		\item Denote by 
		\[
		\mathcal{F} \times_\pi \mathcal{F}
		= \{ (a,b) \in \mathcal{F} \times \mathcal{F} \mid \pi(a) = \pi(b) \}
		\]
		the fibre product of $\mathcal{F}$ with itself over $X$.
		Then the group operations (pointwise addition and inverse)
		\[
		(a,b) \longmapsto a \cdot b,
		\qquad
		a \longmapsto a^{-1},
		\]
		defined fibrewise on $\mathcal{F} \times_\pi \mathcal{F}$,
		are continuous maps.
	\end{enumerate}
\end{definition}

\begin{remark}
	The condition that “$\pi$ is a local homeomorphism” means that for every 
	$a \in \mathcal{F}$ with $\pi(a) = x$, there exist open neighbourhoods 
	$\mathcal{W}$ of $a$ in $\mathcal{F}$ and $U$ of $x$ in $X$ 
	such that the restriction 
	\[
	\pi|_{\mathcal{W}} : \mathcal{W} \to U
	\]
	is a homeomorphism. 
	Intuitively, each small open set $U \subset X$ is locally identified with 
	its “slice” $\mathcal{W} \subset \mathcal{F}$ lying above it.
\end{remark}
	
\begin{example}[Constant Sheaf]
	Let 
	\[
	\mathcal{F} = X \times \mathbb{Z},
	\]
	with the product topology, where $\mathbb{Z}$ carries the discrete topology.  
	
	Then the projection
	\[
	\pi : \mathcal{F} \longrightarrow X, \qquad (x,n) \longmapsto x,
	\]
	is continuous and a local homeomorphism.  
	For each $x \in X$, the fibre
	\[
	\mathcal{F}_x = \pi^{-1}(x) = \{(x,n) \mid n \in \mathbb{Z}\}
	\]
	is naturally identified with the group $\mathbb{Z}$ by $(x,n) \mapsto n$.
	Each fibre $\mathcal{F}_x$ therefore carries the abelian group structure of $\mathbb{Z}$,
	and the operations of addition and inverse
	\[
	(x,n_1) + (x,n_2) = (x, n_1 + n_2), 
	\qquad 
	-(x,n) = (x, -n),
	\]
	are continuous because $\mathbb{Z}$ is discrete.
	The identity section $e(x) = (x,0)$ is also continuous.
	

	
	\medskip
	\noindent
	\emph{Geometric interpretation.}
	Since $\mathbb{Z}$ is discrete, the total space decomposes as a 
	disjoint union
	\[
	\mathcal{F} 
	= \bigsqcup_{n \in \mathbb{Z}} (X \times \{n\}),
	\]
	where each piece $X \times \{n\}$ is homeomorphic to $X$.
	Hence $\mathcal{F}$ can be viewed as 
	a countable family of ``layers,'' each one a copy of $X$,
	indexed by the integers $n \in \mathbb{Z}$.
	Over each point $x \in X$, the fibre $\mathcal{F}_x$ consists of
	countably many points $(x,n)$ stacked vertically, one for each $n$,
	forming a discrete abelian group isomorphic to $\mathbb{Z}$.
	
\end{example}

\medskip

Now let $X$ be a topological manifold.  
We construct another fundamental example: the sheaf of germs of continuous functions on $X$.

Consider all triples $(U, f, x)$ where
$U$ is a connected open neighbourhood of $x \in X$, and  
$f : U \to \C$ is a continuous function.

Define an equivalence relation:
\[
(U, f, x) \sim (V, h, y)
\quad \Longleftrightarrow \quad
\begin{cases}
	x = y, \\[4pt]
	\text{and there exists an open neighbourhood } 
	W \subset U \cap V \text{ of } x \\[4pt]
	\text{such that } f|_W = h|_W.
\end{cases}
\]

\begin{lemma}
	Let
	\[
	\mathcal{F} := \{ (U, f, x) \} \big/ \sim
	\]
	be the set of equivalence classes of such triples, and define
	\[
	\pi : \mathcal{F} \to X, 
	\qquad 
	\pi([U,f,x]) = x.
	\]
	Then $\mathcal{F}$, with the natural topology induced by neighbourhoods of the form
	\[
	\{ [V, f|_V, x] \mid x \in V \subset U \},
	\]
	is a sheaf of abelian groups over $X$.
	Each fibre $\mathcal{F}_x$ consists of germs of continuous functions at $x$.
	This is called the \emph{sheaf of germs of continuous functions} on $X$.
\end{lemma}

	
\begin{example}[Sheaf of Holomorphic Functions]
	In this example we denote by 
	\[
	\mathcal{F} = \mathcal{O}_X
	\]
	the \emph{sheaf of germs of holomorphic functions} on a complex manifold \(X\).
	
	Given a germ \(f_x \in \mathcal{O}_x\) (the equivalence class of a holomorphic function \(f\) defined near \(x\)),
	the connected component of the domain of definition containing \(x\)
	determines the \emph{maximal domain of existence} of that germ.
	Geometrically, one can think of this as the \emph{Riemann surface}
	spread over \(\C\) that describes a locally defined holomorphic function.
\end{example}

\begin{example}[Sheaf of Holomorphic Sections of a Vector Bundle]
	Let \(E \to X\) be a holomorphic vector bundle over a complex manifold \(X\).
	Then the sheaf
	\[
	\mathcal{O}(E)
	\quad (\text{often written simply as } \mathcal{E})
	\]
	is the \emph{sheaf of germs of holomorphic sections of \(E\)}.
	That is, the fibre of the projection
	\[
	\pi : \mathcal{O}(E) \to X
	\]
	at a point \(x \in X\) consists of all germs of local holomorphic
	sections \(s : U \to E\) defined in a neighbourhood \(U\) of \(x\).
\end{example}

\begin{example}[Sheaf of Nowhere–Vanishing Holomorphic Functions]
	Let
	\[
	\mathcal{O}_X^* \subset \mathcal{O}_X
	\]
	denote the subset consisting of germs of \emph{nowhere-vanishing}
	holomorphic functions.
	The projection
	\[
	\pi : \mathcal{O}_X^* \to X
	\]
	is again a local homeomorphism.
	Each fibre \(\mathcal{O}_x^*\) is the multiplicative group of nonzero complex numbers
	under the operation
	\[
	(f_x, h_x) \longmapsto (f h)_x,
	\qquad \text{with identity } e_x = 1.
	\]
	Thus \(\mathcal{O}_X^*\) is a sheaf of \emph{abelian groups under multiplication}.
\end{example}


\section{Mittag--Leffler Problem}
	
\begin{definition}\label{def:acyclic-cover}
	With the terminology introduced in Definition~\ref{def:cech-cohomology}, 
	an open cover $\mathcal{U} = \{ U_\alpha \}$ of a topological space $X$ is called 
	an \emph{acyclic cover for a sheaf $\mathcal{F}$} if
	for all finite intersections 
	$U_{\alpha_0 \dots \alpha_p} := U_{\alpha_0} \cap \cdots \cap U_{\alpha_p}$ we have
	\[
	H^k(U_{\alpha_0 \dots \alpha_p}, \mathcal{F}) = 0
	\quad \text{for all } k > 0.
	\]
\end{definition}

\begin{example}[Acyclic Cover on a Riemann Surface]
	Let $X$ be a Riemann surface and let $\mathcal{O}_X$ denote the sheaf of holomorphic functions on $X$.
	If $\mathcal{U} = \{ U_\alpha \}$ is a covering of $X$ by open sets each biholomorphic to a domain in $\mathbb{C}$,
	then $\mathcal{U}$ is an \emph{acyclic cover} for $\mathcal{O}_X$.
	
	This follows from the fact that each $U_\alpha$ is a Stein domain, and by the Cauchy–Riemann equations
	we have
	\[
	H^k(U_\alpha, \mathcal{O}_X) = 0 \quad \text{for } k > 0.
	\]
\end{example}

\underline{Fact:} For any Riemann surface $X$, we have
\[
H^2(X, \mathcal{O}_X) = 0.
\]

\begin{example}[Computation for the Projective Line]
	Let $X = \mathbb{P}^1$ and consider the standard covering
	\[
	\mathcal{U} = \{ U_0, U_1 \}, \qquad 
	U_0 = \{ [z_0 : z_1] \mid z_0 \neq 0 \} \cong \mathbb{C}, \quad
	U_1 = \{ [z_0 : z_1] \mid z_1 \neq 0 \} \cong \mathbb{C}.
	\]
	Since $\mathcal{U}$ consists of Stein open sets, it is acyclic for $\mathcal{O}_X$, and hence
	\[
	C^2(\mathcal{U}, \mathcal{O}) = 0 \;\Rightarrow\;
	H^2(\mathbb{P}^1, \mathcal{O}) = 0.
	\]
\end{example}

\begin{remark}[Dolbeault Cohomology Perspective]
	To compute cohomology more analytically,
	one introduces the $\bar{\partial}$--complex (Dolbeault complex)
	\[
	0 \;\longrightarrow\; \mathcal{A}^{0,0}
	\xrightarrow{\;\bar{\partial}\;}
	\mathcal{A}^{0,1}
	\xrightarrow{\;\bar{\partial}\;}
	\mathcal{A}^{0,2} \;\longrightarrow\; \cdots
	\]
	where $\mathcal{A}^{0,q}$ denotes the sheaf of smooth $(0,q)$-forms.
	The Dolbeault theorem identifies
	\[
	H^q(X, \mathcal{O}_X) \;\cong\; H_{\bar{\partial}}^{0,q}(X).
	\]
	On a Riemann surface ($\dim_{\mathbb{C}}X = 1$), 
	there are no nonzero $(0,2)$-forms, so $H^2(X, \mathcal{O}_X)=0$.
\end{remark}

\begin{example}[Čech–Dolbeault Description of Cocycles]
	Let $\mathcal{U} = \{ U_0, U_1 \}$ be the standard covering of $\mathbb{P}^1$,
	and let $\mathcal{O} = \mathcal{O}_X$ denote the sheaf of holomorphic functions.
	Suppose $s_0 \in \mathcal{O}(U_0)$ and $s_1 \in \mathcal{O}(U_1)$ are local
	holomorphic sections.  
	
	The \emph{Čech coboundary operator} acts on a $0$--cochain
	$s = (s_0, s_1) \in C^0(\mathcal{U}, \mathcal{O})$
	according to the general formula
	\[
	(\delta^0 s)_{\alpha\beta}
	= s_\beta|_{U_\alpha \cap U_\beta} - s_\alpha|_{U_\alpha \cap U_\beta}.
	\]
	For our two--set cover, this gives
	\[
	(\delta^0 s)_{01}
	= s_1|_{U_0 \cap U_1} - s_0|_{U_0 \cap U_1}.
	\]
	
	Up to a sign convention, we may write the same difference as
	\[
	\phi_{01} = s_0 - s_1,
	\]
	which is a holomorphic function on the overlap $U_0 \cap U_1$.  
	Hence
	\[
	\phi = \{ \phi_{01} \} \in C^1(\mathcal{U}, \mathcal{O}),
	\]
	and since there is no triple intersection,
	this automatically satisfies $\delta^1 \phi = 0$.
	Thus $\phi$ defines a \emph{$1$--cocycle} in $C^1(\mathcal{U}, \mathcal{O})$.
	
	If one can find local sections $s_0, s_1$ such that
	$\phi_{01} = s_0 - s_1 = (\delta^0 s)_{01}$,
	then the cocycle $\phi$ is a coboundary, implying
	\[
	H^1(\mathcal{U}, \mathcal{O}) = 0.
	\]
	Because the covering $\mathcal{U}$ is acyclic for $\mathcal{O}$,
	we conclude
	\[
	H^1(\mathbb{P}^1, \mathcal{O}) = 0.
	\]
\end{example}


\begin{proposition}[Mittag--Leffler Problem]
	On a complex manifold $X$, if $H^1(X, \mathcal{O}_X) = 0$, 
	then the \emph{Mittag--Leffler problem} for meromorphic functions on $X$ 
	has a solution.
\end{proposition}

\begin{proof}
	Let $\mathcal{U} = \{ U_\alpha \}_{\alpha \in A}$ be an acyclic open cover of $X$ for the sheaf $\mathcal{O}_X$.
	Suppose that on each $U_\alpha$ we are given a meromorphic function 
	$f_\alpha \in \mathcal{M}(U_\alpha)$,
	and that on overlaps $U_{\alpha\beta} := U_\alpha \cap U_\beta$ 
	the differences
	\[
	s_{\alpha\beta} := f_\alpha - f_\beta \in \Gamma(U_{\alpha\beta}, \mathcal{O}_X)
	\]
	are holomorphic. 
	
	First, we check that $\{ s_{\alpha\beta} \}$ defines a Čech $1$--cocycle in $Z^1(\mathcal{U}, \mathcal{O}_X)$.
	Recall that the Čech coboundary for a $1$--cochain $\{s_{\alpha\beta}\}$ is given by
	\[
	(\delta s)_{\alpha\beta\gamma}
	= s_{\beta\gamma} - s_{\alpha\gamma} + s_{\alpha\beta}.
	\]
	Substituting $s_{\mu\nu} = f_\mu - f_\nu$, we get
	\[
	(\delta s)_{\alpha\beta\gamma}
	= (f_\beta - f_\gamma)
	- (f_\alpha - f_\gamma)
	+ (f_\alpha - f_\beta)
	= 0.
	\]
	Hence $(\delta s)_{\alpha\beta\gamma} = 0$ on all triple intersections, and therefore
	$\{ s_{\alpha\beta} \} \in Z^1(\mathcal{U}, \mathcal{O}_X)$.
	
	Since $H^1(X, \mathcal{O}_X) = 0$ and $\mathcal{U}$ is acyclic,
	we have $H^1(\mathcal{U}, \mathcal{O}_X) = 0$.
	This means that every $1$--cocycle is a $1$--coboundary; that is,
	\[
	\{ s_{\alpha\beta} \} \in B^1(\mathcal{U}, \mathcal{O}_X).
	\]
	Hence there exists a $0$--cochain $\{ c_\alpha \}$ with $c_\alpha \in \Gamma(U_\alpha, \mathcal{O}_X)$ such that
	\[
	(\delta c)_{\alpha\beta} = c_\beta - c_\alpha = s_{\alpha\beta}.
	\]
	By the definition of $s_{\alpha\beta}$, this means
	\[
	f_\alpha - f_\beta = c_\beta - c_\alpha
	\quad \text{on } U_{\alpha\beta}.
	\]
	
	Now define holomorphic functions
	\[
	h_\alpha := f_\alpha + c_\alpha \quad \text{on } U_\alpha.
	\]
	On intersections $U_{\alpha\beta}$ we have
	\[
	h_\alpha - h_\beta
	= (f_\alpha - f_\beta) + (c_\alpha - c_\beta)
	= s_{\alpha\beta} - s_{\alpha\beta} = 0.
	\]
	Thus the functions $h_\alpha$ agree on overlaps and therefore glue together to give a global meromorphic function
	\[
	h \in \Gamma(X, \mathcal{M}_X),
	\quad \text{with } h|_{U_\alpha} = h_\alpha = f_\alpha + c_\alpha.
	\]
	
	On each $U_\alpha$, the difference between $h$ and $f_\alpha$ is
	\[
	h|_{U_\alpha} - f_\alpha = c_\alpha,
	\]
	where $c_\alpha$ is holomorphic. Since adding a holomorphic function does not change the polar part of a meromorphic function, the poles and their coefficients of $h$ and $f_\alpha$ coincide.
	In the quotient sheaf $\mathcal{M}_X / \mathcal{O}_X$, this means
	\[
	[h|_{U_\alpha}] = [f_\alpha],
	\]
	so $h$ and $f_\alpha$ have the same principal part on $U_\alpha$.
	
	Consequently, $h$ is a global meromorphic function on $X$ having on each $U_\alpha$ the prescribed principal part of $f_\alpha$.
	This proves that the Mittag--Leffler problem has a global solution.
\end{proof}

 \medskip
 
For the multiplicative sheaf $\mathcal{O}^*$ and an acyclic cover $\mathcal{U}$ for $\mathcal{O}^*$, we have
\[
H^1(X,\mathcal{O}^*) = H^1(\mathcal{U},\mathcal{O}^*) = \{1\}
\]
if and only if every $1$--cocycle in $\mathcal{O}^*$ is a $1$--coboundary.

A \emph{$1$--cocycle} in the multiplicative sheaf has components
\[
\{\phi_{\alpha\beta}\}_{\alpha,\beta}
\quad \text{such that} \quad
\phi_{\alpha\beta}\phi_{\beta\gamma}\phi_{\gamma\alpha} = 1
\quad \text{on } U_{\alpha\beta\gamma}.
\]
It is a \emph{$1$--coboundary} if there exist $\{\psi_\alpha\}$ (a $0$--cochain with values in $\mathcal{O}^*$) such that
\[
\phi_{\alpha\beta} = (\delta^0\psi)_{\alpha\beta} = \frac{\psi_\beta}{\psi_\alpha}.
\]

Hence
\[
H^1(X,\mathcal{O}^*) = \{1\}
\quad \Longleftrightarrow \quad
\forall\,\{\phi_{\alpha\beta}\}\in Z^1(\mathcal{U},\mathcal{O}^*),
\ \exists\,\{\psi_\alpha\} \text{ with } 
\phi_{\alpha\beta} = \frac{\psi_\beta}{\psi_\alpha}.
\]

 
 \medskip
 
 Then $\mathrm{Pic}(X)$ is trivial if and only if $H^1(X, \mathcal{O}^*) = 0.$
 
 \medskip
 
 Since holomorphic line bundles on $X$ are described by their transition functions
 $\{\phi_{\alpha\beta}\} \in Z^1(\mathcal{U}, \mathcal{O}^*)$,
 two line bundles $E$ and $E'$ defined on the same cover by
 $\{\phi_{\alpha\beta}\}$ and $\{\phi'_{\alpha\beta}\}$ are \emph{isomorphic}
 if there exist nowhere–vanishing holomorphic functions $\{\psi_\alpha\}$ on $\{U_\alpha\}$
 such that
 \[
 \frac{\phi'_{\alpha\beta}}{\phi_{\alpha\beta}}
 = \frac{\psi_\beta}{\psi_\alpha}.
 \]
 That is, the ratio of their $1$--cocycles is a $1$--coboundary.
 
 \medskip
 
 Equation above means the ratio 
 $\{\phi'_{\alpha\beta}/\phi_{\alpha\beta}\}$ 
 is of the special form
 \[
 \frac{\phi'_{\alpha\beta}}{\phi_{\alpha\beta}}
 = (\delta^0 \psi)_{\alpha\beta}
 = \frac{\psi_\beta}{\psi_\alpha},
 \]
 that is, $\phi'/\phi \in B^1(\mathcal{U},\mathcal{O}^*)$, the group of $1$--coboundaries.
 
 Thus two cocycles $\phi,\phi'\in Z^1(\mathcal{U},\mathcal{O}^*)$
 represent \emph{isomorphic line bundles}
 if and only if they differ by a coboundary.
 This defines an equivalence relation on $Z^1(\mathcal{U},\mathcal{O}^*)$:
 \[
 \phi' \sim \phi
 \quad \Longleftrightarrow \quad
 \exists\,\psi\ \text{such that}\ 
 \phi'_{\alpha\beta} = (\delta^0\psi)_{\alpha\beta}\phi_{\alpha\beta}.
 \]
 
 \medskip
 
 Hence the set of isomorphism classes of holomorphic line bundles
 is precisely the group of $1$--cocycles modulo $1$--coboundaries:
 \[
 \mathrm{Pic}(X)
 = Z^1(\mathcal{U}, \mathcal{O}^*) / B^1(\mathcal{U}, \mathcal{O}^*)
 \cong H^1(X, \mathcal{O}^*).
 \]
 
 In particular, $\mathrm{Pic}(X)$ is \emph{trivial} 
 (i.e.\ every holomorphic line bundle over $X$ is isomorphic to the trivial one)
 if and only if
 \[
 H^1(X, \mathcal{O}^*) = 0,
 \]
 since this condition means every $1$--cocycle $\{\phi_{\alpha\beta}\}$
 is a $1$--coboundary of the form $\phi_{\alpha\beta} = \psi_\beta / \psi_\alpha$,
 so all transition functions can be trivialized by changing local frames.
 
 
 
 \section{Sequences in Sheaf Cohomology}

 
 
 
\begin{definition}[Short exact sequence of sheaves]
	\label{def:short_exact_sequence}
	A sequence of sheaves
	\[
	0 \longrightarrow \mathcal{F}' 
	\xrightarrow{i} \mathcal{F}
	\xrightarrow{\sigma} \mathcal{F}'' 
	\longrightarrow 0
	\]
	is called a short exact sequence of sheaves if and only if:
	\begin{enumerate}
		\item $i$ is injective;
		\item $\sigma$ is surjective;
		\item $\mathrm{im}(i) = \ker(\sigma)$.
	\end{enumerate}
\end{definition}

\begin{theorem}[Long exact sequence in cohomology]
	\label{thm:long_exact_sequence}
	Given a short exact sequence of sheaves as in Definition~\ref{def:short_exact_sequence},
	\[
	0 \longrightarrow \mathcal{F}' 
	\xrightarrow{i} \mathcal{F}
	\xrightarrow{\sigma} \mathcal{F}'' 
	\longrightarrow 0,
	\]
	there exists a natural long exact sequence in sheaf cohomology:
	\[
	\begin{aligned}
		0 \longrightarrow
		&H^0(X, \mathcal{F}')
		\xrightarrow{i_*} H^0(X, \mathcal{F})=\Gamma(X,\mathcal{F})
		\xrightarrow{\sigma_*} H^0(X, \mathcal{F}'')
		\xrightarrow{\delta} H^1(X, \mathcal{F}') \\
		&\xrightarrow{i_*} H^1(X, \mathcal{F})
		\xrightarrow{\sigma_*} H^1(X, \mathcal{F}'')
		\xrightarrow{\delta} H^2(X, \mathcal{F}')
		\longrightarrow \cdots
	\end{aligned}
	\]
	where $\delta$ denotes the connecting homomorphisms in cohomology.
\end{theorem}

\begin{example}[Connecting homomorphism in the long exact sequence]
	\label{ex:connecting_map}
	Let $s \in H^0(X, \mathcal{F}'') = \Gamma(X, \mathcal{F}'')$ be a global section. 
	Choose an open cover $\mathcal{U} = \{ U_\alpha \}$ of $X$ such that, 
	by Definition~\ref{def:short_exact_sequence}, each restriction 
	$s|_{U_\alpha}$ can be locally lifted to a section 
	$t_\alpha \in \Gamma(U_\alpha, \mathcal{F})$ satisfying 
	\[
	\sigma(t_\alpha) = s|_{U_\alpha}.
	\]
	
	On the overlaps $U_{\alpha\beta} = U_\alpha \cap U_\beta$, we have
	\[
	\sigma_*(t_\alpha - t_\beta)
	= \sigma(t_\alpha) - \sigma(t_\beta)
	= s|_{U_\alpha} - s|_{U_\beta}
	= 0.
	\]
	Hence
	\[
	u_{\alpha\beta} := t_\alpha - t_\beta 
	\in \Gamma(U_{\alpha\beta}, \mathcal{F}').
	\]
	
	Moreover, on triple overlaps $U_{\alpha\beta\gamma}$,
	\[
	u_{\alpha\beta} + u_{\beta\gamma} + u_{\gamma\alpha} = 0,
	\]
	so that $\{u_{\alpha\beta}\}$ is a $1$--cocycle:
	\[
	\{ u_{\alpha\beta} \} \in Z^1(\mathcal{U}, \mathcal{F}').
	\]
	
	The cohomology class $[ \{ u_{\alpha\beta} \} ] \in H^1(X, \mathcal{F}')$ 
	is by definition the image $\delta(s)$ of $s$ under the connecting homomorphism
	\[
	\delta : H^0(X, \mathcal{F}'') \longrightarrow H^1(X, \mathcal{F}').
	\]
\end{example}
 
\underline{Claim:} 
\[
\varinjlim_{\mathcal{U}} H^1(\mathcal{U}, \mathcal{O}^*) \cong \mathrm{Pic}(X).
\]

\begin{remark}
	$\mathrm{Pic}(X)$ is trivial $\;\Longleftrightarrow\; H^1(X, \mathcal{O}^*) = \{1\}.$
\end{remark}

\noindent
\textbf{Sketch of the idea.}
Let $L$ and $L'$ be holomorphic line bundles on $X$, defined respectively by transition functions
\[
\{\phi_{\alpha\beta}\} \subset \mathcal{O}^*(U_{\alpha\beta})
\quad \text{and} \quad
\{\phi'_{\alpha\beta}\} \subset \mathcal{O}^*(U_{\alpha\beta})
\]
on an open cover $\mathcal{U} = \{U_\alpha\}$ of $X$.
These satisfy
\[
\phi_{\alpha\beta} \phi_{\beta\alpha} = 1, 
\qquad
\phi_{\alpha\beta} \phi_{\beta\gamma} \phi_{\gamma\alpha} = 1,
\]
for all indices $\alpha, \beta, \gamma$,
which means that $\{\phi_{\alpha\beta}\}$ defines a Čech $1$--cocycle:
\[
\{\phi_{\alpha\beta}\} \in Z^1(\mathcal{U}, \mathcal{O}^*).
\]

\medskip

\noindent
Two bundles $L$ and $L'$ are isomorphic if and only if their transition functions 
differ by a Čech 1--coboundary; that is, there exist local nowhere-vanishing holomorphic functions 
$\{\psi_\alpha\} \subset \mathcal{O}^*(U_\alpha)$ such that
\[
\frac{\phi'_{\alpha\beta}}{\phi_{\alpha\beta}}
= \frac{\psi_\alpha}{\psi_\beta}, 
\qquad \forall\, \alpha, \beta.
\]
Hence $\{\phi_{\alpha\beta}\}$ and $\{\phi'_{\alpha\beta}\}$ represent the same cohomology class in 
$H^1(\mathcal{U}, \mathcal{O}^*)$.

\medskip

\noindent
The correspondence
\[
[\{\phi_{\alpha\beta}\}] \;\longmapsto\; [L]
\]
defines a bijection between elements of $H^1(\mathcal{U}, \mathcal{O}^*)$ and 
isomorphism classes of line bundles defined with respect to the cover $\mathcal{U}$.
Passing to the direct limit over refinements of covers gives the natural identification
\[
\varinjlim_{\mathcal{U}} H^1(\mathcal{U}, \mathcal{O}^*) \cong \mathrm{Pic}(X).
\]



\begin{theorem}[Exponential sequence on a Riemann surface $X$]
	There is a short exact sequence of sheaves
	\[
	\begin{array}{ccccccccc}
		0 & \longrightarrow & \underline{\mathbb{Z}} 
		& \longrightarrow & \mathcal{O} 
		& \xrightarrow{\;\exp(2\pi i \cdot)\;} & \mathcal{O}^* 
		& \longrightarrow & 1, \\[1ex]
		& & & & f_x & \longmapsto & e^{2\pi i f_x} & &
	\end{array}
	\]
	called the \emph{exponential sequence} on $X$.
	
	\medskip
	
	The associated long exact sequence in cohomology is
	\[
	H^1(X, \mathcal{O})
	\;\longrightarrow\;
	H^1(X, \mathcal{O}^*) \cong \mathrm{Pic}(X)
	\;\xrightarrow{\,c_1\,}\;
	H^2(X, \underline{\mathbb{Z}})
	\;\longrightarrow\;
	H^2(X, \mathcal{O}),
	\]
	where $c_1$ denotes the first Chern class map.
\end{theorem}

\begin{example}[$X = \mathbb{P}^1$]
	For the Riemann sphere $\mathbb{P}^1$, the exponential sequence yields
	\[
	H^1(\mathbb{P}^1, \mathcal{O})
	\;\longrightarrow\;
	\mathrm{Pic}(\mathbb{P}^1)
	\;\longrightarrow\;
	H^2(\mathbb{P}^1, \mathbb{Z}) = \mathbb{Z}
	\;\longrightarrow\;
	H^2(\mathbb{P}^1, \mathcal{O}) = 0.
	\]
	Hence
	\[
	\mathrm{Pic}(\mathbb{P}^1) \cong \mathbb{Z}.
	\]
\end{example}

\begin{example}[$X = \mathbb{C}/L$ (complex torus)]
	On a complex torus $X = \mathbb{C}/L$ with lattice $L \cong \mathbb{Z}^2$, the corresponding portion of the long exact sequence is
	\[
	\begin{array}{ccccccccc}
		H^1(X, \underline{\mathbb{Z}}) & \hookrightarrow & H^1(X, \mathcal{O})
		& \longrightarrow & H^1(X, \mathcal{O}^*)=\mathrm{Pic}(X)
		& \xrightarrow{\,c_1\,} & H^2(X, \underline{\mathbb{Z}}) & \longrightarrow & 0 \\[2.5ex]
		
		\shortparallel & & & & \shortparallel & & \\[-1ex]
		
		\mathbb{Z}^2 & \longrightarrow & \mathbb{C}
		& \twoheadrightarrow &
		\mathbb{C}/\mathbb{Z}^2 = \mathrm{Jac}(X) = \mathrm{Pic}_0(X). & &
	\end{array}
	\]
	Thus, $\mathrm{Pic}_0(X)$ is the group of degree-zero line bundles on $X$, and one has
	\[
	\mathrm{Pic}(X) \;\cong\; \mathrm{Pic}_0(X) \times \mathbb{Z},
	\]
	where the integer factor corresponds to the image of the first Chern class 
	$c_1 : H^1(X,\mathcal{O}^*) \to H^2(X,\mathbb{Z})$.
\end{example}
 
	
\end{document}