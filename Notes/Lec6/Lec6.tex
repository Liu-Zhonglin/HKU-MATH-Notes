\documentclass[12pt]{article}

% Packages for math and formatting
\usepackage[utf8]{inputenc}
\usepackage{amsmath, amssymb, amsthm}
\usepackage{geometry}
\usepackage{enumitem}
\usepackage{hyperref}
\usepackage{mathtools}
\usepackage{epigraph}
\usepackage{tikz}
\usetikzlibrary{arrows.meta, decorations.pathmorphing, positioning}

% Page setup
\geometry{margin=1in}

% Theorem-like environments
\newtheorem{theorem}{Theorem}[section]
\newtheorem{proposition}[theorem]{Proposition}
\newtheorem{lemma}[theorem]{Lemma}
\newtheorem{corollary}[theorem]{Corollary}
\theoremstyle{definition}
\newtheorem{definition}[theorem]{Definition}
\newtheorem{problem}[theorem]{Problem}
\theoremstyle{remark}
\newtheorem{remark}[theorem]{Remark}
\newtheorem{example}[theorem]{Example}

% Custom commands
\newcommand{\R}{\mathbb{R}}
\newcommand{\N}{\mathbb{N}}
\newcommand{\Q}{\mathbb{Q}}
\newcommand{\Z}{\mathbb{Z}}
\newcommand{\C}{\mathbb{C}}
\newcommand{\PP}{\mathbb{P}}
\newcommand{\D}{\mathbb{D}}
\newcommand{\HH}{\mathbb{H}} % upper half-plane

% Operator names
\DeclareMathOperator{\Aut}{Aut}
\DeclareMathOperator{\FractLin}{FractLin}
\DeclareMathOperator{\ord}{ord}
\DeclareMathOperator{\Res}{Res}
\DeclareMathOperator{\SL}{SL}

\newcommand{\PSL}{\mathrm{PSL}}

\title{Compact Riemann surfaces of genus $g\geq 2$,\\
Part 1}
\author{Joe}
\date{November 5, 2025}

\begin{document}
	
	\maketitle
	

	

	
	
	
	
	By the \textbf{Uniformization Theorem}, any simply connected Riemann surface is
	biholomorphic to exactly one of the following:
	\[
	\widehat{\mathbb{C}}, \qquad \mathbb{C}, \qquad \text{or} \qquad 
	\mathbb{D} = \{\, z \in \mathbb{C} : |z| < 1 \,\}.
	\]
	In particular, if $X$ is a compact Riemann surface of genus $g \ge 2$, its
	universal covering surface is the unit disk~$\mathbb{D}$, and hence
	\[
	X \;\cong\; \mathbb{D} / \Gamma,
	\]
	where $\Gamma \subset \Aut(\mathbb{D})$ is a discrete subgroup of Möbius
	transformations acting properly discontinuously (a \emph{Fuchsian group}).
	
	\medskip
	
	The unit disk~$\mathbb{D}$ and the upper half‑plane
	\[
	\mathbb{H} = \{\, \tau \in \mathbb{C} \mid \Im(\tau) > 0 \,\}
	\]
	are conformally equivalent.
	The equivalence is given by the \textbf{Cayley transform}
	\[
	z = \frac{\tau - i}{\tau + i}, \qquad 
	\tau = i\,\frac{1 + z}{1 - z}.
	\]
	Thus, one can work equally well with the model
	$\mathbb{D}/\Gamma \subset \Aut(\mathbb{D})$ or with the equivalent model
	$\mathbb{H}/\Gamma' \subset \PSL_2(\mathbb{R})$,
	where $\Gamma' = C^{-1} \Gamma C$ under the Cayley conjugation~$C$.
	
\section*{Poincaré Series}

Let \( X = \mathbb{D} / \Gamma \) be a compact Riemann surface of genus 
\( g \ge 2 \), where 
\( \Gamma \subset \Aut(\mathbb{D}) \) is a discrete group 
(of Möbius transformations acting properly discontinuously on~\(\mathbb{D}\)).

\medskip

Let \( h \in H^\infty(\mathbb{D}) \), i.e.\ a bounded holomorphic function on~\(\mathbb{D}\),
and let \( k \ge 2 \) be an integer.  
Define the \textbf{Poincaré series} of weight \(k\) associated to \(h\) by
\[
\boxed{
	P_k^\Gamma(h)(z)
	\;=\;
	\sum_{\gamma \in \Gamma}
	h(\gamma z) \, (\gamma'(z))^k,
	\qquad z \in \mathbb{D}.
}
\]

Then the series
\[
f(z) = P_k^\Gamma(h)(z)
\]
converges \emph{uniformly on compact subsets} of~\(\mathbb{D}\); hence \(f\) defines 
a holomorphic function on~\(\mathbb{D}\).
Moreover, for every $\gamma \in \Gamma$, we have
\[
f(\gamma z) = \frac{1}{(\gamma'(z))^k}\, f(z).
\]

Thus $f$ is a $\Gamma$‑automorphic holomorphic function of weight $k$.




\paragraph{Preparation}

Let \( X = \mathbb{D} / \Gamma \) be a compact Riemann surface of genus \( g \ge 2 \),
where \( \Gamma \subset \Aut(\mathbb{D}) \) is a discrete group of Möbius transformations.

\medskip

A \textbf{holomorphic $1$‑form} on \(X\) can be lifted to a holomorphic $1$‑form 
on the disk~\(\mathbb{D}\) of the form \( f(z)\,dz \).
It descends to a well‑defined form on~\(X\) if and only if it is
\(\Gamma\)-invariant, i.e.
\[
f(\gamma z)\, d(\gamma z) = f(z)\,dz
\qquad 
\forall\, \gamma \in \Gamma,\, z \in \mathbb{D}.
\]
Since
\[
\frac{d(\gamma z)}{dz} = \gamma'(z) \quad \Rightarrow \quad d(\gamma z) = \gamma'(z)\,dz,
\]
the invariance condition becomes
\[
f(\gamma z)\, \gamma'(z)\,dz = f(z)\,dz
\quad\Longleftrightarrow\quad
f(\gamma z) = \frac{1}{\gamma'(z)}\, f(z).
\]

\medskip

A holomorphic $1$‑form is also called a \emph{holomorphic $1$‑differential}.
More generally, for an integer \(k \ge 1\), a \textbf{holomorphic $k$‑differential}
is a local expression of the form
\[
f(z)\,(dz)^k,
\]
subject to the following transformation rule:
if we make a holomorphic change of coordinate \(z \mapsto w(z)\), then
\[
dw = \frac{dw}{dz}\,dz
\quad\text{and hence}\quad
(dw)^k = \Big(\frac{dw}{dz}\Big)^k (dz)^k.
\]

\begin{definition}[Holomorphic $k$‑differential on $X = \mathbb{D}/\Gamma$]
	Let $\Gamma \subset \Aut(\mathbb{D})$ act properly discontinuously on~$\mathbb{D}$.
	A \emph{holomorphic $k$‑differential} on $X = \mathbb{D}/\Gamma$
	is a $\Gamma$‑invariant tensor on~$\mathbb{D}$ of the form
	\[
	g(z) = f(z)\,(dz)^k,
	\]
	where \(f\) is holomorphic on~\(\mathbb{D}\) and satisfies the transformation law
	\[
	\boxed{
		f(\gamma z) = \frac{1}{(\gamma'(z))^k}\, f(z)
		\qquad 
		\forall\, \gamma \in \Gamma,\ \forall\, z \in \mathbb{D}.
	}
	\]
	Equivalently,
	\[
	\gamma^* g = g
	\quad\text{for all }\gamma \in \Gamma,
	\]
	since
	\[
	\gamma^*\big(f(z)(dz)^k\big)
	= f(\gamma z)\,(d(\gamma z))^k
	= f(\gamma z)\,(\gamma'(z))^k\,(dz)^k.
	\]
\end{definition}

\medskip

In particular, when \( f = P_k^{\Gamma}(h) \) is a Poincaré series constructed from a bounded holomorphic function \( h \in H^\infty(\mathbb{D}) \),
the expression
\[
f(z)\,(dz)^k
\]
is $\Gamma$‑invariant and hence defines a holomorphic $k$‑differential on the quotient Riemann surface \(X\).

\begin{remark}
	\leavevmode
	\begin{enumerate}
		\item Even if \(h \in H^\infty(\mathbb{D})\) (i.e.\ $h$ is bounded on the disk),
		the associated Poincaré series \(f = P_k^{\Gamma}(h)\) need not be bounded on~\(\mathbb{D}\).
		\item There exist linearly independent holomorphic $1$‑forms on~\(X\)
		which \emph{cannot} be obtained as Poincaré series of the type \(P_k^{\Gamma}(h)\).
	\end{enumerate}
\end{remark}


\begin{proof}
	We first check	\textbf{$\Gamma$‑invariance.}
	
	Assume that
	\[
	f = P_k^{\Gamma}(h)(z)
	= \sum_{\gamma \in \Gamma} h(\gamma z)\,(\gamma'(z))^k
	\]
	is absolutely convergent and uniformly convergent on compact subsets of~\(\mathbb{D}\).
	Then \(f(z)\) defines a holomorphic function on~\(\mathbb{D}\), i.e.\ \(f \in H(\mathbb{D})\).
	
	
	Let \(\mu \in \Gamma\).  
	We compute \(f(\mu z)\):
	\[
	f(\mu z)
	= \sum_{\gamma \in \Gamma} h(\gamma(\mu z))\,(\gamma'(\mu z))^k.
	\]
	By the chain rule,
	\[
	(\gamma \circ \mu)'(z) = \gamma'(\mu z)\,\mu'(z),
	\quad\text{so}\quad
	\gamma'(\mu z) = \frac{(\gamma \mu)'(z)}{\mu'(z)}.
	\]
	Substituting this into the series, we get
	\[
	f(\mu z)
	= \sum_{\gamma \in \Gamma}
	h((\gamma \mu)(z))
	\left(\frac{(\gamma \mu)'(z)}{\mu'(z)}\right)^k.
	\]
	
	Since \(\Gamma\) is a group, the set \(\{\gamma \mu : \gamma \in \Gamma\}\)
	is again all of~\(\Gamma\).  
	Write \(\delta = \gamma \mu\).  
	Then the sum becomes
	\[
	f(\mu z)
	= \frac{1}{(\mu'(z))^k}
	\sum_{\delta \in \Gamma} h(\delta z)\,(\delta'(z))^k.
	\]
	Hence
	\[
	f(\mu z)
	= \frac{1}{(\mu'(z))^k}\, f(z),
	\quad \forall\, \mu \in \Gamma,\ z \in \mathbb{D}.
	\]
	
	\medskip
	\noindent
Multiplying both sides by \((\mu'(z))^k (dz)^k\),
	we find
	\[
	f(\mu z)\,(\mu'(z))^k\,(dz)^k
	= f(z)\,(dz)^k,
	\]
	or, in invariant form,
	\[
	f(\mu z)\,(d(\mu z))^k = f(z)\,(dz)^k.
	\]
	Thus \(f(z)(dz)^k\) is a \(\Gamma\)-invariant holomorphic \(k\)-differential on \(X = \mathbb{D}/\Gamma\).
	
\medskip
\noindent


It remains to verify that, for \(k \ge 0\), the series
\[
P_k^{\Gamma}(h)(z)
= \sum_{\gamma \in \Gamma} h(\gamma z)\,(\gamma'(z))^k
\]
is \textbf{absolutely and uniformly convergent} on compact subsets of~\(\mathbb{D}\).

\medskip
\noindent
\textbf{(a) Preliminary estimate.}
For any finite subset \(\Gamma_0 \subset \Gamma\),
\[
|f(z)|
= \Big|
\sum_{\gamma \in \Gamma_0}
h(\gamma z)\,(\gamma'(z))^k
\Big|
\le
\sum_{\gamma \in \Gamma}
|h(\gamma z)|\,|\gamma'(z)|^{k}.
\]
Since \(h \in H^\infty(\mathbb{D})\), there exists \(M > 0\) with
\(|h(z)| \le M\) for all \(z \in \mathbb{D}\).
Hence,
\[
|f(z)| \le
M \sum_{\gamma \in \Gamma} |\gamma'(z)|^k.
\]
We will show that
\[
\sum_{\gamma \in \Gamma} |\gamma'(z)|^2 \le C_K
\quad\text{for all } z \in K,
\]
for each compact \(K \subset \mathbb{D}\) and some constant \(C_K > 0\).
This implies absolute and uniform convergence on \(K\).

\medskip
\noindent
\textbf{(b) Integral estimate via disjointness of fundamental domains.}

Let \(U \subset \mathbb{D}\) be a relatively compact open set such that
\(\gamma(U) \cap U = \emptyset\) for all \(\gamma \ne \mathrm{id}\).
Then, for each \(\gamma \in \Gamma\),
by letting \(w = \gamma(z)\) (so that \(dw = \gamma'(z)\,dz\)),
\[
\int_U |\gamma'(z)|^2\,|dz|^2
= \int_{\gamma(U)} |dw|^2
= \mathrm{Area}(\gamma(U)).
\]
Summing over all \(\gamma\),
\[
\int_U 
\sum_{\gamma \in \Gamma} |\gamma'(z)|^2\,|dz|^2
\;\le\;
\sum_{\gamma \in \Gamma} \mathrm{Area}\bigl(\gamma(U)\bigr)
=
\mathrm{Area}\!\left(\bigcup_{\gamma \in \Gamma} \gamma(U)\right)
=\mathrm{Area}(\mathbb{D}) 
< \infty.
\]
Thus the function \(\sum_{\gamma \in \Gamma} |\gamma'(z)|^2\) is locally integrable and bounded in average.

\medskip
\noindent
\textbf{(c) From integral to pointwise estimates.}

Let \(S(z)\) be any holomorphic function on a disk \(D(a,R)\).
By the \(L^2\) mean‑value property,
\[
|S(a)|^2
\le
\frac{1}{\pi R^2}
\int_{D(a,R)} |S(z)|^2\,dx\,dy.
\]
Indeed, if \(S(z) = \sum_{n=0}^{\infty} a_n (z - a)^n\),
then
\[
\int_{D(a,R)} |S(z)|^2\,dx\,dy
= \int_0^R \int_0^{2\pi}
\Big|\sum_{n} a_n r^n e^{in\theta}\Big|^2 r\,d\theta\,dr
= \pi \sum_{n} |a_n|^2 \frac{R^{2n+2}}{n+1},
\]
and the orthogonality of \(e^{in\theta}\) implies
\(|a_0|^2 \le \frac{1}{\pi R^2} \int_{D(a,R)} |S(z)|^2\,dx\,dy.\)

\medskip
Applying this to the (partial sums of the) function
\( S(z) = P_k^{\Gamma}(h)(z) \)
and the integral estimate from part (b), we obtain
for every compact \(K \subset \mathbb{D}\)
that there exists a constant \(C_K > 0\) with
\[
\sup_{z \in K} \sum_{\gamma \in \Gamma} |\gamma'(z)|^2 \le C_K.
\]
Therefore the series
\[
\sum_{\gamma \in \Gamma} h(\gamma z)(\gamma'(z))^k
\]
converges absolutely and uniformly on compact subsets of~\(\mathbb{D}\).

This completes the proof of convergence.
\end{proof}







	
\end{document}