\documentclass[12pt]{article}

% Packages for math and formatting
\usepackage[utf8]{inputenc}
\usepackage{amsmath, amssymb, amsthm, mathtools} % core math
\usepackage{dsfont} 
\usepackage{geometry}
\usepackage{enumitem}
\usepackage{hyperref}
\usepackage{mathtools}
\usepackage{epigraph}
\usepackage[all]{xy} 
\usepackage{tikz}
\usepackage{mathrsfs}
\usetikzlibrary{arrows.meta, decorations.pathmorphing, positioning}

% Page setup
\geometry{margin=1in}

% Theorem-like environments
\newtheorem{theorem}{Theorem}[section]
\newtheorem{proposition}[theorem]{Proposition}
\newtheorem{lemma}[theorem]{Lemma}
\newtheorem{corollary}[theorem]{Corollary}
\theoremstyle{definition}
\newtheorem{definition}[theorem]{Definition}
\newtheorem{problem}[theorem]{Problem}
\theoremstyle{remark}
\newtheorem{remark}[theorem]{Remark}
\newtheorem{example}[theorem]{Example}

% Custom commands
\newcommand{\R}{\mathbb{R}}
\newcommand{\N}{\mathbb{N}}
\newcommand{\Q}{\mathbb{Q}}
\newcommand{\Z}{\mathbb{Z}}
\newcommand{\C}{\mathbb{C}}
\newcommand{\PP}{\mathbb{P}}
\newcommand{\D}{\mathbb{D}}
\newcommand{\HH}{\mathbb{H}} % upper half-plane

% Operator names
\DeclareMathOperator{\Aut}{Aut}
\DeclareMathOperator{\FractLin}{FractLin}
\DeclareMathOperator{\ord}{ord}
\DeclareMathOperator{\Res}{Res}
\DeclareMathOperator{\SL}{SL}

\title{Holomorphic Line Bundles}
\author{Joe}
\date{November 26, 2025}

\begin{document}
	
	\maketitle
	
	
		\tableofcontents	
		
		\vspace{1em}
		
\underline{Backgroud}: Let $X$ be a compact Riemann surface and let $\mathcal{M}(X)$ denote the field of meromorphic functions on $X$.

\begin{itemize}
	\item For $g=0$, we have $X \cong \mathbb{P}^1$ and 
	\[
	\mathcal{M}(\mathbb{P}^1) = \{ \text{rational functions} \}.
	\]
	
	\item For $g=1$, we have $X = \mathbb{C}/L$ (a complex torus) and 
	\[
	\mathcal{M}(X) \cong \{ \text{elliptic functions} \}.
	\]
	In this case, $\mathcal{M}(X)$ is generated by the Weierstrass functions $\wp$ and $\wp'$ satisfying the relation
	\[
	(\wp')^2 = 4\wp^3 + g_2 \wp + g_3,
	\]
	for some constants $g_2, g_3$.
	
	\item For $g \ge 2$, we have $X = D / \Gamma$, where $D$ is the unit disk and $\Gamma$ is a discrete subgroup of $\mathrm{Aut}(D)$. 
	In this case, the space of automorphic forms
	\[
	P_k^\Gamma(h), \quad h \in H^\infty(D),
	\]
	is used, and for sufficiently large $k$, the corresponding Poincaré series define an embedding
	\[
	X \hookrightarrow \mathbb{P}^N.
	\]
\end{itemize}


To obtain more holomorphic or meromorphic functions, we consider holomorphic line bundles. 
Sections of holomorphic line bundles behave like \emph{twisted holomorphic functions}.


\noindent
When $g=0$:
We can write
\[
\mathbb{P}^1 = \{ [t_0 : t_1] \mid t_0,t_1 \in \mathbb{C} \},
\]
so $\mathbb{C} \subset \mathbb{P}^1$ via $z = t_1/t_0$.  
Rational functions on $\mathbb{P}^1$ are of the form
\[
f = \frac{P(t_0,t_1)}{Q(t_0,t_1)},
\]
where $P, Q \in \mathbb{C}[t_0,t_1]$ are homogeneous polynomials of the same degree $\ge 0$, with $Q \ne 0$.

For example,
\[
f([t_0:t_1]) = \frac{t_0^2 + 2t_0t_1}{3t_1^2 + 5t_0^2}
= \frac{1 + 2z}{3z^2 + 5},
\quad \text{where } z = \frac{t_1}{t_0}.
\]


Why is $\dfrac{P}{Q}$ a well-defined function on $\mathbb{P}^1$?  
Because homogeneity makes it independent of the representative $t = (t_0, t_1)$ of a point $[t_0:t_1]$:
\[
P(\lambda t) = P(\lambda t_0, \lambda t_1) = \lambda^d P(t_0,t_1), 
\quad
Q(\lambda t) = \lambda^d Q(t_0,t_1),
\]
so
\[
\frac{P(\lambda t)}{Q(\lambda t)} = \frac{P(t)}{Q(t)}, 
\quad \forall\, \lambda \in \mathbb{C}^*.
\]
Therefore, the quotient $\frac{P}{Q}$ depends only on the point $[t_0:t_1] \in \mathbb{P}^1$, not on its coordinates.\\
	

 
 When $g=1$, we have $X = \mathbb{C}/L$, where $L$ is a lattice in $\mathbb{C}$.
 A \emph{system of factors of automorphy} is a family of holomorphic functions
 \[
 \{\exp(P_\omega z + Q_\omega)\}_{\omega \in L},
 \]
 such that these functions satisfy the compatibility (or cocycle) condition described below.
 

 
 A theta function $\theta$ with respect to the system 
 $\{\exp(P_\omega z + Q_\omega)\}$ satisfies
 \[
 \theta(z + \omega) = \exp(P_\omega z + Q_\omega)\, \theta(z), 
 \qquad \forall\, \omega \in L.
 \]
 Applying this twice gives the requirement for compatibility:
 \[
 \begin{aligned}
 	\theta(z + \omega + \omega')
 	&= \exp(P_{\omega'} (z+\omega)+ Q_{\omega'})\, \theta(z + \omega) \\
 	&= \exp(P_{\omega'}(z+\omega) + Q_{\omega'}) \exp(P_\omega z + Q_\omega)\, \theta(z) \\
 	&= \exp(P_{\omega+\omega'} z + Q_{\omega+\omega'})\, \theta(z).
 \end{aligned}
 \]
 This must hold for all $\omega, \omega' \in L$.

 

 
 Thus, the \emph{compatibility (or cocycle) condition} for the factors of automorphy is
 \[
 \exp(P_{\omega+\omega'} z + Q_{\omega+\omega'})
 = \exp(P_{\omega'}(z+\omega) + Q_{\omega'})\,
 \exp(P_{\omega} z + Q_{\omega}).
 \]
 That is,
 \[
 \boxed{
 	e^{P_{\omega+\omega'} z + Q_{\omega+\omega'}}
 	= e^{P_{\omega'}(z+\omega) + Q_{\omega'}} e^{P_{\omega} z + Q_{\omega}}.
 }\]
 

 
 In general, for a complex manifold $X$ with an open cover 
 \[
 \mathcal{U} = \{\, U_\alpha \,\}_{\alpha \in A}, \qquad
 X = \bigcup_{\alpha \in A} U_\alpha,
 \]
 a system of \emph{transition functions} 
 \[
 \phi_{\alpha\beta} : U_{\alpha\beta} = U_\alpha \cap U_\beta \longrightarrow \mathbb{C}^*
 \]
 encodes a holomorphic line bundle if the following compatibility (cocycle) relations hold:
 \[
 \begin{cases}
 	\text{(a)}~ \phi_{\alpha\alpha} = 1 & \text{on } U_\alpha, \\[4pt]
 	\text{(b)}~ \phi_{\alpha\beta} = (\phi_{\beta\alpha})^{-1} & \text{on } U_{\alpha\beta}, \\[4pt]
 	\text{(c)}~ \phi_{\alpha\beta}\, \phi_{\beta\gamma}\, \phi_{\gamma\alpha} = 1 & \text{on } U_{\alpha\beta\gamma} = U_\alpha \cap U_\beta \cap U_\gamma.
 \end{cases}
 \]
 
 Here each $\phi_{\alpha\beta}$ is a nowhere-vanishing holomorphic function 
 (on its domain of definition).  
 
 In particular, for the elliptic case, we may denote
 \[
 \phi_{\alpha\beta}(x) = \exp(P_{\omega_{\alpha\beta}}(z) + Q_{\omega_{\alpha\beta}}),
 \quad x \in X,
 \]
 and the same cocycle relations (a)--(c) must be satisfied.


\section{Holomorphic Line Bundles}

\begin{definition}[Holomorphic line bundle on a Riemann surface]
	Let $X$ be a Riemann surface. A \emph{holomorphic line bundle} $L$ on $X$ consists of the following data:
	
	\begin{enumerate}
		\item An open covering $\{U_\alpha\}_{\alpha \in A}$ of $X$, that is,
		\[
		U_\alpha \subset X \text{ open}, \qquad X = \bigcup_{\alpha \in A} U_\alpha.
		\]
		
		\item A system of nowhere-vanishing holomorphic functions
		\[
		\phi_{\alpha\beta} : U_{\alpha\beta} = U_\alpha \cap U_\beta \longrightarrow \mathbb{C}^*
		\]
		satisfying the following axioms:
		\begin{enumerate}
			\item[(a)] $\phi_{\alpha\alpha} \equiv 1$ on $U_\alpha$;
			\item[(b)] $\phi_{\alpha\beta} = \dfrac{1}{\phi_{\beta\alpha}}$ on $U_{\alpha\beta}$;
			\item[(c)] $\phi_{\alpha\beta}\, \phi_{\beta\gamma}\, \phi_{\gamma\alpha} \equiv 1$ on $U_{\alpha\beta\gamma} = U_\alpha \cap U_\beta \cap U_\gamma$.
		\end{enumerate}
	\end{enumerate}
\end{definition}

\begin{remark}
	From condition~(c) we may equivalently write
	\[
	\phi_{\alpha\beta}\, \phi_{\beta\gamma}
	= \frac{1}{\phi_{\gamma\alpha}}
	= \phi_{\alpha\gamma},
	\]
	by using (b).
\end{remark}

\noindent
The functions $\phi_{\alpha\beta}$ are called \emph{holomorphic transition functions} (for example, factors of holomorphic powers of the Jacobians $(\gamma')^k$).


\noindent
Geometric meaning of the symbol $L$:
A holomorphic line bundle $L$ is a geometric object defined on a complex manifold \(X\),
which may be one of the familiar spaces such as
\[
\mathbb{C}^n, \quad \mathbb{P}^N, \quad \mathbb{C}^n / \Lambda,
\]
where $\Lambda$ is a lattice.  
Changes of local coordinates are holomorphic maps
\[
w = (w_1, w_2) = \Phi(z) = \Phi(z_1, z_2),
\]
where $\Phi$ is (bi)holomorphic.
	


From a compatible system of transition functions 
$\{\phi_{\alpha\beta}\}$ satisfying conditions (a)--(c), 
we are going to construct a $2$‑dimensional complex manifold \(L\) 
that will serve as the \emph{total space} of a holomorphic line bundle over \(X\).

\bigskip
Define
\[
\mathscr{O} = \bigsqcup_{\alpha} 
\bigl( \{\alpha\} \times U_\alpha \times \mathbb{C} \bigr),
\]
that is, the disjoint union of sets 
\(\{\alpha\}\times U_\alpha \times \mathbb{C}\).
Each piece \(\{\alpha\}\times U_\alpha \times \mathbb{C}\) 
is considered distinct from
\(\{\beta\}\times U_\beta \times \mathbb{C}\)
even if \(U_\alpha\cap U_\beta \neq \varnothing\).

\bigskip
Next, we introduce a relation 
\(\mathcal{E} \subset \mathscr{O}\times \mathscr{O}\)
and write \((A,B)\in\mathcal{E}\) as \(A\sim B\).

\begin{definition}[The relation $\sim=\sim_\mathcal{E}$]
	We declare that
	\[
	(\alpha, x, v^{(\alpha)}) \sim (\beta, y, v^{(\beta)})
	\quad\text{if and only if}\quad
	x = y \in U_{\alpha\beta}
	\ \text{and}\
	v^{(\alpha)} = \phi_{\alpha\beta}(x)\, v^{(\beta)}.
	\]
\end{definition}

\begin{lemma}
	The relation $\sim$ is an equivalence relation on $\mathscr{O}$, hence we can define
$
	L = \mathscr{O} / \sim
$
	as a set.
\end{lemma}



Now that $L = \mathscr{O} / \sim$ is defined as a set, 
we will endow \(L\) with the structure of a $2$‑dimensional complex manifold, 
equipped with a natural projection map
\[
\pi : L \longrightarrow X.
\]



Define
\[
\rho : 
\bigsqcup_{\alpha}
\bigl( \{\alpha\} \times U_\alpha \times \mathbb{C} \bigr)
\;\longrightarrow\; X,
\qquad
\rho(\alpha, x, v^{(\alpha)}) = x.
\]

If \((\alpha, x, v^{(\alpha)}) \sim (\beta, y, v^{(\beta)})\),
then by definition \(x = y\).  
Hence, the map \(\rho\) descends to a well‑defined map on the quotient:
\[
\pi : L = \mathscr{O}/\sim \;\longrightarrow\; X,
\qquad
\pi \bigl( [\,\alpha, x, v^{(\alpha)}\,] \bigr) = x.
\]



Consider an open set \(U_\alpha \subset X\).  
Recall that on the component 
\(\{\alpha\} \times U_\alpha \times \mathbb{C}\),
we have the projection
\[
\rho : \{\alpha\} \times U_\alpha \times \mathbb{C}
\;\longrightarrow\;
U_\alpha,
\qquad
(\alpha, x, v^{(\alpha)}) \mapsto x.
\]

On overlaps \(U_\alpha \cap U_\beta = U_{\alpha\beta}\),
we also have 
\[
\rho :
\{\beta\} \times U_\beta \times \mathbb{C} \;\longrightarrow\; U_\beta.
\]
Using the identification given by \((\alpha, x, v^{(\alpha)}) \sim (\beta, x, v^{(\beta)})\)
where \(v^{(\alpha)} = \phi_{\alpha\beta}(x) v^{(\beta)}\),
we see that
\[
\pi^{-1}(U_\alpha) \cong U_\alpha \times \mathbb{C}.
\]

We denote this by
\[
L|_{U_\alpha} := \pi^{-1}(U_\alpha) \cong U_\alpha \times \mathbb{C},
\]
and call this isomorphism the \emph{local trivialization} of the holomorphic line bundle on \(U_\alpha\).




By construction, we have for each $\alpha$:
\[
L|_{U_\alpha} \;\overset{\cong}{\longrightarrow}\; U_\alpha \times \mathbb{C},
\]
and similarly,
\[
L|_{U_\beta} \;\overset{\cong}{\longrightarrow}\; U_\beta \times \mathbb{C}.
\]

The same point \(v \in L\) projects under \(\pi\) to
\(\pi(v) = x \in U_{\alpha\beta} = U_\alpha \cap U_\beta\).



As a set, we may write:
\[
L = \bigcup_{\alpha \in A} \bigl( U_\alpha \times \mathbb{C} \bigr),
\]
so that a point in \(L\) is represented by \((x, v^{(\alpha)})\)
with respect to the local trivialization
\[
L|_{U_\alpha} := \pi^{-1}(U_\alpha)
\;\cong\;
U_\alpha \times \mathbb{C}.
\]

Likewise, over \(U_\beta\) the same point is represented as
\((x, v^{(\beta)})\).
On the overlap \(U_{\alpha\beta}\), the two descriptions are related by
\[
(x, v^{(\alpha)}) \longleftrightarrow (x, v^{(\beta)}),
\qquad
v^{(\alpha)} = \phi_{\alpha\beta}(x) \, v^{(\beta)},
\qquad
z^{(\alpha)} = z^{(\beta)} = x.
\]
That is,
\[
U_\alpha \times \mathbb{C} \;\overset{\phi_{\alpha\beta}}{\longleftrightarrow}\; 
U_\beta \times \mathbb{C}.
\]



Choose holomorphic coordinate functions
\[
z^{(\alpha)} : U_\alpha \;\overset{\cong}{\longrightarrow}\; D_\alpha \subset \mathbb{C},
\qquad
z^{(\beta)} : U_\beta \;\overset{\cong}{\longrightarrow}\; D_\beta \subset \mathbb{C}.
\]

Then each local chart of \(L\) can be expressed as
\[
U_\alpha \times \mathbb{C} \;\cong\; D_\alpha \times \mathbb{C} \;\subset\; \mathbb{C}^2.
\]

With this identification, the change of variables between overlapping charts is
\[
(z^{(\alpha)}, v^{(\alpha)}) \longleftrightarrow (z^{(\beta)}, v^{(\beta)}),
\qquad
v^{(\alpha)} = \phi_{\alpha\beta}(x)\, v^{(\beta)}.
\]
Thus, transitions are holomorphic mappings between open subsets of \(\mathbb{C}^2\).


\begin{proposition}
	Let \(L = \mathscr{O}/\!\sim\) with the projection map \(\pi : L \to X\)
	as constructed above.
	Then \(L\) carries a natural structure of a $2$‑dimensional complex manifold,
	such that for every \(x \in X\) there exists an open neighborhood
	\(U_x \subset X\) containing \(x\) with
	\[
	\pi^{-1}(U_x) \cong U_x \times \mathbb{C}.
	\]
	Taking \(U_x = U_\alpha\) for some \(\alpha\) with \(x \in U_\alpha\),
	we have the local trivialization
	\[
	L|_{U_\alpha} \cong U_\alpha \times \mathbb{C}.
	\]
	Moreover, on the overlap \(U_{\alpha\beta}\),
	the change of coordinates between the two trivializations
	\[
	(z^{(\alpha)}, v^{(\alpha)}) \leftrightarrow (z^{(\beta)}, v^{(\beta)})
	\]
	is given by
	\[
	v^{(\alpha)} = \phi_{\alpha\beta}(x) \, v^{(\beta)},
	\]
	which, for each \(x \in U_{\alpha\beta}\), defines an isomorphism of
	one‑dimensional complex vector spaces.
\end{proposition}

From now on, we denote by
\[
\pi : L \;\longrightarrow\; X
\]
the \emph{holomorphic line bundle} constructed above.

\begin{itemize}
	\item The $2$‑dimensional complex manifold \(L\) is called the
	\emph{total space} of the holomorphic line bundle.
	
	\item The map \(\pi\) is called the \emph{canonical projection}.
\end{itemize}














\begin{definition}[Holomorphic sections]
	Let \(\pi : L \to X\) be a holomorphic line bundle over a Riemann surface \(X.\)
	
	A \emph{holomorphic section} of \(\pi\) is a map
	\[
	s : X \longrightarrow L
	\]
	satisfying:
	\begin{enumerate}
		\item[(a)] \(s\) is holomorphic as a map between complex manifolds;
		\item[(b)] \(\pi \circ s = \mathrm{id}_X.\)
	\end{enumerate}
\end{definition}


\begin{theorem}
	Let \(X\) be a compact Riemann surface, and 
	\(\pi : E \to X\) a holomorphic line bundle.
	Then the space of holomorphic sections \(\Gamma(X, E)\) is finite-dimensional.
\end{theorem}

\begin{proof}
	Assume \(\Gamma(X,E)\) is nonzero.
	Pick a section \(s_0 \in \Gamma(X,E)\) such that \(s_0 \not\equiv 0.\)
	Choose a point \(x_0 \in X\) where \(s_0(x_0) \ne 0.\)
	
	For any other section \(s \in \Gamma(X,E)\),
	consider the pointwise ratio
	\[
	\frac{s}{s_0}.
	\]
	This is a meromorphic function on \(X\): it is holomorphic wherever \(s_0\) is nonzero,
	and may have poles exactly at the zeros of \(s_0.\)
	
	Choose a holomorphic coordinate \(z\) on a neighborhood \(U\) of \(x_0\)
	such that \(z(x_0) = 0.\)
	Since \(s_0(x_0) \ne 0,\) the quotient \(\frac{s}{s_0}\) is holomorphic near \(x_0\)
	and has a Taylor expansion
	\[
	\frac{s}{s_0}(z) = \sum_{n=0}^{\infty} c_n(s)\, z^n.
	\]
	
	For each integer \(m \ge 0,\) define the linear map
	\[
	\Phi_m : \Gamma(X, E) \longrightarrow \mathbb{C}^{m+1}, 
	\qquad \Phi_m(s) = (c_0(s), c_1(s), \dots, c_m(s)).
	\]
	Clearly, \(\Phi_m\) is complex-linear.
	
	We will prove that there exists \(m\) such that \(\Phi_m\) is injective.
	This implies that \(\Gamma(X,E)\) is finite-dimensional,
	because it injects into the finite-dimensional space \(\mathbb{C}^{m+1}.\)

	
	Let \(N\) be the total number of zeros of \(s_0\), counted with multiplicity.
	Since \(X\) is compact and \(s_0 \not\equiv 0,\) \(N\) is finite.
	
	Now take any \(s \in \Gamma(X,E)\) with \(s \not\equiv 0.\)
	Then \(\frac{s}{s_0}\) is a nonzero meromorphic function on \(X.\)
	Its poles can occur only where \(s_0\) vanishes,
	so the total number of poles (counted with multiplicity) satisfies
	\[
	\#\text{poles}\!\left(\frac{s}{s_0}\right) \le N.
	\]
	Because \(X\) is compact, any nonzero meromorphic function
	has the same total number of zeros and poles (each counted with multiplicity).
	Therefore,
	\[
	\#\text{zeros}\!\left(\frac{s}{s_0}\right)
	= \#\text{poles}\!\left(\frac{s}{s_0}\right)
	\le N.
	\]
	This means that \(\frac{s}{s_0}\) can have at most \(N\) zeros on \(X.\)

	
	Suppose now that \(s \in \ker(\Phi_m).\)
	Then the first \(m+1\) coefficients \(c_0(s), \dots, c_m(s)\) vanish, so
	\[
	\frac{s}{s_0}(z) = z^{m+1} g(z)
	\]
	for some holomorphic function \(g(z)\) near \(z = 0.\)
	Hence, \(\frac{s}{s_0}\) has a zero of order at least \(m+1\) at the point \(x_0.\)
	
	If \(s \not\equiv 0,\) this gives at least \(m+1\) zeros of \(\frac{s}{s_0}\),
	all at the same point \(x_0.\)
	But from the argument above, the total number of zeros of
	\(\frac{s}{s_0}\) cannot exceed \(N.\)
	Therefore, if \(m+1 > N,\) such an \(s\) cannot exist, 
	forcing \(s \equiv 0.\)
	
	Thus, for all \(m\) with \(m+1 > N,\)
	\[
	\ker(\Phi_m) = \{0\}.
	\]
	That is, \(\Phi_m\) is injective.
	

	Since \(\Phi_m : \Gamma(X,E) \to \mathbb{C}^{m+1}\)
	is an injective linear map into a finite-dimensional space,
	it follows that
	\[
	\dim_{\mathbb{C}} \Gamma(X,E) < \infty.
	\]
	
\end{proof}
	
	
	
	
	
	
\section{The Line Bundle Associated to a Divisor}

\begin{definition}[Divisors on \(X\)]
	Let \(X\) be a compact Riemann surface.  
	A \emph{divisor} on \(X\) is a formal finite sum
	\[
	D = \sum_{k=1}^{s} n_k x_k,
	\]
	where \(x_k \in X\) and \(n_k \in \mathbb{Z}\) for \(1 \leq k \leq s\).
	
	We say that \(D\) is an \emph{effective divisor} if and only if
	\[
	n_k \geq 0 \quad \text{for all } k = 1, \dots, s.
	\]
\end{definition}



\begin{definition}[Divisor of a meromorphic function]
	Let \(f \in \mathcal{M}(X)\) be a nonzero meromorphic function on \(X\).  
	Let \(x_1, \dots, x_s \in X\) be the points where \(f\) has either a zero or a pole,
	so that
	\[
	\operatorname{ord}_x(f) = 0
	\quad \text{for all } x \notin \{x_1, \dots, x_s\}.
	\]
	Write \(n_k = \operatorname{ord}_{x_k}(f)\) for \(1 \leq k \leq s\).
	The \emph{divisor of \(f\)} is defined by
	\[
	\operatorname{div}(f)
	= \sum_{k=1}^{s} n_k x_k.
	\]
\end{definition}




We now associate to each divisor \( D \) on a compact Riemann surface \( X \)
a holomorphic line bundle, denoted by \([D]\).





\medskip
Consider the complex vector space
\[
V_D :=
\bigl\{
f \in \mathcal{M}(X) : f \neq 0,\;
\operatorname{div}(f) \ge -D
\bigr\}
\;\sqcup\; \{0\}.
\]
It is clear that \(V_D\) is a complex vector space under pointwise
addition and scalar multiplication.

\smallskip
Equivalently, \(f \in V_D\) if and only if:
\(f\) is holomorphic on \(X \setminus \operatorname{supp}(D)\), where \(\operatorname{supp}(D) = \{x_1, \dots, x_s\}\); and
at each \(x_k\) (for \(1 \le k \le s\)) we have
\(\operatorname{ord}_{x_k}(f) \ge -n_k.\)

We write
\[
D_1 \ge D_2
\;\Longleftrightarrow\;
D_1 - D_2 \text{ is an effective divisor.}
\]

\begin{example}
	Let \(X = \mathbb{P}^1\) and take \(D = m \cdot \infty\), so that
	\(\operatorname{supp}(D) = \{\infty\}\),
	and the point at infinity corresponds to \(w=1/z=0\).
	Then
	\[
	V_D = V_{m \cdot \infty}
	= \{ P \in \mathbb{C}[z] : P=0 \text{ or } \deg P \le m \}.
	\]
	Hence,
	\[
	\dim_{\mathbb{C}} V_{m\cdot \infty} = m+1 < \infty.
	\]
	(This finiteness property holds for every divisor \(D\)
	on any compact Riemann surface \(X\).)
	
	
\end{example}

\medskip
Moreover, on each open set \(U_\alpha\),
there exists a holomorphic function \(f_\alpha\)
such that
\[
\operatorname{div}(f_\alpha) = D|_{U_\alpha}
= \sum\nolimits'_{x \in D \cap U_\alpha} n_x \cdot x,
\]
where \(n_x = 0\) whenever \(x \notin \{x_1, \dots, x_s\}.\)

We want to choose divisors \( D \) such that 
\( V_D \neq \{0\} \), and in fact so that \(V_D\) is
high‑dimensional.

Suppose \( h \in V_D \) with \(h \neq 0.\)
Then by definition,
\[
\operatorname{div}(h) \ge -D.
\]
Let
\[
D|_{U_\alpha} = \operatorname{div}(f_\alpha)
\]
be a local defining function of the divisor \(D\)
on each open set \(U_\alpha\) of a covering \(\{U_\alpha\}\) of \(X\).
Then on \(U_\alpha\),
\[
\operatorname{div}(h|_{U_\alpha}) 
\ge -\,\operatorname{div}(f_\alpha).
\]

Hence the product
\[
s_\alpha := h \cdot f_\alpha
\]
is holomorphic on \(U_\alpha\):
it can have at worst removable singularities at points
of \(\operatorname{supp}(D) \cap U_\alpha\), because for any such point \(x\),
\[
\operatorname{ord}_x(s_\alpha)
= \operatorname{ord}_x(h) + \operatorname{ord}_x(f_\alpha)
\ge -\,\operatorname{ord}_x(f_\alpha) + \operatorname{ord}_x(f_\alpha)
= 0.
\]
Thus,
\[
h|_{U_\alpha} = \frac{s_\alpha}{f_\alpha},
\qquad s_\alpha \in \mathcal{O}(U_\alpha),
\]
where \(\mathcal{O}(U_\alpha)\) denotes the holomorphic functions on \(U_\alpha.\)



On an overlap \(U_\alpha \cap U_\beta\), we have
\[
h = \frac{s_\alpha}{f_\alpha} = \frac{s_\beta}{f_\beta},
\]
and hence
\[
s_\alpha = \frac{f_\alpha}{f_\beta} s_\beta.
\]

Define
\[
\phi_{\alpha\beta} := \frac{f_\alpha}{f_\beta}.
\]
Then clearly
\[
\phi_{\alpha\alpha} = 1, \qquad
\phi_{\alpha\beta} \phi_{\beta\alpha} = 1, \qquad
\phi_{\alpha\beta} \phi_{\beta\gamma} \phi_{\gamma\alpha} = 1.
\]
Moreover,
\[
\operatorname{div}(\phi_{\alpha\beta})
= \operatorname{div}(f_\alpha) - \operatorname{div}(f_\beta)
= D|_{U_\alpha \cap U_\beta} - D|_{U_\alpha \cap U_\beta}
= 0,
\]
so each \(\phi_{\alpha\beta}\) is a nowhere‑vanishing holomorphic function on \(U_\alpha \cap U_\beta\).
Thus, the family \(\{\phi_{\alpha\beta}\}\) defines a compatible system of transition functions.




Over each open set $U_\alpha$, introduce a local trivialization
\[
\Phi_\alpha : \pi^{-1}(U_\alpha) \longrightarrow U_\alpha \times \mathbb{C},
\]
and define the local holomorphic frame (or ``unit section'')
\[
e_\alpha : U_\alpha \longrightarrow L,
\qquad 
e_\alpha(x) = \Phi_\alpha^{-1}(x,1).
\]
On overlaps $U_\alpha \cap U_\beta$, the change of trivialization is given by
\[
(x, v^{(\alpha)}) \sim (x, v^{(\beta)}) 
\quad \text{whenever} \quad
v^{(\alpha)} = \phi_{\alpha\beta}(x)\,v^{(\beta)}.
\]

A holomorphic section $t \in \Gamma(X, L)$ is represented locally by holomorphic
functions $t_\alpha \in \Gamma(U_\alpha, \mathcal{O})$ satisfying
\[
t_\alpha = \phi_{\alpha\beta}\,t_\beta
\quad \text{on } U_\alpha \cap U_\beta.
\]
That is, on each $U_\alpha$ we can write
\[
t|_{U_\alpha} = t_\alpha e_\alpha,
\]
and the compatibility condition above ensures that these local expressions glue
into a global section.

For any $h \in V_D$ (with $h \neq 0$), we can write locally
\[
h|_{U_\alpha} = \frac{s_\alpha}{f_\alpha},
\]
where each $s_\alpha$ is holomorphic and satisfies
\[
s_\alpha = \phi_{\alpha\beta}s_\beta
\quad \text{on } U_\alpha \cap U_\beta.
\]
If we define
\[
t_\alpha := s_\alpha,
\]
then the family $\{t_\alpha\}$ obeys the same transition rule,
so it defines a global holomorphic section $t = \{t_\alpha\} \in \Gamma(X, L)$.

Conversely, given a holomorphic section $t = \{t_\alpha\}$, we can reconstruct $h$
by setting
\[
h|_{U_\alpha} := \frac{t_\alpha}{f_\alpha}.
\]
This construction provides a natural one-to-one correspondence
\[
V_D \;\longleftrightarrow\; \Gamma(X, L).
\]


\begin{definition}[Divisor Line Bundle]
	Given a divisor \(D\) on \(X\), we denote by
	\[
	[D] := L
	\]
	the holomorphic line bundle constructed above.
	It is called the \emph{divisor line bundle associated to \(D\)}.
\end{definition}

\begin{theorem}
	For any divisor \(D\) on a compact Riemann surface \(X\),
	there is a natural vector‑space isomorphism
	\[
	V_D \;\cong\; \Gamma(X, [D]).
	\]
\end{theorem}

\begin{remark}
	\begin{enumerate}
		\item Although we assumed \(D \ge 0\) (i.e.\ that \(D\) is effective) in the discussion above, 
		the construction of the line bundle \([D]\) from local data 
		\(f_\alpha\) satisfying \(\mathrm{div}(f_\alpha) = D|_{U_\alpha}\) 
		works for any divisor \(D\) on \(X\).
		
		\item If \(D \ge 0\), then under the natural isomorphism
		\[
		V_D \;\cong\; \Gamma(X, [D]),
		\]
		the constant function \(1 \in V_D\) corresponds to a 
		distinguished global holomorphic section
		\[
		s_D \in \Gamma(X, [D]),
		\]
		called the \emph{canonical section} of the divisor line bundle \([D]\).
		
		Indeed, for \(h = 1\), the correspondence
		\(
		h \;\longleftrightarrow\; \{ t_\alpha = h f_\alpha \}
		\)
		gives
		\[
		t_\alpha = f_\alpha.
		\]
		These local functions satisfy the transition rule
		\(
		f_\alpha = \frac{f_\alpha}{f_\beta} f_\beta = \phi_{\alpha\beta} f_\beta
		\)
		on \(U_\alpha \cap U_\beta\), 
		so they \emph{glue} to a well-defined global section
		\[
		s_D = \{ s_{D,\alpha} = f_\alpha \}.
		\]

		
		The section \(s_D\) vanishes precisely along the divisor \(D\);
		in particular, \(\operatorname{div}(s_D) = D\).
	\end{enumerate}
\end{remark}


\noindent
\underline{Observation}:
\[
1 \in V_D
\quad \Longleftrightarrow \quad
D \ge 0.
\]
In this case, \(1 \in V_D\) corresponds to the canonical section
\(s_D \in \Gamma(X, [D])\) defined locally by \(s_D|_{U_\alpha} = f_\alpha.\)



\section{Classification of Holomorphic Line Bundles up to Isomorphism}



\begin{definition}[Isomorphism of holomorphic line bundles]
	Let
	\[
	\pi_1 : E \longrightarrow X,
	\qquad
	\pi_2 : E' \longrightarrow X
	\]
	be holomorphic line bundles over the same complex manifold \(X\).
	
	A \emph{holomorphic line bundle isomorphism}
	from \(E\) to \(E'\) is a biholomorphic map of total spaces
	\[
	\Phi : E \longrightarrow E'
	\]
	such that the following two conditions hold:
	
	\begin{enumerate}
		\item[(i)] The projection maps are preserved; i.e.\ the diagram
		\[
		\xymatrix{
			E \ar[r]^{\Phi} \ar[d]_{\pi_1} & E' \ar[d]^{\pi_2} \\
			X \ar[r]_{\mathrm{id}_X} & X
		}
		\]
		commutes.  In other words,
		\(\pi_2 \circ \Phi = \pi_1.\)
		
		\item[(ii)] For each \(x \in X\),
		the induced map on the fiber
		\[
		\Phi_x := \Phi|_{E_x} : E_x \longrightarrow E_x'
		\]
		is a \emph{complex‑linear isomorphism}
		between the one‑dimensional complex vector spaces \(E_x\) and \(E_x'\).
	\end{enumerate}
	
	If such a map \(\Phi\) exists, we say that \(E\) and \(E'\) are
	\emph{isomorphic holomorphic line bundles}, and we write
	\[
	E \cong E' .
	\]
\end{definition}


	Let
	\[
	\pi_i : E_i \longrightarrow X, \qquad i = 1,2
	\]
	be holomorphic line bundles over a complex manifold \(X\).
	Let \(\{U_\alpha\}_{\alpha \in A}\) be an open cover of \(X\)
	such that each restricted bundle is holomorphically trivial:
	\[
	E_i|_{U_\alpha} \cong U_\alpha \times \mathbb{C}, \qquad i=1,2.
	\]
	On overlaps \(U_{\alpha\beta} = U_\alpha \cap U_\beta\),
	the corresponding transition functions are
	\[
	\phi_{\alpha\beta}^i : U_{\alpha\beta} \longrightarrow \mathbb{C}^*, \qquad i=1,2,
	\]
	which satisfy the cocycle (compatibility) conditions:
	\[
	\phi_{\alpha\beta}^i \,\phi_{\beta\gamma}^i \,\phi_{\gamma\alpha}^i = 1
	\quad\text{on } U_{\alpha\beta\gamma}.
	\]

	
	\emph{(1) Tensor product.}
	We define the tensor product line bundle \(E_1 \otimes E_2\)
	by taking the same open cover \(\{U_\alpha\}\)
	and using new transition functions
	\[
	\psi_{\alpha\beta} := \phi_{\alpha\beta}^1 \, \phi_{\alpha\beta}^2
	\quad\text{on } U_{\alpha\beta}.
	\]
	Since the product of two cocycles is again a cocycle,
	\(\{\psi_{\alpha\beta}\}\) satisfies the compatibility conditions:
	\[
	\psi_{\alpha\beta}\psi_{\beta\gamma}\psi_{\gamma\alpha}
	= (\phi_{\alpha\beta}^1\phi_{\alpha\beta}^2)
	(\phi_{\beta\gamma}^1\phi_{\beta\gamma}^2)
	(\phi_{\gamma\alpha}^1\phi_{\gamma\alpha}^2)
	= 1.
	\]
	Hence \(\{\psi_{\alpha\beta}\}\) defines a holomorphic line bundle,
	denoted \(E_1 \otimes E_2\).
	
If \(E_1\) and \(E_2\) are described on possibly different open covers
\(\{U_a^{\,1}\}_{a\in A_1}\) and \(\{U_A^{\,2}\}_{A\in A_2}\),
we can pass to a common refinement as follows.

For indices \(\alpha = (a,A)\) and \(\beta = (b,B)\),
define
\[
U_{\alpha\beta} := U_a^{\,1} \cap U_b^{\,1} \cap U_A^{\,2} \cap U_B^{\,2}.
\]
The collection
\[
\{\, U_a^{\,1} \cap U_A^{\,2} \;|\; a \in A_1,\, A \in A_2 \,\}
\]
is an open cover of \(X\),
because
\[
\bigcup_{a,A} \, (U_a^{\,1} \cap U_A^{\,2})
= \left(\bigcup_a U_a^{\,1}\right) \cap \left(\bigcup_A U_A^{\,2}\right)
= X \cap X = X.
\]
Hence this refinement is still an open cover of \(X\),
and we can define the transition functions for the tensor product bundle
on these intersections by
\[
\psi_{\alpha\beta} := \phi_{ab}^{\,1}\, \phi_{AB}^{\,2}
\quad \text{on } U_{\alpha\beta}.
\]
	

	\emph{(2) Dual bundle.}
	Given a holomorphic line bundle \(\pi:E\to X\)
	with transition functions \(\{\phi_{\alpha\beta}\}\),
	define the \emph{dual line bundle} \(E^{-1}\) (or \(E^*\))
	by taking the same open cover and setting
	\[
	\psi_{\alpha\beta} := \frac{1}{\phi_{\alpha\beta}}.
	\]
	Since \(\phi_{\alpha\beta}\phi_{\beta\gamma}\phi_{\gamma\alpha}=1,\)
	it follows that \(\psi_{\alpha\beta}\psi_{\beta\gamma}\psi_{\gamma\alpha}=1.\)
	Hence \(\{\psi_{\alpha\beta}\}\) defines a holomorphic line bundle \(E^{-1}\).
	
	The tensor and inverse constructions yield operations:
	\[
	(E_1, E_2) \longmapsto E_1 \otimes E_2, 
	\qquad E \longmapsto E^{-1},
	\]
	which behave analogously to multiplication and inverse in an abelian group.
	

	
	\emph{(3) Canonical isomorphisms.}
	\[
	E_1 \otimes E_2 \cong E_2 \otimes E_1, 
	\qquad E \otimes \mathcal{O}_X \cong E,
	\]
	where the \emph{trivial line bundle} \(\mathcal{O}_X\)
	(or \(\mathds{1}\)) is defined by the constant transition functions
	\(\phi_{\alpha\beta} \equiv 1.\)
	

	
\begin{theorem}
	The set of isomorphism classes of holomorphic line bundles over \(X\),
	denoted by
	\[
	\operatorname{Pic}(X),
	\]
	forms an abelian group under the operation
	\[
	[E_1], [E_2] \longmapsto [E_1 \otimes E_2],
	\]
	called the \emph{Picard group} of \(X\).
	The identity element is the class of the trivial bundle \(\mathcal{O}_X\),
	and the inverse of \([E]\) is \([E^{-1}].\)
\end{theorem}


\noindent
\underline{\text{Preparation:}}  
What does it mean to say \(E \cong E'\)
as holomorphic line bundles over \(X\),
in terms of their transition functions?



Let
\[
\Psi : E \overset{\cong}{\longrightarrow} E'
\]
be an isomorphism of holomorphic line bundles over \(X\).
Choose an open covering 
\(\mathcal{U} = \{U_\alpha\}\)
that works for both \(E\) and \(E'\), i.e.
\[
E|_{U_\alpha} \cong U_\alpha \times \mathbb{C},
\qquad
E'|_{U_\alpha} \cong U_\alpha \times \mathbb{C}.
\]
On each \(U_\alpha\),
\(\Psi\) restricts to a biholomorphic map making the diagram commute:
\[
\xymatrix{
	E|_{U_\alpha} \ar[r]^{\Psi|_{U_\alpha}} \ar[d]_{\cong} &
	E'|_{U_\alpha} \ar[d]^{\cong} \\
	U_\alpha \times \mathbb{C} \ar[r]_{(\mathrm{id},\,\psi_\alpha)} &
	U_\alpha \times \mathbb{C},
}
\]
where
\(\psi_\alpha : U_\alpha \to \mathbb{C}^*\)
is a nowhere‑vanishing holomorphic function
such that locally
\[
\Psi(x, v^{(\alpha)}) = \bigl(x,\, \psi_\alpha(x)\, v^{(\alpha)}\bigr).
\]



On an overlap \(U_{\alpha\beta} = U_\alpha \cap U_\beta\),
we have similar expressions:
\[
\Psi(x, v^{(\alpha)}) = (x,\, \psi_\alpha(x)\, v^{(\alpha)}),
\qquad
\Psi(x, v^{(\beta)}) = (x,\, \psi_\beta(x)\, v^{(\beta)}).
\]

On \(E\), the local trivialisations relate via transition functions
\[
v^{(\alpha)} = \phi_{\alpha\beta}(x)\, v^{(\beta)},
\]
and on \(E'\),
\[
\Psi(x, v^{(\alpha)}) = \phi_{\alpha\beta}'(x)\, \Psi(x, v^{(\beta)}).
\]



Compatibility of these descriptions under \(\Psi\) requires:
\[
\psi_\alpha(x)\, v^{(\alpha)}
= \phi_{\alpha\beta}'(x)\, \psi_\beta(x)\, v^{(\beta)}.
\]

Comparing coefficients of \(v^{(\beta)}\), we obtain
\[
\psi_\alpha(x)\, \phi_{\alpha\beta}(x)
= \phi_{\alpha\beta}'(x)\, \psi_\beta(x),
\]
or equivalently,
\[
\boxed{\;
	\psi_\alpha^{-1}\, \phi_{\alpha\beta}'\, \psi_\beta
	= \phi_{\alpha\beta},
	\quad\text{i.e.}\quad
	\frac{\phi_{\alpha\beta}'}{\phi_{\alpha\beta}}
	= \frac{\psi_\alpha}{\psi_\beta}.
	\;}
\]

\begin{proof}
	We must verify that
	\[
	E_1 \cong \tilde{E}_1, \quad
	E_2 \cong \tilde{E}_2
	\quad \Longrightarrow \quad
	E_1 \otimes E_2 \cong \tilde{E}_1 \otimes \tilde{E}_2.
	\]
	
	Let \(E_1\) and \(E_2\) be holomorphic line bundles on \(X\)
	defined by their transition functions
	\(\{\phi_{\alpha\beta}^1\}\) and \(\{\phi_{\alpha\beta}^2\}\), respectively,  
	and let
	\(\{\tilde{\phi}_{\alpha\beta}^1\}\) and
	\(\{\tilde{\phi}_{\alpha\beta}^2\}\)
	define \(\tilde{E}_1\) and \(\tilde{E}_2\).
	
	\medskip
	\noindent
	The assumption \(E_1 \cong \tilde{E}_1\) means that there exist nowhere‑vanishing holomorphic functions
	\(\{\psi_\alpha^1\}\) such that
	\[
	\frac{\tilde{\phi}_{\alpha\beta}^1}{\phi_{\alpha\beta}^1}
	= \frac{\psi_\alpha^1}{\psi_\beta^1}.
	\]
	Similarly, from \(E_2 \cong \tilde{E}_2\), there exist
	\(\{\psi_\alpha^2\}\) such that
	\[
	\frac{\tilde{\phi}_{\alpha\beta}^2}{\phi_{\alpha\beta}^2}
	= \frac{\psi_\alpha^2}{\psi_\beta^2}.
	\]
	
	\medskip
	\noindent
	The tensor product bundles \(E_1 \otimes E_2\) and
	\(\tilde{E}_1 \otimes \tilde{E}_2\)
	are defined by the products of their transition functions:
	\[
	\phi_{\alpha\beta}^1 \phi_{\alpha\beta}^2
	\quad\text{and}\quad
	\tilde{\phi}_{\alpha\beta}^1 \tilde{\phi}_{\alpha\beta}^2,
	\]
	respectively.
	
	Then we compute:
	\[
	\frac{\tilde{\phi}_{\alpha\beta}^1 \tilde{\phi}_{\alpha\beta}^2}
	{\phi_{\alpha\beta}^1 \phi_{\alpha\beta}^2}
	= \frac{\psi_\alpha^1 \psi_\alpha^2}{\psi_\beta^1 \psi_\beta^2}
	:= \frac{\lambda_\alpha}{\lambda_\beta},
	\quad\text{where}\quad
	\lambda_\alpha := \psi_\alpha^1 \psi_\alpha^2.
	\]
	Hence the bundles \(E_1 \otimes E_2\) and \(\tilde{E}_1 \otimes \tilde{E}_2\)
	are isomorphic.
	
	\medskip
	\noindent
	Similarly, suppose \(E \cong E'\).
	Then there exist functions \(\{\psi_\alpha\}\) such that, if
	\(\{\phi_{\alpha\beta}\}\) and
	\(\{\phi_{\alpha\beta}'\}\)
	define \(E\) and \(E'\), we have
	\[
	\frac{\phi_{\alpha\beta}'}{\phi_{\alpha\beta}}
	= \frac{\psi_\alpha}{\psi_\beta}.
	\]
	The inverse bundle \(E^{-1}\) is defined by the transition functions
	\(\phi_{\alpha\beta}^{-1} = \phi_{\beta\alpha}\).
	Thus,
	\[
	\frac{(\phi_{\alpha\beta}')^{-1}}{(\phi_{\alpha\beta})^{-1}}
	= \frac{\psi_\beta}{\psi_\alpha}
	= \frac{1/\psi_\alpha}{1/\psi_\beta},
	\]
	which shows that
	\[
	E^{-1} \cong (E')^{-1}.
	\]
\end{proof}

\begin{remark}
	The Picard group \(\mathrm{Pic}(X)\) provides an example of a
	\emph{classification (or moduli) problem}:
	it classifies holomorphic line bundles on \(X\)
	up to isomorphism.
\end{remark}



\underline{\text{Question:}} 
Can one determine \(\mathrm{Pic}(X)\) for a given complex manifold \(X\)?

\begin{theorem}\label{thm:PicP1}
	\[
	\mathrm{Pic}(\mathbb{P}^1) \cong \mathbb{Z}.
	\]
	In fact, every holomorphic line bundle 
	\(\pi : E \to \mathbb{P}^1\)
	is isomorphic to a divisor line bundle
	\([m \cdot \infty]\)
	for some integer \(m \in \mathbb{Z}\).
\end{theorem}

\underline{\text{Preparation:}}  


\textbf{Fact:}  
Any holomorphic line bundle over a disk \(D(a; r)\)
is holomorphically trivial, i.e.
\[
\pi : E \longrightarrow D(a; r)
\quad\text{is isomorphic to}\quad
\pi_0 : D(a; r) \times \mathbb{C} \longrightarrow D(a; r).
\]

That is, every holomorphic line bundle over a simply‑connected domain in
\(\mathbb{C}\) is trivial. 

\begin{lemma}
	Let \(f\) be a nowhere–vanishing holomorphic function on \(\mathbb{C}^*\).
	Then there exists an integer \(k \in \mathbb{Z}\) and a holomorphic function
	\(h\) on \(\mathbb{C}^*\) such that
	\[
	f(z) = z^k e^{h(z)}.
	\]
\end{lemma}

\begin{proof}
	Since \(f\) is nowhere zero, the logarithmic derivative
	\(\dfrac{f'(z)}{f(z)}\) is holomorphic on \(\mathbb{C}^*\).
	Consider a positively oriented circle
	\(\gamma(t) = r e^{it}\), \(t \in [0, 2\pi]\), for some fixed \(r > 0\).
	
	Define
	\[
	\frac{1}{2\pi i} \int_\gamma \frac{f'(z)}{f(z)}\,dz
	= \frac{1}{2\pi i} \int_\gamma d(\log f).
	\]
	By the argument principle, this integral equals the winding number
	of \(f(\gamma)\) around the origin, which is an integer:
	\[
	\frac{1}{2\pi i}\int_\gamma \frac{f'(z)}{f(z)}\,dz = k \in \mathbb{Z}.
	\]
	
	Now set
	\[
	g(z) := \frac{f(z)}{z^k}.
	\]
	Then
	\[
	\frac{g'(z)}{g(z)} = \frac{f'(z)}{f(z)} - \frac{k}{z}.
	\]
	The integral of this differential around any closed loop \(\gamma\) in \(\mathbb{C}^*\) vanishes,
	because
	\[
	\oint_\gamma \frac{g'(z)}{g(z)}\,dz
	= \oint_\gamma \frac{f'(z)}{f(z)}\,dz - k \oint_\gamma \frac{dz}{z}
	= 2\pi i k - 2\pi i k = 0.
	\]
	Hence the 1–form \(\dfrac{g'(z)}{g(z)}dz\) is exact on \(\mathbb{C}^*\),
	which means we can define a holomorphic function
	\[
	h(z) := \int_{\gamma_z} \frac{g'(w)}{g(w)}\,dw,
	\]
	where the integral is independent of the path \(\gamma_z\) from a fixed basepoint to \(z\).
	
	Then \(g = e^{h}\), so
	\[
	f(z) = z^k e^{h(z)}.
	\]
\end{proof}


\begin{proof}[Proof of Theorem~\ref{thm:PicP1}]
	Let
	\[
	\mathcal{U} = \{U_0, U_1\}, \quad
	U_0 = \{[t_0:t_1] \in \PP^1 \mid t_0 \neq 0\}, \quad
	U_1 = \{[t_0:t_1] \in \PP^1 \mid t_1 \neq 0\}.
	\]
	In homogeneous coordinates we set
	\[
	z = \frac{t_1}{t_0} \quad \text{on } U_0 \cong \C, 
	\qquad
	w = \frac{t_0}{t_1} = \frac{1}{z} \quad \text{on } U_1 \cong \C.
	\]
	Then their intersection
	\(U_{01} = U_0 \cap U_1 \cong \C^*\),
	with coordinates related by \(z = 1/w\).
	
	\medskip
	
	By the preceding fact, every holomorphic line bundle over a disk is holomorphically trivial.
	Hence, over the open sets \(U_0\) and \(U_1\), the restrictions
	\(E|_{U_0}\) and \(E|_{U_1}\) are both trivial:
	\[
	E|_{U_i} \cong U_i \times \C \qquad (i = 0,1).
	\]
	Therefore, \(E\) is determined by a single transition function
	\(\phi_{01} \in \Gamma(U_{01}, \mathcal{O}^*) = \Gamma(\C^*, \mathcal{O}^*)\)
	relating the local trivialisations:
	\[
	(z, v^{(0)}) \leftrightarrow (w, v^{(1)}),
	\qquad 
	v^{(0)} = \phi_{01}(z) \, v^{(1)}.
	\]
	
	\medskip
	
	By the lemma above, there exist \(k \in \Z\) and 
	\(h \in \Gamma(\C^*, \mathcal{O})\) such that
	\[
	\phi_{01}(z) = z^k e^{h(z)}.
	\]
	Let \(A\) denote the line bundle defined by transition function \(z^k\),
	and \(B\) the bundle defined by \(e^{h}\).  Then \(E \cong A \otimes B.\)
	

	Define local frames:
	\[
	e_0 \equiv 1 \text{ on } U_0,
	\qquad
	e_1 \equiv w^k \text{ on } U_1.
	\]
	Since on \(U_{01}\) we have \(z = 1/w\), these satisfy
	\[
	e_0 = z^k e_1.
	\]
	Hence \(A\) is exactly the standard line bundle
	\(\mathcal{O}_{\PP^1}(k) = [k \cdot \infty].\)
	

	We claim there exist holomorphic functions
	\(s_0 \in \Gamma(U_0, \mathcal{O})\),
	\(s_1 \in \Gamma(U_1, \mathcal{O})\)
	such that
	\[
	h = s_0 - s_1 \quad \text{on } U_{01} \cong \C^*.
	\]
	If so, then
	\(e^{h} = e^{s_0} / e^{s_1}\),
	and by redefining the local frames
	\(\tilde{e}_i := e^{s_i} e_i\),
	the new transition function becomes
	\(\tilde{\phi}_{01} = 1\),
	showing \(B\) is holomorphically trivial.
	
	\medskip
	
	To prove the claim, expand \(h\) as a Laurent series on \(\C^*\):
	\[
	h(z) = \sum_{n=-\infty}^{\infty} c_n z^n.
	\]
	Split this series into nonnegative and negative parts:
	\[
	h(z)
	= \underbrace{\sum_{n \ge 0} c_n z^n}_{=: s_0(z)\ \text{holomorphic on } U_0 \cong \C}
	- \underbrace{\sum_{m \ge 1} (-c_{-m}) z^{-m}}_{=: s_1(w)\ \text{holomorphic on } U_1 \cong \C_w}.
	\]
	Rewriting the negative part in terms of \(w = 1/z\), we have
	\[
	s_1(w) = -\sum_{m \ge 1} c_{-m} w^m,
	\]
	which is holomorphic on \(U_1 \cong \C\).
	Thus \(h = s_0 - s_1\) on \(U_{01}\), proving the claim.
	Therefore, \(B\) is trivial.
	
	\medskip
	
	Combining these, we obtain
	\[
	E \cong A \otimes B \cong [k\cdot \infty].
	\]
	Consequently, every holomorphic line bundle on \(\PP^1\)
	is of the form $[k\cdot \infty]$,
	and distinct integers \(k\) give non‑isomorphic bundles.
	Hence
	\[
	\mathrm{Pic}(\PP^1) \cong \Z.
	\]
\end{proof}


\end{document}
