\documentclass[12pt]{article}

% Packages for math and formatting
\usepackage[utf8]{inputenc}
\usepackage{amsmath, amssymb, amsthm}   % math symbols & environments
\usepackage{geometry}                   % page layout
\usepackage{enumitem}                   % better lists
\usepackage{hyperref}                   % clickable references
\usepackage{mathtools}                  % extended math tools
\usepackage{epigraph}                   % for quotes
\usepackage{tikz}
\usetikzlibrary{arrows.meta}

% Page setup
\geometry{margin=1in}

% Theorem-like environments
\newtheorem{theorem}{Theorem}[section]
\newtheorem{proposition}[theorem]{Proposition}
\newtheorem{lemma}[theorem]{Lemma}
\newtheorem{corollary}[theorem]{Corollary}
\theoremstyle{definition}
\newtheorem{definition}[theorem]{Definition}
\newtheorem{problem}[theorem]{Problem}
\theoremstyle{remark}
\newtheorem{remark}[theorem]{Remark}
\newtheorem{example}[theorem]{Example}

% Custom commands
\newcommand{\R}{\mathbb{R}}
\newcommand{\N}{\mathbb{N}}
\newcommand{\Q}{\mathbb{Q}}
\newcommand{\Z}{\mathbb{Z}}
\newcommand{\C}{\mathbb{C}}
\newcommand{\PP}{\mathbb{P}}
\newcommand{\D}{\mathbb{D}}
% Linear groups
\newcommand{\GL}{\operatorname{GL}}

\newcommand{\HH}{\mathbb{H}} % upper half-plane

\DeclareMathOperator{\Aut}{Aut}
\DeclareMathOperator{\FractLin}{FractLin}
\DeclareMathOperator{\ord}{ord}
\DeclareMathOperator{\Res}{Res}
\DeclareMathOperator{\SL}{SL}

\title{Introduction}
\author{Joe}
\date{September 10, 2025}

\begin{document}
	
	\maketitle
	
	\tableofcontents
	\vspace{1em}
	
	\section{Foundations of Complex Analysis}
	
	Let $\Omega$ be a domain that is open and connected, and consider $f: \Omega \to \C$ as a complex differentiable function.
	
	There are three primary approaches to studying complex analytic (holomorphic) functions:
	\begin{enumerate}
		\item \textbf{Partial Differential Equations} (Cauchy-Riemann equations)
		\item \textbf{Power Series}
		\item \textbf{Integral Formulas}
	\end{enumerate}
	
	\subsection{The Cauchy-Riemann Approach}
	
	For a function $f: \Omega \to \C$, we define the Wirtinger derivatives:
	\begin{align}
		\frac{\partial f}{\partial z} &= \frac{1}{2}\left(\frac{\partial f}{\partial x} - i\frac{\partial f}{\partial y}\right)\\
		\frac{\partial f}{\partial \bar{z}} &= \frac{1}{2}\left(\frac{\partial f}{\partial x} + i\frac{\partial f}{\partial y}\right)
	\end{align}
	
	The Cauchy-Riemann equations are equivalent to the condition:
	$$\frac{\partial f}{\partial \bar{z}} = 0 \text{ everywhere on } \Omega$$
	
	\subsection{Power Series Approach}
	
	For $a \in \C$, consider a convergent power series $\sum_{n \geq 0} c_n(z-a)^n$ with radius of convergence $R > 0$. We define:
	$$f(z) = \sum_{n=0}^{\infty} c_n(z-a)^n \text{ for } |z-a| < R$$
	that is, for $z \in D(a; R)$ (the open disk centered at $a$ with radius $R$).
	
	\begin{theorem}
		If $f$ is holomorphic in $D(a; R)$, then:
		$$f'(z) = \sum_{n \geq 1} nc_n(z-a)^{n-1}$$
	\end{theorem}
	
	\subsection{Cauchy Integral Formula}
	
	Suppose $f: D(a; R) \to \C$ is holomorphic. For any $0 < r < R$ and $z \in D(a; r)$, we have:
	$$f(z) = \frac{1}{2\pi i} \oint_{\partial D(a;r)} \frac{f(\xi)}{z-\xi} d\xi$$
	
	\begin{theorem}[Taylor's Theorem for Holomorphic Functions]
		Let $f : D(a;R) \to \C$ be holomorphic on the open disk $D(a;R) = \{ z \in \C : |z-a| < R \}$, where $R>0$.  
		Then $f$ has a convergent power series expansion around $a$:
		\[
		f(z) = \sum_{n=0}^{\infty} c_n (z-a)^n, \qquad |z-a| < R,
		\]
		where
		\[
		c_n = \frac{f^{(n)}(a)}{n!}.
		\]
		Moreover, the radius of convergence of this series is at least $R$.
	\end{theorem}
	
	
	\section{Meromorpphic Functions}
	
	
	
	
	
	
	
	
	
	
	
\end{document}