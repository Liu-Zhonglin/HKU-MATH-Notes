\documentclass[12pt]{article}

% Packages for math and formatting
\usepackage[utf8]{inputenc}
\usepackage{amsmath, amssymb, amsthm}   % math symbols & environments
\usepackage{geometry}                   % page layout
\usepackage{enumitem}                   % better lists
\usepackage{hyperref}                   % clickable references
\usepackage{mathtools}                  % extended math tools
\usepackage{epigraph}                   % for quotes

% Page setup
\geometry{margin=1in}

% Theorem-like environments
\newtheorem{theorem}{Theorem}[section]
\newtheorem{proposition}[theorem]{Proposition}
\newtheorem{lemma}[theorem]{Lemma}
\newtheorem{corollary}[theorem]{Corollary}
\theoremstyle{definition}
\newtheorem{definition}[theorem]{Definition}
\newtheorem{problem}[theorem]{Problem}
\theoremstyle{remark}
\newtheorem{remark}[theorem]{Remark}

% Custom commands
\newcommand{\R}{\mathbb{R}}
\newcommand{\N}{\mathbb{N}}
\newcommand{\Q}{\mathbb{Q}}
\newcommand{\Z}{\mathbb{Z}}
\newcommand{\C}{\mathbb{C}}
\newcommand{\PP}{\mathbb{P}}

\DeclareMathOperator{\Aut}{Aut}
\DeclareMathOperator{\FractLin}{FractLin}
\DeclareMathOperator{\ord}{ord}
\usepackage{amsmath}
\DeclareMathOperator{\Res}{Res}




\title{Assignment 1}
\author{Liu Zhonglin\\
3035974614}
\date{}


\begin{document}
	
	
	
	\maketitle
	
	
	\section{Q1}
\subsection*{(a)}


\subparagraph{Case 1: $\infty \notin \{x_1, \ldots, x_p\}$.}

Define
\[
f(z) = \prod_{k=1}^{p} (z - x_k)^{n_k}.
\]
Each factor $(z - x_k)^{n_k}$ contributes a zero of order $n_k$ if $n_k > 0$
and a pole of order $|n_k|$ if $n_k < 0$ at $x_k$.
Thus $\operatorname{ord}_{x_k}(f) = n_k$ for every finite $x_k$.

It remains to check the behavior at $\infty$.
We can write
\[
f(z)
= \prod_{k=1}^{p} \bigg[z^{n_k}\!\bigg(1-\frac{x_k}{z}\bigg)^{n_k}\bigg]
= z^{n_1+\cdots+n_p} \prod_{k=1}^{p}\!\bigg(1-\frac{x_k}{z}\bigg)^{n_k}.
\]
Since $\sum_{k=1}^{p} n_k = 0$, the factor $z^{n_1+\cdots+n_p}$ is $1$,
and the remaining product tends to $1$ as $z \to \infty$.
Hence $f$ is holomorphic and nonvanishing at $\infty$, i.e.
\[
\operatorname{ord}_{\infty}(f) = 0.
\]
Therefore $f$ satisfies $\operatorname{ord}_{x_k}(f) = n_k$ for all $1 \le k \le p$,
and $\operatorname{ord}_x(f)=0$ for other points of $\mathbb{P}^1$.

\subparagraph{Case 2: $\infty \in \{x_1, \ldots, x_p\}$.}

Without loss of generality, suppose $x_p = \infty$.
Then
\[
n_p = - \sum_{k=1}^{p-1} n_k.
\]
Define the meromorphic function
\[
f(z) = \prod_{k=1}^{p-1} (z - x_k)^{n_k}.
\]
Each $x_k$ for $k = 1, \ldots, p-1$ is a zero or a pole
of order $n_k$.
To find the order at infinity,
write $w = 1/z$ and express $f$ as
\[
f(1/w)
= \prod_{k=1}^{p-1} \bigg(\frac{1 - x_k w}{w}\bigg)^{n_k}
= w^{-\sum_{k=1}^{p-1} n_k} \prod_{k=1}^{p-1}(1 - x_k w)^{n_k}.
\]
Since $\sum_{k=1}^{p-1} n_k = -n_p$, we obtain
\[
f(1/w) = w^{n_p} \, h(w),
\qquad
h(w) = \prod_{k=1}^{p-1}(1 - x_k w)^{n_k}.
\]
Because $h(0)=1$, $h$ is holomorphic and nonzero at $w=0$.
Hence
\[
\operatorname{ord}_{\infty}(f)
= \operatorname{ord}_{w=0}(f(1/w))
= \operatorname{ord}_{w=0}(w^{n_p}h(w)) = n_p.
\]
Therefore $f$ has order $n_p$ at $\infty$ and order $n_k$ at each finite $x_k$.



In both cases, the function $f$ is meromorphic on $\mathbb{P}^1$ and satisfies
\[
\operatorname{ord}_{x_k}(f) = n_k, \quad k = 1, \ldots, p,
\quad \text{and} \quad
\operatorname{ord}_x(f)=0 \text{ for all other points.}
\]



\subsection*{(b)}




The Weierstrass $\wp$-function associated to the lattice $L$ is defined by
\[
\wp(z) = \frac{1}{z^2} + \sum_{\omega \in L \setminus \{0\}}
\left(\frac{1}{(z-\omega)^2} - \frac{1}{\omega^2}\right).
\]


Observe that $\wp(z)$ is even, so the function
\[
\wp(z) - \wp(x_k)
\]
vanishes when $\wp(z) = \wp(x_k)$, i.e.\ precisely at $z = \pm x_k$ modulo $L$.  
Near $z = x_k$, $\wp(z) - \wp(x_k)$ has a simple zero (since $\wp'(x_k) \neq 0$ if $x_k \neq 0$ or half-periods);
and since $\wp(z)$ is even, it also has a simple zero at $-x_k$.  
Therefore each factor $(\,\wp(z) - \wp(x_k)\,)$ introduces one zero at $x_k$ and one at $-x_k$.

Thus the function
\[
(\wp(z) - \wp(x_k))^{n_k}
\]
will have zeros (if $n_k > 0$) or poles (if $n_k < 0$) of order $n_k$ simultaneously at both
$x_k$ and $-x_k$.


Now define
\[
f(z) := \prod_{k=1}^{p} (\wp(z) - \wp(x_k))^{n_k}.
\]
Each factor is an elliptic (meromorphic, $L$-periodic) function on $X$, and since the product
of elliptic functions is again elliptic, $f(z)$ is a well-defined meromorphic function on $X$.



At any $z = x_j$,
the only term contributing a zero or pole is the factor $(\wp(z) - \wp(x_j))^{n_j}$:
\[
\operatorname{ord}_{x_j}(f) = n_j.
\]
Similarly, by evenness \(\wp(-z) = \wp(z)\),
\[
\operatorname{ord}_{-x_j}(f) = n_j.
\]
At points $z$ not congruent to $\pm x_k$ modulo $L$, each $\wp(z) - \wp(x_k)$ is holomorphic
and nonzero, so these are precisely all the zero and pole points of $f$.



Since $\sum_{k=1}^{p} n_k = 0$,
the total number of zeros (counted with multiplicity) equals the total number of poles, which
and $f$ extends to a meromorphic function on
$X = \mathbb{C}/L$.


\section{Q2}

\subsection*{(a)}





The lattice $L$ is a discrete subset of $\mathbb{C} \simeq \mathbb{R}^2$.  
Let
\[
S_n = \{\, n_1 \omega_1 + n_2 \omega_2 : -n \le n_1, n_2 \le n \,\}, \qquad
T_n = S_n \setminus S_{n-1}, \ n \ge 1.
\]
Then $\lvert S_n \rvert = (2n+1)^2$ and $\lvert T_n \rvert = 8n$.  
The ring $T_n$ consists of all lattice points lying on the boundary of the $2n \!\times\! 2n$ square centered at the origin in the $(\omega_1,\omega_2)$–coordinate grid.



There exist positive constants $a,b>0$ (depending only on $\omega_1,\omega_2$) such that
\[
a n \le |\omega| \le b n \qquad \text{for every } \omega \in T_n.
\]
If $\Omega_1$ denotes the fundamental parallelogram
$\{ s\omega_1 + t\omega_2 : 0 \le s,t \le 1 \}$,
then each $T_n$ lies in the dilate $n \,\overline{\Omega_1}$,
so setting 
\[
a = \min_{\zeta \in \overline{\Omega_1}} |\zeta|, 
\quad b = \max_{\zeta \in \overline{\Omega_1}} |\zeta|
\]
gives the desired inequality.



Hence for all $k > 0$,
\[
\frac{1}{b^k} \sum_{n=1}^{\infty} \frac{|T_n|}{n^k}
\le \sum_{\omega \in L'} \frac{1}{|\omega|^{k}}
\le \frac{1}{a^k} \sum_{n=1}^{\infty} \frac{|T_n|}{n^k}.
\]
Since $|T_n| = 8n$, this becomes
\[
\frac{8}{b^k} \sum_{n=1}^{\infty} \frac{1}{n^{\,k-1}}
\le
\sum_{\omega \in L'} \frac{1}{|\omega|^{k}}
\le
\frac{8}{a^k} \sum_{n=1}^{\infty} \frac{1}{n^{\,k-1}}.
\]



The comparison series 
$\sum_{n=1}^{\infty} \frac{1}{n^{k-1}}$
converges if and only if $k-1>1$, i.e.\ $k>2$.  
Therefore, for all $k \ge 3$,
the lattice sum is absolutely convergent:
\[
\boxed{
	\displaystyle
	\sum_{\omega \in L'} \frac{1}{|\omega|^{k}} < \infty.
}
\]

\subsection*{(b)}








Define
\[
h(z) := \frac{z}{z^6 - 1}.
\]
The function $h$ is meromorphic on~$\mathbb{C}$ and has simple poles at the six sixth‑roots of unity
\[
P = \{ e^{i\pi j/3} : j = 0,1,\ldots,5 \}.
\]
For large $|z|$,
\[
|h(z)| = \frac{|z|}{|z|^6 |1 - z^{-6}|}
= O\!\left(\frac{1}{|z|^{5}}\right),
\]
so $h(z)$ tends to~0 like~$1/|z|^5$ as $|z|\to\infty$.



Fix~$z$ in a compact region~$K$ disjoint from the poles of~$h$.
For sufficiently large lattice points~$\omega\in L$,
the translation $z+\omega$ satisfies $\tfrac12|\omega| < |z+\omega| < \tfrac32|\omega|$.
Hence
\[
|h(z+\omega)| \le C\,\frac{1}{|\omega|^{5}}
\]
for some constant~$C$ depending only on~$K$.
Consequently,
\[
\sum_{\omega\in L\setminus\{0\}} |h(z+\omega)|
\le C \sum_{\omega\in L\setminus\{0\}} \frac{1}{|\omega|^{5}}.
\]
By part~(a), the lattice sum $\sum_{\omega\neq 0} 1/|\omega|^{k}$ converges for every $k\ge3$.  
Hence the above bound shows that the series for $f(z)$ converges absolutely
and uniformly on every compact region avoiding the set of poles.



Because the series converges absolutely and locally uniformly away from the poles,
finitely many terms can be removed if a compact set contains some poles of~$h$.
This ensures that $f(z)$ defines a meromorphic function on~$\mathbb{C}$ with possible poles
only at the points $p-\omega$ where $p\in P$ and $\omega\in L$.
Every pole is simple because each comes from a simple pole of~$h$.



Let $\omega_0\in L$.  Then
\[
f(z+\omega_0)
= \sum_{\omega\in L} h((z+\omega_0)+\omega).
\]
Since translation by $\omega_0$ gives a bijection
$L\to L$, $\omega\mapsto\omega+\omega_0$,
the summation index may be re‑labelled:
\[
f(z+\omega_0)
= \sum_{\omega'\in L} h(z+\omega') = f(z).
\]
Hence $f$ is invariant under translation by $\omega_1$ and $\omega_2$,
so $f$ is elliptic with respect to~$L$.

\subsection*{(c)}




Each summand
\[
h_\omega(z)=\frac{z+\omega}{(z+\omega)^6-1}
\]
has simple poles where $(z+\omega)^6=1$, i.e.
\[
z+\omega = \zeta_j, \quad 
\zeta_j=e^{i\pi j/3},\ j=0,1,\dots,5.
\]
Hence $z=\zeta_j-\omega$, and therefore the poles of $f$
are the lattice–periodic points
\[
P = \{\zeta_j-\omega : j=0,\dots,5,\ \omega\in L\}.
\]
Modulo $L$, there are exactly six distinct poles on the torus~$X$.

Near one of them, say $z_0=\zeta_j-\omega$, write $z=z_0+\delta$.
Then $(z+\omega)^6-1=6\zeta_j^5\delta+O(\delta^2)$ and
\[
h_\omega(z)
=\frac{\zeta_j+\delta}{(\zeta_j+\delta)^6-1}
=\frac{1}{6\zeta_j^4}\frac{1}{z-z_0}+O(1).
\]
Therefore the pole is \emph{simple}.  
Since all other terms are holomorphic near~$z_0$,
each pole of~$f$ is simple.



By the general properties of elliptic functions,
if $Z(f|_\Pi)$ and $P(f|_\Pi)$ denote the zeros and poles of~$f$
inside a fundamental domain~$\Pi$ (with no zeros or poles on~$\partial\Pi$), then
\[
\sum_{a_k\in Z(f|_\Pi)} \operatorname{ord}_{a_k}(f)
+ 
\sum_{b_\ell\in P(f|_\Pi)} \operatorname{ord}_{b_\ell}(f) = 0.
\]
Hence on~$X=\mathbb{C}/L$, the total number of zeros (counted with multiplicity)
equals the total number of poles (counted with multiplicity).

Since $f$ has six poles and all are simple,
we obtain
\[
\text{Total zero order} = 6.
\]
Therefore, as a meromorphic function on the elliptic curve~$X$,
$f$ has exactly six zeros (counting multiplicity).

\section{Q3}







Fix $\omega\in L^{*}$ and set
\[
g_{\omega}(z)
= \frac{1}{z+\omega}
+\frac{z}{\omega^{2}}
-\frac{1}{\omega}.
\]
For $|z|\le R$ and $|\omega|>2R$, we have $|z/\omega|<\tfrac12$,
so we can expand $\frac{1}{z+\omega}$ as a geometric series:
\[
\frac{1}{z+\omega}
= \frac{1}{\omega}\frac{1}{1+z/\omega}
= \frac{1}{\omega}
\left(1-\frac{z}{\omega}+\frac{z^{2}}{\omega^{2}}
-\frac{z^{3}}{\omega^{3}}+\cdots\right).
\]
Substitute this expansion into $g_{\omega}(z)$:
\[
\begin{aligned}
	g_{\omega}(z)
	&= \left(
	\frac{1}{\omega}
	-\frac{z}{\omega^{2}}
	+\frac{z^{2}}{\omega^{3}}
	-\frac{z^{3}}{\omega^{4}}
	+\cdots
	\right)
	+\frac{z}{\omega^{2}}-\frac{1}{\omega}  \\[4pt]
	&=  \frac{z^{2}}{\omega^{3}}
	-\frac{z^{3}}{\omega^{4}}
	+\frac{z^{4}}{\omega^{5}}
	-\cdots .
\end{aligned}
\]
Hence
\[
|g_{\omega}(z)| \le C_{R}\,\frac{1}{|\omega|^{3}} 
\qquad(|z|\le R)
\]
for some constant $C_{R}>0$ depending only on~$R$.


By the estimate above and the result of (2a),
\[
\sum_{\omega\in L^{*}}|g_{\omega}(z)|
\le C_{R}\sum_{\omega\in L^{*}}\frac{1}{|\omega|^{3}}
< \infty.
\]
Hence the series
\[
\sum_{\omega\in L^{*}} g_{\omega}(z)
\]
converges absolutely and uniformly on $D(R)$.
By the Weierstrass~$M$–test it therefore converges to a holomorphic
function on~$D(R)$.


Let 
\[
L_{R}=\{\omega\in L^{*}:|\omega|\le2R\}, \qquad
L_{R}^{c}=L^{*}\setminus L_{R}.
\]
Then we may rewrite $\zeta$ as
\[
\begin{aligned}
	\zeta(z)
	&= \frac{1}{z}
	+ \sum_{\omega\in L_{R}}
	\left(
	\frac{1}{z+\omega}
	+\frac{z}{\omega^{2}}
	-\frac{1}{\omega}
	\right)
	+ \sum_{\omega\in L_{R}^{c}}
	\left(
	\frac{1}{z+\omega}
	+\frac{z}{\omega^{2}}
	-\frac{1}{\omega}
	\right).
\end{aligned}
\]


\begin{itemize}
	\item The first two terms,
	$\displaystyle \frac{1}{z}+\sum_{\omega\in L_{R}}(\cdots)$,
	form a \emph{finite sum of meromorphic functions} on $D(R)$,
	because only finitely many poles ($z=-\omega$ with $|\omega|\le2R$)
	lie in that disk.
	\item The second (infinite) sum is a uniformly convergent
	series of holomorphic functions on $D(R)$, hence represents
	a holomorphic function.
\end{itemize}

\section{Q4}

\subsection*{(a)}



We aim to show that for every $\omega_{0}\in L$,
\[
\boxed{\wp(z+\omega_{0})=\wp(z).}
\]


Compute
\[
\begin{aligned}
	\wp(z+\omega_{0})
	&= \frac{1}{(z+\omega_{0})^{2}}
	+ \sum_{\omega\in L'}
	\left(
	\frac{1}{(z+\omega_{0}+\omega)^{2}}
	-\frac{1}{\omega^{2}}
	\right).
\end{aligned}
\]
Because $L$ is a lattice, adding $\omega_{0}$ to every element of $L$
simply permutes its points.  
That is, the map $\omega\mapsto \omega+\omega_{0}$ is a bijection of $L$
onto itself, and of $L'$ onto $L'\cup\{\,-\omega_{0}\,\}\setminus\{\,\omega_{0}\,\}$.

Let $\omega'=\omega+\omega_{0}$.  Then
\[
\wp(z+\omega_{0})
= \frac{1}{(z+\omega_{0})^{2}}
+ \sum_{\omega'\in L'}
\left(
\frac{1}{(z+\omega')^{2}}
-\frac{1}{(\omega'-\omega_{0})^{2}}
\right)
+ \Bigl[
\frac{1}{z^{2}}-\frac{1}{(z+\omega_{0})^{2}}
\Bigr],
\]
where the bracketed correction accounts for the terms corresponding to
$\omega'=-\omega_{0}$ (missing) and $\omega'=\omega_{0}$ (extra).

The bracket cancels the term $\frac{1}{(z+\omega_{0})^{2}}$, leaving
\[
\wp(z+\omega_{0})
= \frac{1}{z^{2}}
+ \sum_{\omega'\in L'}
\left(
\frac{1}{(z+\omega')^{2}}
-\frac{1}{(\omega'-\omega_{0})^{2}}
\right).
\]


Since the set $\{\omega'-\omega_{0}:\omega'\in L'\}$ equals $L'$
(it’s just a translation of the lattice by~$\omega_{0}$),
we can rename $\omega' \mapsto \omega$ and obtain
\[
\wp(z+\omega_{0})
= \frac{1}{z^{2}}
+ \sum_{\omega\in L'}
\left(
\frac{1}{(z+\omega)^{2}}-\frac{1}{\omega^{2}}
\right)
= \wp(z).
\]

\subsection*{(b)}





Every pole of $\wp$ is double, hence near any lattice point $\omega \in L$,
\[
\wp(z) \sim \frac{1}{(z - \omega)^2}, \qquad
\wp'(z) \sim -\frac{2}{(z - \omega)^3}.
\]
Therefore $f(z)$ has poles of order $6$ at the lattice points.
Since $g(z)$ is a cubic polynomial in $\wp(z)$ and each $\wp(z)$
has poles of order $2$, $g$ likewise has poles of order $6$ and no others.
Thus $f$ and $g$ have exactly the same poles (counting multiplicities).


Because $\wp(z)$ is even, we have $\wp'(-z) = -\wp'(z)$, so the zeros of $\wp'$
are the half-periods
\[
\frac{\omega_1}{2}, \quad
\frac{\omega_2}{2}, \quad
\frac{\omega_1+\omega_2}{2},
\]
and these are simple zeros.
Hence $f(z) = (\wp'(z))^2$ has double zeros at these three points.

At these same points, $\wp'(z) = 0$, so $\wp(z)$ takes the values
$\wp(\tfrac{\omega_1}{2})$, $\wp(\tfrac{\omega_2}{2})$,
and $\wp(\tfrac{\omega_1+\omega_2}{2})$.
By definition of $g(z)$, those are precisely the zeros of $g$,
each also of multiplicity $2$.
Thus $f$ and $g$ share the same zeros (with multiplicities).


Consider
\[
h(z) = \frac{f(z)}{g(z)}
= \frac{(\wp'(z))^2}
{(\wp(z) - \wp(\tfrac{\omega_1}{2}))
	(\wp(z) - \wp(\tfrac{\omega_2}{2}))
	(\wp(z) - \wp(\tfrac{\omega_1+\omega_2}{2}))}.
\]
Since $f$ and $g$ have the same zeros and poles,
$h(z)$ has no poles anywhere and is elliptic.
A holomorphic elliptic function must be constant, so $h(z) = C$.


Near $z = 0$ we have
\[
\wp(z) = \frac{1}{z^2} + O(z^2), \qquad
\wp'(z) = -\frac{2}{z^3} + O(z),
\]
so
\[
f(z) = (\wp'(z))^2 = \frac{4}{z^6} + O(z^{-2}), \qquad
g(z) = (\wp(z))^3 + O(z^{-2}) = \frac{1}{z^6} + O(z^{-2}).
\]
Hence $C = \lim_{z \to 0} \frac{f(z)}{g(z)} = 4$.


Thus
\[
(\wp'(z))^2
= 4
(\wp(z) - \wp(\tfrac{\omega_1}{2}))
(\wp(z) - \wp(\tfrac{\omega_2}{2}))
(\wp(z) - \wp(\tfrac{\omega_1+\omega_2}{2})).
\]




\subsection*{(c)}

Assume that $\wp$ satisfies
\[
(\wp'(z))^{2} = 4\wp(z)^{3} + a\wp(z) + b,
\]
for some constants $a,b$ depending on the lattice $L$.
Let
\[
e_1 = \wp\!\left(\frac{\omega_1}{2}\right), \quad
e_2 = \wp\!\left(\frac{\omega_2}{2}\right), \quad
e_3 = \wp\!\left(\frac{\omega_1+\omega_2}{2}\right).
\]
From part (b),
\[
(\wp'(z))^{2}
= 4(\wp(z)-e_1)(\wp(z)-e_2)(\wp(z)-e_3).
\]
Expanding the right-hand side gives
\[
(\wp'(z))^{2}
= 4\wp(z)^3 - 4(e_1+e_2+e_3)\wp(z)^2
+ 4(e_1e_2 + e_2e_3 + e_3e_1)\wp(z)
- 4e_1e_2e_3.
\]
Comparing this with the assumed form
\[
(\wp'(z))^{2} = 4\wp(z)^3 + a\wp(z) + b,
\]
we find that the coefficients must satisfy
\[
\begin{cases}
	-4(e_1+e_2+e_3) = 0,\\
	4(e_1e_2 + e_2e_3 + e_3e_1) = a,\\
	-4e_1e_2e_3 = b.
\end{cases}
\]
The first equation yields immediately
\[
e_1 + e_2 + e_3 = 0.
\]
Hence
\[
\boxed{
	\wp\!\left(\frac{\omega_1}{2}\right)
	+ \wp\!\left(\frac{\omega_2}{2}\right)
	+ \wp\!\left(\frac{\omega_1+\omega_2}{2}\right) = 0.}
\]
	
\end{document}